\documentclass{report}
\usepackage{luatexja} % LuaTeXで日本語を使うためのパッケージ
\usepackage{luatexja-fontspec} % LuaTeX用の日本語フォント設定

% --- 数学関連 ---
\usepackage{amsmath, amssymb, amsfonts, mathtools, bm, amsthm} % 基本的な数学パッケージ
\usepackage{type1cm, upgreek} % 数式フォントとギリシャ文字k
\usepackage{physics, mhchem} % 物理や化学の記号や式の表記を簡単にする

% --- 表関連 ---
\usepackage{multirow, longtable, tabularx, array, colortbl, dcolumn, diagbox} % 表のレイアウトを柔軟にする
\usepackage{tablefootnote, truthtable} % 表中に注釈を追加、真理値表
\usepackage{tabularray} % 高度な表組みレイアウト

% --- グラフィック関連 ---
\usepackage{tikz, graphicx} % 図の描画と画像の挿入
% \usepackage{background} % ウォーターマークの設定
\usepackage{caption, subcaption} % 図や表のキャプション設定
\usepackage{float, here} % 図や表の位置指定

% --- レイアウトとページ設定 ---
\usepackage{fancyhdr} % ページヘッダー、フッター、余白の設定
\usepackage[top = 20truemm, bottom = 20truemm, left = 20truemm, right = 20truemm]{geometry}
\usepackage{fancybox, ascmac} % ボックスのデザイン

% --- 色とスタイル ---
\usepackage{xcolor, color, colortbl, tcolorbox} % 色とカラーボックス
\usepackage{listings, jvlisting} % コードの色付けとフォーマット

% --- 参考文献関連 ---
\usepackage{biblatex, usebib} % 参考文献の管理と挿入
\usepackage{url, hyperref} % URLとリンクの設定

% --- その他の便利なパッケージ ---
\usepackage{footmisc} % 脚注のカスタマイズ
\usepackage{multicol} % 複数段組
\usepackage{comment} % コメントアウトの拡張
\usepackage{siunitx} % 単位の表記
\usepackage{docmute}
% \usepackage{appendix}
% --- tcolorboxとtikzの設定 ---
\tcbuselibrary{theorems, breakable} % 定理のボックスと改ページ設定
\usetikzlibrary{decorations.markings, arrows.meta, calc} % tikzの装飾や矢印の設定

% --- 定理スタイルと数式設定 ---
\theoremstyle{definition} % 定義スタイル
\numberwithin{equation}{section} % 式番号をサブセクション単位でリセット

% --- hyperrefの設定 ---
\hypersetup{
  setpagesize = false,
  bookmarks = true,
  bookmarksdepth = tocdepth,
  bookmarksnumbered = true,
  colorlinks = false,
  pdftitle = {}, % PDFタイトル
  pdfsubject = {}, % PDFサブジェクト
  pdfauthor = {}, % PDF作者
  pdfkeywords = {} % PDFキーワード
}

% --- siunitxの設定 ---
\sisetup{
  table-format = 1.5, % 小数点以下の桁数
  table-number-alignment = center, % 数値の中央揃え
}


% --- その他の設定 ---
\allowdisplaybreaks % 数式の途中改ページ許可
\newcolumntype{t}{!{\vrule width 0.1pt}} % 新しいカラムタイプ
\newcolumntype{b}{!{\vrule width 1.5pt}} % 太いカラム
\UseTblrLibrary{amsmath, booktabs, counter, diagbox, functional, hook, html, nameref, siunitx, varwidth, zref} % tabularrayのライブラリ
\setlength{\columnseprule}{0.4pt} % カラム区切り線の太さ
\captionsetup[figure]{font = bf} % 図のキャプションの太字設定
\captionsetup[table]{font = bf} % 表のキャプションの太字設定
\captionsetup[lstlisting]{font = bf} % コードのキャプションの太字設定
\captionsetup[subfigure]{font = bf, labelformat = simple} % サブ図のキャプション設定
\setcounter{secnumdepth}{4} % セクションの深さ設定
\newcolumntype{d}{D{.}{.}{5}} % 数値のカラム
\newcolumntype{M}[1]{>{\centering\arraybackslash}m{#1}} % センター揃えのカラム
\DeclareMathOperator{\diag}{diag}
\everymath{\displaystyle} % 数式のスタイル
\newcommand{\inner}[2]{\left\langle #1, #2 \right\rangle}
\renewcommand{\figurename}{図}
\renewcommand{\i}{\mathrm{i}} % 複素数単位i
\renewcommand{\laplacian}{\grad^2} % ラプラシアンの記号
\renewcommand{\thesubfigure}{(\alph{subfigure})} % サブ図の番号形式
\newcommand{\m}[3]{\multicolumn{#1}{#2}{#3}} % マルチカラムのショートカット
\renewcommand{\r}[1]{\mathrm{#1}} % mathrmのショートカット
\newcommand{\e}{\mathrm{e}} % 自然対数の底e
\newcommand{\Ef}{E_{\mathrm{F}}} % フェルミエネルギー
\renewcommand{\c}{\si{\degreeCelsius}} % 摂氏記号
\renewcommand{\d}{\r{d}} % d記号
\renewcommand{\t}[1]{\texttt{#1}} % タイプライタフォント
\newcommand{\kb}{k_{\mathrm{B}}} % ボルツマン定数
% \renewcommand{\phi}{\varphi} % ϕをφに変更
\renewcommand{\epsilon}{\varepsilon}
\newcommand{\fullref}[1]{\textbf{\ref{#1} \nameref{#1}}}
\newcommand{\reff}[1]{\textbf{図\ref{#1}}} % 図参照のショートカット
\newcommand{\reft}[1]{\textbf{表\ref{#1}}} % 表参照のショートカット
\newcommand{\refe}[1]{\textbf{式\eqref{#1}}} % 式参照のショートカット
\newcommand{\refp}[1]{\textbf{コード\ref{#1}}} % コード参照のショートカット
\renewcommand{\lstlistingname}{コード} % コードリストの名前
\renewcommand{\theequation}{\thesection.\arabic{equation}} % 式番号の形式
\renewcommand{\footrulewidth}{0.4pt} % フッターの線
\newcommand{\mar}[1]{\textcircled{\scriptsize #1}} % 丸囲み文字
\newcommand{\combination}[2]{{}_{#1} \mathrm{C}_{#2}} % 組み合わせ
\newcommand{\thline}{\noalign{\hrule height 0.1pt}} % 細い横線
\newcommand{\bhline}{\noalign{\hrule height 1.5pt}} % 太い横線

% --- カスタム色定義 ---
\definecolor{burgundy}{rgb}{0.5, 0.0, 0.13} % バーガンディ色
\definecolor{charcoal}{rgb}{0.21, 0.27, 0.31} % チャコール色
\definecolor{forest}{rgb}{0.0, 0.35, 0} % 森の緑色

% --- カスタム定理環境の定義 ---
\newtcbtheorem[number within = chapter]{myexc}{練習問題}{
  fonttitle = \gtfamily\sffamily\bfseries\upshape,
  colframe = forest,
  colback = forest!2!white,
  rightrule = 1pt,
  leftrule = 1pt,
  bottomrule = 2pt,
  colbacktitle = forest,
  theorem style = standard,
  breakable,
  arc = 0pt,
}{exc-ref}
\newtcbtheorem[number within = chapter]{myprop}{命題}{
  fonttitle = \gtfamily\sffamily\bfseries\upshape,
  colframe = blue!50!black,
  colback = blue!50!black!2!white,
  rightrule = 1pt,
  leftrule = 1pt,
  bottomrule = 2pt,
  colbacktitle = blue!50!black,
  theorem style = standard,
  breakable,
  arc = 0pt
}{proposition-ref}
\newtcbtheorem[number within = chapter]{myrem}{注意}{
  fonttitle = \gtfamily\sffamily\bfseries\upshape,
  colframe = yellow!20!black,
  colback = yellow!50,
  rightrule = 1pt,
  leftrule = 1pt,
  bottomrule = 2pt,
  colbacktitle = yellow!20!black,
  theorem style = standard,
  breakable,
  arc = 0pt
}{remark-ref}
\newtcbtheorem[number within = chapter]{myex}{例題}{
  fonttitle = \gtfamily\sffamily\bfseries\upshape,
  colframe = black,
  colback = white,
  rightrule = 1pt,
  leftrule = 1pt,
  bottomrule = 2pt,
  colbacktitle = black,
  theorem style = standard,
  breakable,
  arc = 0pt
}{example-ref}
\newtcbtheorem[number within = chapter]{exc}{Requirement}{myexc}{exc-ref}
\newcommand{\rqref}[1]{{\bfseries\sffamily 練習問題 \ref{exc-ref:#1}}}
\newtcbtheorem[number within = chapter]{definition}{Definition}{mydef}{definition-ref}
\newcommand{\dfref}[1]{{\bfseries\sffamily 定義 \ref{definition-ref:#1}}}
\newtcbtheorem[number within = chapter]{prop}{命題}{myprop}{proposition-ref}
\newcommand{\prref}[1]{{\bfseries\sffamily 命題 \ref{proposition-ref:#1}}}
\newtcbtheorem[number within = chapter]{rem}{注意}{myrem}{remark-ref}
\newcommand{\rmref}[1]{{\bfseries\sffamily 注意 \ref{remark-ref:#1}}}
\newtcbtheorem[number within = chapter]{ex}{例題}{myex}{example-ref}
\newcommand{\exref}[1]{{\bfseries\sffamily 例題 \ref{example-ref:#1}}}
% --- 再定義コマンド ---
% \mathtoolsset{showonlyrefs=true} % 必要な式番号のみ表示
\pagestyle{fancy} % ヘッダー・フッターのスタイル設定
% \chead{応用量子物性講義ノート} % 中央ヘッダー
% \rhead{}
\fancyhead[R]{\rightmark}
\renewcommand{\subsectionmark}[1]{\markright{\thesubsection\ #1}}
\cfoot{\thepage} % 中央フッターにページ番号
\lhead{}
\rfoot{Haruki Aoki and Hiroki Fukuhara} % 右フッターに名前
\setcounter{tocdepth}{4} % 目次の深さ
\makeatletter
\@addtoreset{equation}{section} % セクwションごとに式番号をリセット
\makeatother

% --- メタ情報 ---
\title{量子光学・量子情報科学ノート}
\date{更新日: \today}
\author{Haruki Aoki and Hiroki Fukuhara}

\begin{document}
  本節では,1次元調和振動子モデルでハミルトニアンが書けるときの波動函数の表示を求める.
  波動函数とは,Schr\"odinger方程式,
  \begin{align}
    \hat{H}\ket{\psi} = E\ket{\psi}\label{schrodinger-equation}
  \end{align}
  を満たす$\ket{\psi}$について,
  \begin{align}
    \psi(x) \coloneqq \braket{x}{\psi}
  \end{align}
  となるように(一般化)座標$x$へ射影したものである.
  \refe{schrodinger-equation}に対して$\bra{x}$を左から書ければ,
  \begin{align}
    \mel**{x}{\hat{H}}{\psi} = E\psi(x)\label{bra-x-from-left}
  \end{align}
  となるのだから,左辺を計算して$\psi(x)$に演算子がかかる形に変形すれば,波動函数を求めることができる.
  本ノートにおいて,$\hat{\cdot}$を演算子として,その固有値を$\cdot$,固有ベクトル(固有函数)を$\ket{\cdot}$と書く.
  \begin{align}
    \hat{x}\ket{x} &= x\ket{x} \\ 
    \hat{p}\ket{x} &= p\ket{x}
  \end{align}
  である.また,$\hat{x}$や$\hat{p}$は物理量であり,Hermite演算子だからその固有ベクトルは,
  \begin{align}
    \braket{x'}{x} &= \delta\qty(x' - x) \\ 
    \braket{p'}{p} &= \delta\qty(p' - p)
  \end{align}
  と規格化してあり,
  \begin{align}
    \int\dd{x}\ketbra{x}{x} &= \hat{1}
    \int\dd{p}\ketbra{p}{p} &= \hat{1}
  \end{align}
  が成立する.なお,特に断らない限り積分範囲は$-\infty$から$\infty$である.
  \subsection{ハミルトニアン}
    古典的な1次元調和振動子のハミルトニアン$H$は,
    \begin{align}
      H = \frac{1}{2}m\omega^2x^2 + \frac{1}{2m}p^2
    \end{align}
    である.ただし,質量を$m$,固有角周波数を$\omega$,座標を$x$,運動量を$p$とした.
    $x$と$p$は正準共役な変数の組であるから,$x\to \hat{x}$,$p \to \hat{p}$として,
    \begin{align}
      \hat{H} &= \frac{1}{2}m\omega^2\hat{x}^2 + \frac{1}{2m}\hat{p}^2 \label{harmonic-oscillator-hamiltonian}\\ 
      &= \hbar\omega\qty(\frac{m\omega}{2\hbar}\hat{x}^2 + \frac{1}{2m\hbar\omega}\hat{p}^2) \\ 
      &= \hbar\omega\qty[\qty(\sqrt{\frac{m\omega}{2\hbar}}\hat{x} - \i\sqrt{\frac{1}{2m\hbar\omega}}\hat{p})\qty(\sqrt{\frac{m\omega}{2\hbar}}\hat{x} + \i\sqrt{\frac{1}{2m\hbar\omega}}\hat{p}) - \i\sqrt{\frac{m\omega}{2\hbar}}\sqrt{\frac{1}{2m\hbar\omega}}\qty[\hat{x}, \hat{p}]] \\ 
      &= \hbar\omega\qty(\hat{a}^{\dag}\hat{a} + \frac{1}{2})
    \end{align}
    となる.ただし,
    \begin{align}
      \hat{a}^{\dag} &\coloneqq \qty(\sqrt{\frac{m\omega}{2\hbar}}\hat{x} - \i\sqrt{\frac{1}{2m\hbar\omega}}\hat{p}) \\ 
      \hat{a} &\coloneqq \qty(\sqrt{\frac{m\omega}{2\hbar}}\hat{x} + \i\sqrt{\frac{1}{2m\hbar\omega}}\hat{p})
    \end{align}
    と定義した.
  \subsection{Hermite多項式}
    以降の議論で用いるために,特殊函数の1つであるHermite多項式を紹介しておこう.
    Hermite多項式はStrum-Liouville演算子のうちの1つの演算子の固有函数であり,実数全体で定義された実数函数$H_n(s)$に対して,
    \begin{align}
      \qty(\dv[2]{s} - 2s\dv{s} + 2n)H_n(s) = 0
    \end{align}
    なる$H_n(s)$である.
    なお,$H_n(s)$は適当な回数だけ微分可能であるとする.
    また,$H_n(s)$が張る空間$V$の内積は,$f, g\in V$として,
    \begin{align}
      \inner{f}{g} \coloneqq \int_{-\infty}^{\infty}\dd{s}f(s)g(s)\e^{-s^2}
    \end{align}
    である\footnote{
      Strum-Liouville演算子の形,
      \begin{align*}
        \frac{1}{\rho(x)}\qty[\dv{x}\qty{p(x)\dv{x}} + q(x)]
      \end{align*}
      と,Strum-Liouville演算子の固有函数が張る空間の内積が,
      \begin{align*}
        \int_{a}^{b} f^*(x)g(x)\rho(x)\dd{x}
      \end{align*}
      と書けることを思い出せば,内積に$\e^{-s^2}$なる重み函数が入ることは自然なことである.
    }.
    Hermite多項式は,適切に境界条件が設定された(Hermite性のある)Strum-Liouville演算子の固有函数であり,
    そのような演算子の固有函数は直交基底となり,完全系を成すことが知られていて,実際,
    \begin{align}
      \int_{-\infty}^{\infty}H_m(s)H_n(s)\e^{-s^2}\dd{s} = \sqrt{\pi}2^nn!\delta_n^m
    \end{align}
    のように直交する.
  \subsection{波動函数を用いたSchr\"odinger方程式}
    以下では,波動函数を用いたchr\"odinger方程式である,
    \begin{align}
      \qty(-\frac{\hbar^2}{2m}\dv[2]{x} + \frac{1}{2}m\omega^2x^2)\psi(x) = E\psi(x)
    \end{align}
    を得る.
    \par
    \refe{bra-x-from-left}に\refe{harmonic-oscillator-hamiltonian}で示した$\hat{H}$の表式を代入して,
    \begin{align}
      \frac{1}{2}m\omega^2\mel**{x}{\hat{x}^2}{\psi} + \frac{1}{2m}\mel**{x}{\hat{p}^2}{\psi} &= E\psi(x) \\ 
      \frac{1}{2}m\omega^2x^2\psi(x) + \frac{1}{2m}\mel**{x}{\hat{p}^2}{\psi} &= E\psi(x) \label{input-to-schrodinger}
    \end{align}
    となる.
    $\mel**{x}{\hat{p}^2}{\psi}$は以下のレシピで計算できる.
    \begin{enumerate}
      \item $f(x)\dv{x}\delta(x) = -\dv{x}f(x)\delta(x)$
      \item $\mel**{x}{\hat{p}}{\psi} = -\i\hbar\dv{x}\psi(x)$
      \item $\braket{x}{p} = \frac{1}{\sqrt{2\pi p}}\exp\qty(\i\frac{xp}{\hbar})$
      \item $\mel**{x}{\hat{p}^2}{\psi}$の計算
    \end{enumerate}
    $\delta(x)$はデルタ函数であり,積分して初めて意味を持つ函数である.
    \begin{enumerate}
      \item $f(x)\dv{x}\delta(x) = -\dv{x}f(x)\delta(x)$
        左辺を積分して右辺になればよい.ただし,$f(x)$は,
        \begin{align}
          \lim_{\abs{x} \to \infty}f(x) = 0
        \end{align}
        であるとする.
        実際に,
        \begin{align}
          \int_{-\infty}^{\infty}\dd{x}f(x)\dv{x}\delta(x) &= \qty[f(x)\delta(x)]_{-\infty}{\infty} - \int_{-\infty}^{\infty}\dd{x}\dv{x}f(x)\delta(x) \\ 
          &= - \int_{-\infty}^{\infty}\dd{x}\dv{x}f(x)\delta(x)
        \end{align}
        であるから,
        \begin{align}
          f(x)\dv{x}\delta(x) = -\dv{x}f(x)\delta(x)\label{delta-function-diff}
        \end{align}
        である.
      \item $\mel**{x}{\hat{p}}{\psi} = -\i\hbar\dv{x}\psi(x)$
        $\mel**{x}{\qty[\hat{x}, \hat{p}]}{x'}$を2種類の方法で計算する.
        まず,愚直に計算すると,
        \begin{align}
          \mel**{x}{\qty[\hat{x}, \hat{p}]}{x'} &= \mel**{x}{\hat{x}\hat{p} - \hat{p}\hat{x}}{x'} \\ 
          &= x\mel**{x}{\hat{p}}{x'} - x'\mel**{x}{\hat{p}}{x'} \\ 
          &= \qty(x - x')\mel**{x}{\hat{p}}{x'}\label{xxpxprime-1}
        \end{align}
        である.一方,$\qty[\hat{x}, \hat{p}] = \i\hbar$を用いれば,
        \begin{align}
          \mel**{x}{\qty[\hat{x}, \hat{p}]}{x'} &= \i\hbar\braket{x}{x'} \\ 
          &= \i\hbar\delta(x - x')\label{xxpxprime-2}
        \end{align}
        2つの方法で計算した$\mel**{x}{\qty[\hat{x}, \hat{p}]}{x'}$である\refe{xxpxprime-1}と\refe{xxpxprime-2}を等号で結んで,
        \refe{delta-function-diff}で示したデルタ函数の微分を用いて表現すれば,
        \begin{align}
          \mel**{x}{\hat{p}}{x'} &= \i\hbar\frac{\delta(x - x')}{x - x'} \\ 
          &= -\i\hbar\dv{\qty(x - x')}\delta(x - x') \\ 
          &= \i\hbar\dv{x'}\delta(x - x')\label{xpxrime}
        \end{align}
        となる.
        \par
        さて,$\mel**{x}{\hat{p}}{\psi}$を計算しよう.
        $\ket{x}$の完全性と,\refe{xpxrime}で示した関係を用いれば,
        \begin{align}
          \mel**{x}{\hat{p}}{\psi} &= \mel**{x}{\hat{p}\hat{1}}{\psi} \\ 
          &= \bra{x}\hat{p}\int\dd{x'}\ketbra{x'}{x'}\ket{\psi} \\ 
          &= \int\dd{x'}\mel**{x}{\hat{p}}{x'}\braket{x'}{\psi} \\ 
          &= \i\hbar\int\dd{x'}\qty[\dv{x}\delta(x - x')]\phi(x') \\ 
          &= \i\hbar\qty{[\delta(x - x')]_{-\infty}^{\infty} - \int\dd{x'}\dv{x'}\phi(x')\delta(x - x')} \\ 
          &= -\i\hbar\dv{x}\phi(x)\label{psix-diff}
        \end{align}
        を得る.
      \item $\braket{x}{p} = \frac{1}{\sqrt{2\pi p}}\exp\qty(\i\frac{xp}{\hbar})$
        $\mel**{x}{\hat{p}}{p}$を2種類の方法で計算する.
        まず,愚直に計算すると,
        \begin{align}
          \mel**{x}{\hat{p}}{p} &= p\braket{x}{p} \\ 
          &= pp(x)\label{xpp-1}
        \end{align}
        となる.ただし$p(x)$は$\ket{p}$の$x$への射影である.
        一方,\refe{psix-diff}で示した関係で$\ket{\psi} \to \ket{p}$を用いると,
        \begin{align}
          \mel**{x}{\hat{p}}{\psi} = -\i\hbar\dv{x}p(x)\label{xpp-2}
        \end{align}
        となる.\refe{xpp-1}と\reff{xpp-2}より,
        \begin{align}
          -\i\hbar\dv{x}p(x) &= pp(x) \\ 
          \Rightarrow p(x) &= C\exp\qty(\i\frac{xp}{\hbar})
        \end{align}
        となる.$C$は規格化定数である.
        \par
        さて,$C$を求めるために,$\braket{x}{x'}$を計算すると,
        \begin{align}
          \delta(x - x') &= \braket{x}{x'} \\ 
          &= \bra{x}\int\dd{p}\ketbra{p}{p}\ket{x'} \\ 
          &= \int\dd{p}p(x)p(x') \\ 
          &= \abs{C}^2\int\exp\qty(\i\frac{\i\qty(x - x')p}{\hbar})
        \end{align}
        となる.ところで,デルタ函数のFouirer変換とその逆変換が,
        \begin{align}
          1 &= \int_{-\infty}^{\infty}\dd{t}\delta(t)\e^{-\i\omega t} \\ 
          \delta(t) &= \frac{1}{2\pi}\int\dd{\omega}1\cdot\e^{\i\omega t}\label{delta-fourier-inv}
        \end{align}
        と書けることより,\refe{delta-fourier-inv}において,
        \begin{align}
          \omega &\to \frac{p}{\hbar} \\ 
          t &\to x - x'
        \end{align}
        と変換すれば,
        \begin{align}
          \delta(x - x') = \frac{1}{2\pi\hbar}\int\dd{p}\exp\qty(\i\frac{x - x'}{\hbar}p)
        \end{align}
        となるので,係数を比較して,
        \begin{align}
          \abs{C}^2 &= \frac{1}{2\pi\hbar} \\ 
          \Rightarrow C &= \frac{1}{\sqrt{2\pi\hbar}} 
        \end{align}
        となる.よって,
        \begin{align}
          \braket{x}{p} = \frac{1}{\sqrt{2\pi\hbar}}\exp\qty(\i\frac{xp}{\hbar})\label{braket-xp}
        \end{align}
        となる.
      \item $\mel**{x}{\hat{p}^2}{\psi}$の計算
        さて,いよいよ$\mel**{x}{\hat{p}^2}{\psi}$を計算する道具がそろった.
        \refe{braket-xp}を用いながら計算すると,
        \begin{align}
          \mel**{x}{\hat{p}^2}{\psi} &= \mel**{x}{\hat{p}^2\hat{1}}{\psi} \\
          &= \bra{x}\hat{p}^2\int\dd{p}\ketbra{p}{p}\ket{\psi} \\ 
          &= \int\dd{p}p^2\braket{x}{p}\braket{p}{\psi} \\ 
          &= \int\dd{p}p^2\frac{1}{\sqrt{2\pi\hbar}}\exp\qty(\i\frac{xp}{\hbar})\braket{p}{\psi} \\ 
          &= \int\dd{p}\frac{1}{\sqrt{2\pi\hbar}}\qty{\dv[2]{x}\qty(\frac{\hbar}{\i})^2\braket{x}{p}}\braket{p}{\psi} \\ 
          &= \qty(\frac{\hbar}{\i})^2\int\dd{p}\qty{\dv[2]{x}\braket{x}{p}}\braket{p}{\psi} \\ 
          &= \qty(\frac{\hbar}{\i})^2\dv[2]{x}\bra{x}\qty[\int\dd{p}\ketbra{p}{p}]\ket{\psi} \\ 
          &= \qty(\frac{\hbar}{\i})^2\dv[2]{x}\braket{x}{\psi} \\
          &= \qty(\frac{\hbar}{\i})^2\dv[2]{x}\psi(x)\label{before-schrodinger-oscillator}
        \end{align}
        となる.
    \end{enumerate}
    \refe{before-schrodinger-oscillator}を\refe{input-to-schrodinger}に代入すれば,
    \begin{align}
      \frac{1}{2}m\omega^2x^2\psi(x) + \frac{1}{2m}\mel**{x}{\hat{p}^2}{\psi} &= E\psi(x)\psi(x) + \frac{1}{2m}\mel**{x}{\hat{p}^2}{\psi} &= E\psi(x) \\ 
      \Leftrightarrow \frac{1}{2}m\omega^2x^2\psi(x) + \frac{1}{2m}\qty(\frac{\hbar}{\i})^2\dv[2]{x}\psi(x) = E\psi(x) \\ 
      \Leftrightarrow \qty(-\frac{\hbar}{2m}\dv[2]{x} + \frac{1}{2}m\omega^2x^2)\psi(x) = E\psi(x)
    \end{align}
    となる.
  \subsection{Schrodingerの解法}
\end{document}