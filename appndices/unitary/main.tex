\documentclass{report}
\input{../../head.tex}
\begin{document}
  ユニタリ行列は一般に,
  \begin{align}
    U = \e^{\i\Lambda/2}
    \mqty(
      \e^{\i\Psi/2} & 0 \\ 
      0 & \e^{-\i\Psi/2}
    )
    \mqty(
      \cos(\Theta / 2) & \sin(\Theta / 2) \\ 
      -\sin(\Theta / 2) & \cos(\Theta / 2)
    )\mqty(
      \e^{\i\Phi/2} & 0 \\ 
      0 & \e^{-\i\Phi/2}
    )
  \end{align}
  と分解できる.
  具体的に$U$を計算すると,
  \begin{align}
    U &= \e^{\i\Lambda/2}
    \mqty(
      \e^{\i\Psi/2} & 0 \\ 
      0 & \e^{-\i\Psi/2}
    )
    \mqty(
      \cos(\Theta / 2) & \sin(\Theta / 2) \\ 
      -\sin(\Theta / 2) & \cos(\Theta / 2)
    )\mqty(
      \e^{\i\Phi/2} & 0 \\ 
      0 & \e^{-\i\Phi/2}
    ) \\ 
    &= \e^{\i\Lambda/2}\mqty(
      \e^{\i\Psi/2}\cos(\Theta / 2) & \e^{\i\Psi/2}\sin(\Theta / 2) \\ 
      -\e^{-\i\Psi/2}\sin(\Theta / 2) & \e^{-\i\Psi/2}\cos(\Theta / 2)
    )\mqty(
      \e^{\i\Phi/2} & 0 \\ 
      0 & \e^{-\i\Phi/2}
    ) \\ 
    &= \e^{\i\Lambda/2}\mqty(
      \e^{\i(\Psi + \Phi) / 2}\cos(\Theta / 2) & \e^{\i(\Psi - \Phi) / 2}\sin(\Theta / 2) \\ 
      -\e^{-\i(\Psi - \Phi) / 2}\sin(\Theta / 2) & \e^{-\i(\Psi + \Phi) / 2}\cos(\Theta / 2)
    )\label{unitary-representation}
  \end{align} 
  であり,$\alpha = \Psi + \Phi$,$\beta = \Psi - \Phi$とすると,
  \begin{align}
    U &= \e^{\i\Lambda/2}\mqty(
      \e^{\i\alpha / 2}\cos(\Theta / 2) & \e^{\i\beta / 2}\sin(\Theta / 2) \\ 
      -\e^{-\i\beta / 2}\sin(\Theta / 2) & \e^{-\i\alpha / 2}\cos(\Theta / 2)
    ) \\ 
    &= \mqty(
      \e^{\i(\Lambda + \alpha) / 2}\cos(\Theta / 2) & \e^{\i(\Lambda + \beta) / 2}\sin(\Theta / 2) \\ 
      -\e^{\i(\Lambda - \beta) / 2}\sin(\Theta / 2) & \e^{\i(\Lambda - \alpha) / 2}\cos(\Theta / 2)
    ) 
  \end{align}
  と書ける.
  \begin{proof}
    任意$2\times 2$の行列は,実数$r_{ij}$と$\theta_{ij}$を用いて,
    \begin{align}
      M = \mqty(
        r_{11}\e^{\i\theta_{11}} & r_{12}\e^{\i\theta_{12}} \\ 
        r_{21}\e^{\i\theta_{21}} & r_{22}\e^{\i\theta_{22}} \\ 
      )
    \end{align}
    と書けて,
    \begin{align}
      M^{\dag}M &= \mqty(
        r_{11}\e^{-\i\theta_{11}} & r_{21}\e^{-\i\theta_{21}} \\ 
        r_{12}\e^{-\i\theta_{12}} & r_{22}\e^{-\i\theta_{22}} \\ 
      )\mqty(
        r_{11}\e^{\i\theta_{11}} & r_{12}\e^{\i\theta_{12}} \\ 
        r_{21}\e^{\i\theta_{21}} & r_{22}\e^{\i\theta_{22}} \\ 
      ) \\ 
      &= \mqty(
        r_{11}^2 + r_{21}^2 & r_{11}r_{12}\e^{-\i(\theta_{11} - \theta_{12})} + r_{21}r_{22}\e^{-\i(\theta_{21} - \theta_{22})} \\ 
        r_{11}r_{12}\e^{\i(\theta_{11} - \theta_{12})} + r_{21}r_{22}\e^{\i(\theta_{21} - \theta_{22})} & r_{12}^2 + r_{22}^2 \\ 
      ) \\ 
      MM^{\dag} &= \mqty(
        r_{11}\e^{\i\theta_{11}} & r_{12}\e^{\i\theta_{12}} \\ 
        r_{21}\e^{\i\theta_{21}} & r_{22}\e^{\i\theta_{22}} \\ 
      )\mqty(
        r_{11}\e^{-\i\theta_{11}} & r_{21}\e^{-\i\theta_{21}} \\ 
        r_{12}\e^{-\i\theta_{12}} & r_{22}\e^{-\i\theta_{22}} \\ 
      ) \\ 
      &= \mqty(
        r_{11}^2 + r_{12}^2 & r_{11}r_{21}\e^{\i(\theta_{11} - \theta_{21})} + r_{11}r_{22}\e^{\i(\theta_{12} - \theta_{22})} \\ 
        r_{11}r_{21}\e^{-\i(\theta_{11} - \theta_{21})} + r_{12}r_{22}\e^{-\i(\theta_{12} - \theta_{22})} & r_{21}^2 + r_{22}^2 \\ 
      )
    \end{align}
    となる.$M$がユニタリ行列であることの必要十分条件は,
    \begin{align}
      r_{11}^2 + r_{21}^2 = 1 \label{r11-r21}\\ 
      r_{12}^2 + r_{22}^2 = 1 \label{r12-r22}\\ 
      r_{11}^2 + r_{12}^2 = 1 \label{r11-r12}\\ 
      r_{21}^2 + r_{22}^2 = 1 \label{r21-r22}\\ 
      r_{11}r_{12}\e^{\i(\theta_{11} - \theta_{12})} + r_{21}r_{22}\e^{\i(\theta_{21} - \theta_{22})} = 0 \label{angle-1}\\ 
      r_{11}r_{21}\e^{\i(\theta_{11} - \theta_{21})} + r_{11}r_{22}\e^{\i(\theta_{12} - \theta_{22})} = 0 \label{angle-2}
    \end{align}
    である.$M^{\dag}M$や$MM^{\dag}$の非対角成分は複素共役になっていることに注意する.
    \refe{r11-r21}から\refe{r21-r22}を満たすような$r_{ij}$の組は,実数$\Theta$を用いて,
    \begin{align}
      r_{11} = r_{22} = \cos(\Theta / 2) \\ 
      r_{12} = -r_{21} = \sin(\Theta / 2)
    \end{align}
    なるものである.
    また,これらの$r_{ij}$の値を\refe{angle-1}と\refe{angle-2}に代入すると,
    \begin{align}
      \e^{\i(\theta_{11} - \theta_{12})} - \e^{\i(\theta_{21} - \theta_{22})} = 0 \\ 
      -\e^{\i(\theta_{11} - \theta_{21})} + \e^{\i(\theta_{12} - \theta_{22})} = 0
    \end{align}
    が成立する.
    \begin{align}
      \Phi = \theta_{11} - \theta_{12} = \theta_{21} - \theta_{22} \\ 
      \Psi = \theta_{11} - \theta_{21} = \theta_{12} - \theta_{22} \\ 
    \end{align}
    とすると,
    \begin{align}
      \theta_{11} = \frac{\Lambda + \Psi + \Phi}{2} \\ 
      \theta_{12} = \frac{\Lambda + \Psi - \Phi}{2} \\ 
      \theta_{21} = \frac{\Lambda - \Psi + \Phi}{2} \\ 
      \theta_{22} = \frac{\Lambda - \Psi - \Phi}{2}
    \end{align}
    となり,\refe{unitary-representation}を得る.
    つまり,任意のユニタリ行列は\refe{unitary-representation}で書けることが示された.
  \end{proof}
  実際に\refe{unitary-representation}がユニタリ行列であることを確かめる.
  \begin{align}
    U^{\dag}U &= \e^{-\i\Lambda/2}\mqty(
      \e^{-\i\alpha / 2}\cos(\Theta / 2) & -\e^{\i\beta / 2}\sin(\Theta / 2) \\ 
      \e^{-\i\beta / 2}\sin(\Theta / 2) & \e^{\i\alpha / 2}\cos(\Theta / 2)
    )
    \e^{\i\Lambda/2}\mqty(
      \e^{\i\alpha / 2}\cos(\Theta / 2) & \e^{\i\beta / 2}\sin(\Theta / 2) \\ 
      -\e^{-\i\beta / 2}\sin(\Theta / 2) & \e^{-\i\alpha / 2}\cos(\Theta / 2)
    )
    = \mqty(1 & 0 \\ 0 & 1) \\ 
    UU^{\dag} &= \e^{\i\Lambda/2}\mqty(
      \e^{\i\alpha / 2}\cos(\Theta / 2) & \e^{\i\beta / 2}\sin(\Theta / 2) \\ 
      -\e^{-\i\beta / 2}\sin(\Theta / 2) & \e^{-\i\alpha / 2}\cos(\Theta / 2)
    )
    \e^{-\i\Lambda/2}\mqty(
      \e^{-\i\alpha / 2}\cos(\Theta / 2) & -\e^{\i\beta / 2}\sin(\Theta / 2) \\ 
      \e^{-\i\beta / 2}\sin(\Theta / 2) & \e^{\i\alpha / 2}\cos(\Theta / 2)
    )
    = \mqty(1 & 0 \\ 0 & 1)
  \end{align}
  となり,$U$はユニタリ行列であることが分かる.
\end{document}