\documentclass{report}
\input{../../head.tex}
\begin{document}
  \subsection{電磁場のハミルトニアン}
    前節での議論により,系のハミルトニアンは,
    \begin{align}
      \hat{H}_{\r{sys}} = \int\dd[3]{k} \sum_{\sigma = 1}^{2}\frac{\hbar\omega_{\bm{k}}}{2}\qty(\hat{a}_{\bm{k\sigma}}^{\dag}\hat{a}_{\bm{k}\sigma} + \hat{a}_{\bm{k}\sigma}\hat{a}_{\bm{k}\sigma}^{\dag})
    \end{align}
    と書けるのであった.
    以下では,簡単のために,1方向成分・シングルモードの波を考える.
    \begin{align}
      \hat{H}_{\r{sys}} &= \frac{\hbar\omega}{2}\qty(\hat{a}^{\dag}\hat{a} + \hat{a}\hat{a}^{\dag}) \\ 
      &= \hbar\omega\qty(\hat{a}^{\dag}\hat{a} + \frac{1}{2})
    \end{align}
    と書ける.
    屈折率が$n$の物質中では\footnote{謎である.屈折率により波動は変化しないはずである.},
    \begin{align}
      \hat{H}_{n, \r{sys}} = \frac{\hbar\omega}{n}\qty(\hat{a}^{\dag}\hat{a} + \frac{1}{2})
    \end{align}
    と書ける.
  \subsection{ビームスプリッタ}
    2入力2出力のビームスプリッタを考える.
    $E_1$と$E_2$の電場が入射して,$E'_1$と$E'_2$が出力されるとする.
    古典的に考えると,
    \begin{align}
      \mqty(E'_1 \\ E'_2) = \mqty(\xmat*{B}{2}{2})\mqty(E_1 \\ E_2)
    \end{align}
    と書ける.
    このまま電場演算子を中心に議論を勧めることはいささか冗長である.
    なぜならば,$\hat{a}_1$と$\hat{a}_1^{\dag}$は複素共役の関係にあるのだから,片方が定まれば自然ともう片方が定まるからだ.
    よって,
    \begin{align}
      \mqty(\hat{a}_1' \\ \hat{a}_2') = \mqty(\xmat*{B}{2}{2})\mqty(\hat{a}_1 \\ \hat{a}_2)
    \end{align}
    と書ける.
    $B$はビームスプリッタ行列という.
    光子数が保存することから,
    \begin{align}
      \hat{a}_1^{\dag}\hat{a}_1 + \hat{a}_2^{\dag}\hat{a}_2 &= \hat{a}_1'^{\dag}\hat{a}_1' + \hat{a}_2'^{\dag}\hat{a}_2' \\
      &= \mqty(B_{11}\hat{a}_1 + B_{12}\hat{a}_2)^{\dag}\mqty(B_{11}\hat{a}_1 + B_{12}\hat{a}_2) + \mqty(B_{21}\hat{a}_1 + B_{22}\hat{a}_2)^{\dag}\mqty(B_{21}\hat{a}_1 + B_{22}\hat{a}_2) \\ 
      &= \mqty(B_{11}^*\hat{a}_1^{\dag} + B_{12}^*\hat{a}_2^{\dag})\mqty(B_{11}\hat{a}_1 + B_{12}\hat{a}_2) + \mqty(B_{21}^*\hat{a}_1^{\dag} + B_{22}^*\hat{a}_2^{\dag})\mqty(B_{21}\hat{a}_1 + B_{22}\hat{a}_2) \\ 
      &= \mqty(\abs{B_{11}}^2 + \abs{B_{21}}^2)\hat{a}_1^{\dag}\hat{a}_1 + \mqty(\abs{B_{12}}^2 + \abs{B_{22}}^2)\hat{a}_2^{\dag}\hat{a}_2 + \mqty(B_{11}^*B_{12} + B_{21}^*B_{22})\hat{a}_1^{\dag}\hat{a}_2 + \qty(B_{12}^*B_{11} + B_{21}^*B_{21})\hat{a}_2^{\dag}\hat{a}_1 \\ 
      &= \mqty(\abs{B_{11}}^2 + \abs{B_{21}}^2)\hat{a}_1^{\dag}\hat{a}_1 + \mqty(\abs{B_{12}}^2 + \abs{B_{22}}^2)\hat{a}_2^{\dag}\hat{a}_2 + \mqty(B_{11}^*B_{12} + B_{21}^*B_{22})\hat{a}_1^{\dag}\hat{a}_2 + \mqty(B_{11}^*B_{12} + B_{21}^*B_{22})^*\hat{a}_2^{\dag}\hat{a}_1
    \end{align}
    となり,
    \begin{align}
      \begin{dcases}
        \abs{B_{11}}^2 + \abs{B_{21}}^2 = \abs{B_{12}}^2 + \abs{B_{22}}^2 = 1 \\ 
        B_{11}^*B_{12} + B_{21}^*B_{22} = 0\\ 
      \end{dcases} \\ 
      \Leftrightarrow 
      B^{\dag}B = \mqty(B_{11}^* & B_{21}^* \\ B_{12}^* & B_{22}^*)\mqty(B_{11} & B_{12} \\ B_{21} & B_{22}) = \mqty(1 & 0 \\ 0 & 1)
    \end{align}
    となればよい.
    つまり,ビームスプリッタ行列$B$がユニタリ行列であれば良い.
    また,ユニタリ行列は一般に,
    \begin{align}
      U = \e^{\i\Lambda/2}
      \mqty(
        \e^{\i\Psi/2} & 0 \\ 
        0 & \e^{-\i\Psi/2}
      )
      \mqty(
        \cos(\Theta/2) & \sin(\Theta/2) \\ 
        -\sin(\Theta/2) & \cos(\Theta/2)
      )\mqty(
        \e^{\i\Phi/2} & 0 \\ 
        0 & \e^{-\i\Phi/2}
      )
    \end{align}
    と分解できる.
    実際,
    \begin{align}
      U &= \e^{\i\Lambda/2}
      \mqty(
        \e^{\i\Psi/2} & 0 \\ 
        0 & \e^{-\i\Psi/2}
      )
      \mqty(
        \cos(\Theta/2) & \sin(\Theta/2) \\ 
        -\sin(\Theta/2) & \cos(\Theta/2)
      )\mqty(
        \e^{\i\Phi/2} & 0 \\ 
        0 & \e^{-\i\Phi/2}
      ) \\ 
      &= \e^{\i\Lambda/2}\mqty(
        \e^{\i\Psi/2}\cos\Theta/2 & \e^{\i\Psi/2}\sin\Theta/2 \\ 
        -\e^{-\i\Psi/2}\sin\Theta/2 & \e^{-\i\Psi/2}\cos\Theta/2
      )\mqty(
        \e^{\i\Phi/2} & 0 \\ 
        0 & \e^{-\i\Phi/2}
      ) \\ 
      &= \e^{\i\Lambda/2}\mqty(
        \e^{\i(\Psi + \Psi) / 2}\cos\Theta/2 & \e^{\i(\Psi - \Phi) / 2}\sin\Theta/2 \\ 
        -\e^{-\i(\Psi - \Phi) / 2}\sin\Theta/2 & \e^{-\i(\Psi + \Phi) / 2}\cos\Theta/2
      )
    \end{align}
    であり,$\psi = \Psi + \Phi$,$\phi = \Psi - \Phi$とすると,
    \begin{align}
      U = \e^{\i\Lambda/2}\mqty(
        \e^{\i\psi / 2}\cos\Theta/2 & \e^{\i\phi / 2}\sin\Theta/2 \\ 
        -\e^{-\i\phi / 2}\sin\Theta/2 & \e^{-\i\psi / 2}\cos\Theta/2
      )
    \end{align}
    と書けて,
    \begin{align}
      U^{\dag}U = \e^{-\i\Lambda/2}\mqty(
        \e^{-\i\psi / 2}\cos\Theta/2 & -\e^{\i\phi / 2}\sin\Theta/2 \\ 
        \e^{-\i\phi / 2}\sin\Theta/2 & \e^{\i\psi / 2}\cos\Theta/2
      )
      \e^{\i\Lambda/2}\mqty(
        \e^{\i\psi / 2}\cos\Theta/2 & \e^{\i\phi / 2}\sin\Theta/2 \\ 
        -\e^{-\i\phi / 2}\sin\Theta/2 & \e^{-\i\psi / 2}\cos\Theta/2
      )
      = \mqty(1 & 0 \\ 0 & 1)
    \end{align}
    となり,ユニタリであることが分かる.
\end{document}