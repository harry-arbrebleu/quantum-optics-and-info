\documentclass{report}
\usepackage{luatexja} % LuaTeXで日本語を使うためのパッケージ
\usepackage{luatexja-fontspec} % LuaTeX用の日本語フォント設定

% --- 数学関連 ---
\usepackage{amsmath, amssymb, amsfonts, mathtools, bm, amsthm} % 基本的な数学パッケージ
\usepackage{type1cm, upgreek} % 数式フォントとギリシャ文字k
\usepackage{physics, mhchem} % 物理や化学の記号や式の表記を簡単にする

% --- 表関連 ---
\usepackage{multirow, longtable, tabularx, array, colortbl, dcolumn, diagbox} % 表のレイアウトを柔軟にする
\usepackage{tablefootnote, truthtable} % 表中に注釈を追加、真理値表
\usepackage{tabularray} % 高度な表組みレイアウト

% --- グラフィック関連 ---
\usepackage{tikz, graphicx} % 図の描画と画像の挿入
% \usepackage{background} % ウォーターマークの設定
\usepackage{caption, subcaption} % 図や表のキャプション設定
\usepackage{float, here} % 図や表の位置指定

% --- レイアウトとページ設定 ---
\usepackage{fancyhdr} % ページヘッダー、フッター、余白の設定
\usepackage[top = 20truemm, bottom = 20truemm, left = 20truemm, right = 20truemm]{geometry}
\usepackage{fancybox, ascmac} % ボックスのデザイン

% --- 色とスタイル ---
\usepackage{xcolor, color, colortbl, tcolorbox} % 色とカラーボックス
\usepackage{listings, jvlisting} % コードの色付けとフォーマット

% --- 参考文献関連 ---
\usepackage{biblatex, usebib} % 参考文献の管理と挿入
\usepackage{url, hyperref} % URLとリンクの設定

% --- その他の便利なパッケージ ---
\usepackage{footmisc} % 脚注のカスタマイズ
\usepackage{multicol} % 複数段組
\usepackage{comment} % コメントアウトの拡張
\usepackage{siunitx} % 単位の表記
\usepackage{docmute}
% \usepackage{appendix}
% --- tcolorboxとtikzの設定 ---
\tcbuselibrary{theorems, breakable} % 定理のボックスと改ページ設定
\usetikzlibrary{decorations.markings, arrows.meta, calc} % tikzの装飾や矢印の設定

% --- 定理スタイルと数式設定 ---
\theoremstyle{definition} % 定義スタイル
\numberwithin{equation}{section} % 式番号をサブセクション単位でリセット

% --- hyperrefの設定 ---
\hypersetup{
  setpagesize = false,
  bookmarks = true,
  bookmarksdepth = tocdepth,
  bookmarksnumbered = true,
  colorlinks = false,
  pdftitle = {}, % PDFタイトル
  pdfsubject = {}, % PDFサブジェクト
  pdfauthor = {}, % PDF作者
  pdfkeywords = {} % PDFキーワード
}

% --- siunitxの設定 ---
\sisetup{
  table-format = 1.5, % 小数点以下の桁数
  table-number-alignment = center, % 数値の中央揃え
}


% --- その他の設定 ---
\allowdisplaybreaks % 数式の途中改ページ許可
\newcolumntype{t}{!{\vrule width 0.1pt}} % 新しいカラムタイプ
\newcolumntype{b}{!{\vrule width 1.5pt}} % 太いカラム
\UseTblrLibrary{amsmath, booktabs, counter, diagbox, functional, hook, html, nameref, siunitx, varwidth, zref} % tabularrayのライブラリ
\setlength{\columnseprule}{0.4pt} % カラム区切り線の太さ
\captionsetup[figure]{font = bf} % 図のキャプションの太字設定
\captionsetup[table]{font = bf} % 表のキャプションの太字設定
\captionsetup[lstlisting]{font = bf} % コードのキャプションの太字設定
\captionsetup[subfigure]{font = bf, labelformat = simple} % サブ図のキャプション設定
\setcounter{secnumdepth}{4} % セクションの深さ設定
\newcolumntype{d}{D{.}{.}{5}} % 数値のカラム
\newcolumntype{M}[1]{>{\centering\arraybackslash}m{#1}} % センター揃えのカラム
\DeclareMathOperator{\diag}{diag}
\everymath{\displaystyle} % 数式のスタイル
\newcommand{\inner}[2]{\left\langle #1, #2 \right\rangle}
\renewcommand{\figurename}{図}
\renewcommand{\i}{\mathrm{i}} % 複素数単位i
\renewcommand{\laplacian}{\grad^2} % ラプラシアンの記号
\renewcommand{\thesubfigure}{(\alph{subfigure})} % サブ図の番号形式
\newcommand{\m}[3]{\multicolumn{#1}{#2}{#3}} % マルチカラムのショートカット
\renewcommand{\r}[1]{\mathrm{#1}} % mathrmのショートカット
\newcommand{\e}{\mathrm{e}} % 自然対数の底e
\newcommand{\Ef}{E_{\mathrm{F}}} % フェルミエネルギー
\renewcommand{\c}{\si{\degreeCelsius}} % 摂氏記号
\renewcommand{\d}{\r{d}} % d記号
\renewcommand{\t}[1]{\texttt{#1}} % タイプライタフォント
\newcommand{\kb}{k_{\mathrm{B}}} % ボルツマン定数
% \renewcommand{\phi}{\varphi} % ϕをφに変更
\renewcommand{\epsilon}{\varepsilon}
\newcommand{\fullref}[1]{\textbf{\ref{#1} \nameref{#1}}}
\newcommand{\reff}[1]{\textbf{図\ref{#1}}} % 図参照のショートカット
\newcommand{\reft}[1]{\textbf{表\ref{#1}}} % 表参照のショートカット
\newcommand{\refe}[1]{\textbf{式\eqref{#1}}} % 式参照のショートカット
\newcommand{\refp}[1]{\textbf{コード\ref{#1}}} % コード参照のショートカット
\renewcommand{\lstlistingname}{コード} % コードリストの名前
\renewcommand{\theequation}{\thesection.\arabic{equation}} % 式番号の形式
\renewcommand{\footrulewidth}{0.4pt} % フッターの線
\newcommand{\mar}[1]{\textcircled{\scriptsize #1}} % 丸囲み文字
\newcommand{\combination}[2]{{}_{#1} \mathrm{C}_{#2}} % 組み合わせ
\newcommand{\thline}{\noalign{\hrule height 0.1pt}} % 細い横線
\newcommand{\bhline}{\noalign{\hrule height 1.5pt}} % 太い横線

% --- カスタム色定義 ---
\definecolor{burgundy}{rgb}{0.5, 0.0, 0.13} % バーガンディ色
\definecolor{charcoal}{rgb}{0.21, 0.27, 0.31} % チャコール色
\definecolor{forest}{rgb}{0.0, 0.35, 0} % 森の緑色

% --- カスタム定理環境の定義 ---
\newtcbtheorem[number within = chapter]{myexc}{練習問題}{
  fonttitle = \gtfamily\sffamily\bfseries\upshape,
  colframe = forest,
  colback = forest!2!white,
  rightrule = 1pt,
  leftrule = 1pt,
  bottomrule = 2pt,
  colbacktitle = forest,
  theorem style = standard,
  breakable,
  arc = 0pt,
}{exc-ref}
\newtcbtheorem[number within = chapter]{myprop}{命題}{
  fonttitle = \gtfamily\sffamily\bfseries\upshape,
  colframe = blue!50!black,
  colback = blue!50!black!2!white,
  rightrule = 1pt,
  leftrule = 1pt,
  bottomrule = 2pt,
  colbacktitle = blue!50!black,
  theorem style = standard,
  breakable,
  arc = 0pt
}{proposition-ref}
\newtcbtheorem[number within = chapter]{myrem}{注意}{
  fonttitle = \gtfamily\sffamily\bfseries\upshape,
  colframe = yellow!20!black,
  colback = yellow!50,
  rightrule = 1pt,
  leftrule = 1pt,
  bottomrule = 2pt,
  colbacktitle = yellow!20!black,
  theorem style = standard,
  breakable,
  arc = 0pt
}{remark-ref}
\newtcbtheorem[number within = chapter]{myex}{例題}{
  fonttitle = \gtfamily\sffamily\bfseries\upshape,
  colframe = black,
  colback = white,
  rightrule = 1pt,
  leftrule = 1pt,
  bottomrule = 2pt,
  colbacktitle = black,
  theorem style = standard,
  breakable,
  arc = 0pt
}{example-ref}
\newtcbtheorem[number within = chapter]{exc}{Requirement}{myexc}{exc-ref}
\newcommand{\rqref}[1]{{\bfseries\sffamily 練習問題 \ref{exc-ref:#1}}}
\newtcbtheorem[number within = chapter]{definition}{Definition}{mydef}{definition-ref}
\newcommand{\dfref}[1]{{\bfseries\sffamily 定義 \ref{definition-ref:#1}}}
\newtcbtheorem[number within = chapter]{prop}{命題}{myprop}{proposition-ref}
\newcommand{\prref}[1]{{\bfseries\sffamily 命題 \ref{proposition-ref:#1}}}
\newtcbtheorem[number within = chapter]{rem}{注意}{myrem}{remark-ref}
\newcommand{\rmref}[1]{{\bfseries\sffamily 注意 \ref{remark-ref:#1}}}
\newtcbtheorem[number within = chapter]{ex}{例題}{myex}{example-ref}
\newcommand{\exref}[1]{{\bfseries\sffamily 例題 \ref{example-ref:#1}}}
% --- 再定義コマンド ---
% \mathtoolsset{showonlyrefs=true} % 必要な式番号のみ表示
\pagestyle{fancy} % ヘッダー・フッターのスタイル設定
% \chead{応用量子物性講義ノート} % 中央ヘッダー
% \rhead{}
\fancyhead[R]{\rightmark}
\renewcommand{\subsectionmark}[1]{\markright{\thesubsection\ #1}}
\cfoot{\thepage} % 中央フッターにページ番号
\lhead{}
\rfoot{Haruki Aoki and Hiroki Fukuhara} % 右フッターに名前
\setcounter{tocdepth}{4} % 目次の深さ
\makeatletter
\@addtoreset{equation}{section} % セクwションごとに式番号をリセット
\makeatother

% --- メタ情報 ---
\title{量子光学・量子情報科学ノート}
\date{更新日: \today}
\author{Haruki Aoki and Hiroki Fukuhara}

\begin{document}
  \maketitle
  \tableofcontents
  \chapter{量子光学}
    量子力学は物理量$\hat{A}$の期待値$\ev{A} = \mel**{\psi}{\hat{A}}{\psi}$を検討する学問である.
    本章は以下のような構成である.
    まず,Sch\"odinger描像とHeisenberg描像や電場の量子化について説明する.
    また,量子もつれにおいて重要なビームスプリッタを紹介する.
    次に,量子光学において重要なコヒーレント状態とスクイーズド状態について議論する.
    続いて密度演算子を定義したあと,今まで議論した状態を具体的に測定するための方法として,バランス型ホモダイン測定を紹介する.
    \section{Sch\"odinger描像とHeisenberg描像}
      WIP
    \section{調和振動子}
      \documentclass{report}
\usepackage{luatexja} % LuaTeXで日本語を使うためのパッケージ
\usepackage{luatexja-fontspec} % LuaTeX用の日本語フォント設定

% --- 数学関連 ---
\usepackage{amsmath, amssymb, amsfonts, mathtools, bm, amsthm} % 基本的な数学パッケージ
\usepackage{type1cm, upgreek} % 数式フォントとギリシャ文字k
\usepackage{physics, mhchem} % 物理や化学の記号や式の表記を簡単にする

% --- 表関連 ---
\usepackage{multirow, longtable, tabularx, array, colortbl, dcolumn, diagbox} % 表のレイアウトを柔軟にする
\usepackage{tablefootnote, truthtable} % 表中に注釈を追加、真理値表
\usepackage{tabularray} % 高度な表組みレイアウト

% --- グラフィック関連 ---
\usepackage{tikz, graphicx} % 図の描画と画像の挿入
% \usepackage{background} % ウォーターマークの設定
\usepackage{caption, subcaption} % 図や表のキャプション設定
\usepackage{float, here} % 図や表の位置指定

% --- レイアウトとページ設定 ---
\usepackage{fancyhdr} % ページヘッダー、フッター、余白の設定
\usepackage[top = 20truemm, bottom = 20truemm, left = 20truemm, right = 20truemm]{geometry}
\usepackage{fancybox, ascmac} % ボックスのデザイン

% --- 色とスタイル ---
\usepackage{xcolor, color, colortbl, tcolorbox} % 色とカラーボックス
\usepackage{listings, jvlisting} % コードの色付けとフォーマット

% --- 参考文献関連 ---
\usepackage{biblatex, usebib} % 参考文献の管理と挿入
\usepackage{url, hyperref} % URLとリンクの設定

% --- その他の便利なパッケージ ---
\usepackage{footmisc} % 脚注のカスタマイズ
\usepackage{multicol} % 複数段組
\usepackage{comment} % コメントアウトの拡張
\usepackage{siunitx} % 単位の表記
\usepackage{docmute}
% \usepackage{appendix}
% --- tcolorboxとtikzの設定 ---
\tcbuselibrary{theorems, breakable} % 定理のボックスと改ページ設定
\usetikzlibrary{decorations.markings, arrows.meta, calc} % tikzの装飾や矢印の設定

% --- 定理スタイルと数式設定 ---
\theoremstyle{definition} % 定義スタイル
\numberwithin{equation}{section} % 式番号をサブセクション単位でリセット

% --- hyperrefの設定 ---
\hypersetup{
  setpagesize = false,
  bookmarks = true,
  bookmarksdepth = tocdepth,
  bookmarksnumbered = true,
  colorlinks = false,
  pdftitle = {}, % PDFタイトル
  pdfsubject = {}, % PDFサブジェクト
  pdfauthor = {}, % PDF作者
  pdfkeywords = {} % PDFキーワード
}

% --- siunitxの設定 ---
\sisetup{
  table-format = 1.5, % 小数点以下の桁数
  table-number-alignment = center, % 数値の中央揃え
}


% --- その他の設定 ---
\allowdisplaybreaks % 数式の途中改ページ許可
\newcolumntype{t}{!{\vrule width 0.1pt}} % 新しいカラムタイプ
\newcolumntype{b}{!{\vrule width 1.5pt}} % 太いカラム
\UseTblrLibrary{amsmath, booktabs, counter, diagbox, functional, hook, html, nameref, siunitx, varwidth, zref} % tabularrayのライブラリ
\setlength{\columnseprule}{0.4pt} % カラム区切り線の太さ
\captionsetup[figure]{font = bf} % 図のキャプションの太字設定
\captionsetup[table]{font = bf} % 表のキャプションの太字設定
\captionsetup[lstlisting]{font = bf} % コードのキャプションの太字設定
\captionsetup[subfigure]{font = bf, labelformat = simple} % サブ図のキャプション設定
\setcounter{secnumdepth}{4} % セクションの深さ設定
\newcolumntype{d}{D{.}{.}{5}} % 数値のカラム
\newcolumntype{M}[1]{>{\centering\arraybackslash}m{#1}} % センター揃えのカラム
\DeclareMathOperator{\diag}{diag}
\everymath{\displaystyle} % 数式のスタイル
\newcommand{\inner}[2]{\left\langle #1, #2 \right\rangle}
\renewcommand{\figurename}{図}
\renewcommand{\i}{\mathrm{i}} % 複素数単位i
\renewcommand{\laplacian}{\grad^2} % ラプラシアンの記号
\renewcommand{\thesubfigure}{(\alph{subfigure})} % サブ図の番号形式
\newcommand{\m}[3]{\multicolumn{#1}{#2}{#3}} % マルチカラムのショートカット
\renewcommand{\r}[1]{\mathrm{#1}} % mathrmのショートカット
\newcommand{\e}{\mathrm{e}} % 自然対数の底e
\newcommand{\Ef}{E_{\mathrm{F}}} % フェルミエネルギー
\renewcommand{\c}{\si{\degreeCelsius}} % 摂氏記号
\renewcommand{\d}{\r{d}} % d記号
\renewcommand{\t}[1]{\texttt{#1}} % タイプライタフォント
\newcommand{\kb}{k_{\mathrm{B}}} % ボルツマン定数
% \renewcommand{\phi}{\varphi} % ϕをφに変更
\renewcommand{\epsilon}{\varepsilon}
\newcommand{\fullref}[1]{\textbf{\ref{#1} \nameref{#1}}}
\newcommand{\reff}[1]{\textbf{図\ref{#1}}} % 図参照のショートカット
\newcommand{\reft}[1]{\textbf{表\ref{#1}}} % 表参照のショートカット
\newcommand{\refe}[1]{\textbf{式\eqref{#1}}} % 式参照のショートカット
\newcommand{\refp}[1]{\textbf{コード\ref{#1}}} % コード参照のショートカット
\renewcommand{\lstlistingname}{コード} % コードリストの名前
\renewcommand{\theequation}{\thesection.\arabic{equation}} % 式番号の形式
\renewcommand{\footrulewidth}{0.4pt} % フッターの線
\newcommand{\mar}[1]{\textcircled{\scriptsize #1}} % 丸囲み文字
\newcommand{\combination}[2]{{}_{#1} \mathrm{C}_{#2}} % 組み合わせ
\newcommand{\thline}{\noalign{\hrule height 0.1pt}} % 細い横線
\newcommand{\bhline}{\noalign{\hrule height 1.5pt}} % 太い横線

% --- カスタム色定義 ---
\definecolor{burgundy}{rgb}{0.5, 0.0, 0.13} % バーガンディ色
\definecolor{charcoal}{rgb}{0.21, 0.27, 0.31} % チャコール色
\definecolor{forest}{rgb}{0.0, 0.35, 0} % 森の緑色

% --- カスタム定理環境の定義 ---
\newtcbtheorem[number within = chapter]{myexc}{練習問題}{
  fonttitle = \gtfamily\sffamily\bfseries\upshape,
  colframe = forest,
  colback = forest!2!white,
  rightrule = 1pt,
  leftrule = 1pt,
  bottomrule = 2pt,
  colbacktitle = forest,
  theorem style = standard,
  breakable,
  arc = 0pt,
}{exc-ref}
\newtcbtheorem[number within = chapter]{myprop}{命題}{
  fonttitle = \gtfamily\sffamily\bfseries\upshape,
  colframe = blue!50!black,
  colback = blue!50!black!2!white,
  rightrule = 1pt,
  leftrule = 1pt,
  bottomrule = 2pt,
  colbacktitle = blue!50!black,
  theorem style = standard,
  breakable,
  arc = 0pt
}{proposition-ref}
\newtcbtheorem[number within = chapter]{myrem}{注意}{
  fonttitle = \gtfamily\sffamily\bfseries\upshape,
  colframe = yellow!20!black,
  colback = yellow!50,
  rightrule = 1pt,
  leftrule = 1pt,
  bottomrule = 2pt,
  colbacktitle = yellow!20!black,
  theorem style = standard,
  breakable,
  arc = 0pt
}{remark-ref}
\newtcbtheorem[number within = chapter]{myex}{例題}{
  fonttitle = \gtfamily\sffamily\bfseries\upshape,
  colframe = black,
  colback = white,
  rightrule = 1pt,
  leftrule = 1pt,
  bottomrule = 2pt,
  colbacktitle = black,
  theorem style = standard,
  breakable,
  arc = 0pt
}{example-ref}
\newtcbtheorem[number within = chapter]{exc}{Requirement}{myexc}{exc-ref}
\newcommand{\rqref}[1]{{\bfseries\sffamily 練習問題 \ref{exc-ref:#1}}}
\newtcbtheorem[number within = chapter]{definition}{Definition}{mydef}{definition-ref}
\newcommand{\dfref}[1]{{\bfseries\sffamily 定義 \ref{definition-ref:#1}}}
\newtcbtheorem[number within = chapter]{prop}{命題}{myprop}{proposition-ref}
\newcommand{\prref}[1]{{\bfseries\sffamily 命題 \ref{proposition-ref:#1}}}
\newtcbtheorem[number within = chapter]{rem}{注意}{myrem}{remark-ref}
\newcommand{\rmref}[1]{{\bfseries\sffamily 注意 \ref{remark-ref:#1}}}
\newtcbtheorem[number within = chapter]{ex}{例題}{myex}{example-ref}
\newcommand{\exref}[1]{{\bfseries\sffamily 例題 \ref{example-ref:#1}}}
% --- 再定義コマンド ---
% \mathtoolsset{showonlyrefs=true} % 必要な式番号のみ表示
\pagestyle{fancy} % ヘッダー・フッターのスタイル設定
% \chead{応用量子物性講義ノート} % 中央ヘッダー
% \rhead{}
\fancyhead[R]{\rightmark}
\renewcommand{\subsectionmark}[1]{\markright{\thesubsection\ #1}}
\cfoot{\thepage} % 中央フッターにページ番号
\lhead{}
\rfoot{Haruki Aoki and Hiroki Fukuhara} % 右フッターに名前
\setcounter{tocdepth}{4} % 目次の深さ
\makeatletter
\@addtoreset{equation}{section} % セクwションごとに式番号をリセット
\makeatother

% --- メタ情報 ---
\title{量子光学・量子情報科学ノート}
\date{更新日: \today}
\author{Haruki Aoki and Hiroki Fukuhara}

\begin{document}
  本節では,1次元調和振動子モデルでハミルトニアンが書けるときの波動函数の表示を求める.
  波動函数とは,Schr\"odinger方程式,
  \begin{align}
    \hat{H}\ket{\psi} = E\ket{\psi}\label{schrodinger-equation}
  \end{align}
  を満たす$\ket{\psi}$について,
  \begin{align}
    \psi(x) \coloneqq \braket{x}{\psi}
  \end{align}
  となるように(一般化)座標$x$へ射影したものである.
  \refe{schrodinger-equation}に対して$\bra{x}$を左から書ければ,
  \begin{align}
    \mel**{x}{\hat{H}}{\psi} = E\psi(x)\label{bra-x-from-left}
  \end{align}
  となるのだから,左辺を計算して$\psi(x)$に演算子がかかる形に変形すれば,波動函数を求めることができる.
  本ノートにおいて,$\hat{\cdot}$を演算子として,その固有値を$\cdot$,固有ベクトル(固有函数)を$\ket{\cdot}$と書く.
  \begin{align}
    \hat{x}\ket{x} &= x\ket{x} \\ 
    \hat{p}\ket{x} &= p\ket{x}
  \end{align}
  である.また,$\hat{x}$や$\hat{p}$は物理量であり,Hermite演算子だからその固有ベクトルは,
  \begin{align}
    \braket{x'}{x} &= \delta\qty(x' - x) \\ 
    \braket{p'}{p} &= \delta\qty(p' - p)
  \end{align}
  と規格化してあり,
  \begin{align}
    \int\dd{x}\ketbra{x}{x} &= \hat{1} \\ 
    \int\dd{p}\ketbra{p}{p} &= \hat{1}
  \end{align}
  が成立する.なお,特に断らない限り積分範囲は$-\infty$から$\infty$である.
  \subsection{ハミルトニアン}
    古典的な1次元調和振動子のハミルトニアン$H$は,
    \begin{align}
      H = \frac{1}{2}m\omega^2x^2 + \frac{1}{2m}p^2
    \end{align}
    である.ただし,質量を$m$,固有角周波数を$\omega$,座標を$x$,運動量を$p$とした.
    $x$と$p$は正準共役な変数の組であるから,$x\to \hat{x}$,$p \to \hat{p}$として,
    \begin{align}
      \hat{H} &= \frac{1}{2}m\omega^2\hat{x}^2 + \frac{1}{2m}\hat{p}^2 \label{harmonic-oscillator-hamiltonian}\\ 
      &= \hbar\omega\qty(\frac{m\omega}{2\hbar}\hat{x}^2 + \frac{1}{2m\hbar\omega}\hat{p}^2) \\ 
      &= \hbar\omega\qty[\qty(\sqrt{\frac{m\omega}{2\hbar}}\hat{x} - \i\sqrt{\frac{1}{2m\hbar\omega}}\hat{p})\qty(\sqrt{\frac{m\omega}{2\hbar}}\hat{x} + \i\sqrt{\frac{1}{2m\hbar\omega}}\hat{p}) - \i\sqrt{\frac{m\omega}{2\hbar}}\sqrt{\frac{1}{2m\hbar\omega}}\qty[\hat{x}, \hat{p}]] \\ 
      &= \hbar\omega\qty(\hat{a}^{\dag}\hat{a} + \frac{1}{2})
    \end{align}
    となる.ただし,
    \begin{align}
      \hat{a}^{\dag} &\coloneqq \qty(\sqrt{\frac{m\omega}{2\hbar}}\hat{x} - \i\sqrt{\frac{1}{2m\hbar\omega}}\hat{p}) \\ 
      \hat{a} &\coloneqq \qty(\sqrt{\frac{m\omega}{2\hbar}}\hat{x} + \i\sqrt{\frac{1}{2m\hbar\omega}}\hat{p})
    \end{align}
    と定義した.
  \subsection{Hermite多項式}
    以降の議論で用いるために,特殊函数の1つであるHermite多項式を紹介しておこう.
    Hermite多項式はStrum-Liouville演算子のうちの1つの演算子の固有函数であり,実数全体で定義された実数函数$H_n(s)$に対して,
    \begin{align}
      \qty(\dv[2]{s} - 2s\dv{s} + 2n)H_n(s) = 0
    \end{align}
    なる$H_n(s)$である.
    なお,$H_n(s)$は適当な回数だけ微分可能であるとする.
    また,$H_n(s)$が張る空間$V$の内積は,$f, g\in V$として,
    \begin{align}
      \inner{f}{g} \coloneqq \int_{-\infty}^{\infty}\dd{s}f(s)g(s)\e^{-s^2}
    \end{align}
    である\footnote{
      Strum-Liouville演算子の形,
      \begin{align*}
        \frac{1}{\rho(x)}\qty[\dv{x}\qty{p(x)\dv{x}} + q(x)]
      \end{align*}
      と,Strum-Liouville演算子の固有函数が張る空間の内積が,
      \begin{align*}
        \int_{a}^{b} f^*(x)g(x)\rho(x)\dd{x}
      \end{align*}
      と書けることを思い出せば,内積に$\e^{-s^2}$なる重み函数が入ることは自然なことである.
    }.
    Hermite多項式は,適切に境界条件が設定された(Hermite性のある)Strum-Liouville演算子の固有函数であり,
    そのような演算子の固有函数は直交基底となり,完全系を成すことが知られていて,実際,
    \begin{align}
      \int_{-\infty}^{\infty}H_m(s)H_n(s)\e^{-s^2}\dd{s} = \sqrt{\pi}2^nn!\delta_n^m
    \end{align}
    のように直交する.
  \subsection{波動函数を用いたSchr\"odinger方程式}
    以下では,波動函数を用いたchr\"odinger方程式である,
    \begin{align}
      \qty(-\frac{\hbar^2}{2m}\dv[2]{x} + \frac{1}{2}m\omega^2x^2)\psi(x) = E\psi(x)
    \end{align}
    を得る.
    \par
    \refe{bra-x-from-left}に\refe{harmonic-oscillator-hamiltonian}で示した$\hat{H}$の表式を代入して,
    \begin{align}
      \frac{1}{2}m\omega^2\mel**{x}{\hat{x}^2}{\psi} + \frac{1}{2m}\mel**{x}{\hat{p}^2}{\psi} &= E\psi(x) \\ 
      \frac{1}{2}m\omega^2x^2\psi(x) + \frac{1}{2m}\mel**{x}{\hat{p}^2}{\psi} &= E\psi(x) \label{input-to-schrodinger}
    \end{align}
    となる.
    $\mel**{x}{\hat{p}^2}{\psi}$は以下のレシピで計算できる.
    \begin{enumerate}
      \item $f(x)\dv{x}\delta(x) = -\dv{x}f(x)\delta(x)$
      \item $\mel**{x}{\hat{p}}{\psi} = -\i\hbar\dv{x}\psi(x)$
      \item $\braket{x}{p} = \frac{1}{\sqrt{2\pi p}}\exp\qty(\i\frac{xp}{\hbar})$
      \item $\mel**{x}{\hat{p}^2}{\psi}$の計算
    \end{enumerate}
    $\delta(x)$はデルタ函数であり,積分して初めて意味を持つ函数である.
    \begin{enumerate}
      \item $f(x)\dv{x}\delta(x) = -\dv{x}f(x)\delta(x)$\par
        左辺を積分して右辺になればよい.ただし,$f(x)$は,
        \begin{align}
          \lim_{\abs{x} \to \infty}f(x) = 0
        \end{align}
        であるとする.
        実際に,
        \begin{align}
          \int_{-\infty}^{\infty}\dd{x}f(x)\dv{x}\delta(x) &= \qty[f(x)\delta(x)]_{-\infty}{\infty} - \int_{-\infty}^{\infty}\dd{x}\dv{x}f(x)\delta(x) \\ 
          &= - \int_{-\infty}^{\infty}\dd{x}\dv{x}f(x)\delta(x)
        \end{align}
        であるから,
        \begin{align}
          f(x)\dv{x}\delta(x) = -\dv{x}f(x)\delta(x)\label{delta-function-diff}
        \end{align}
        である.  
      \item $\mel**{x}{\hat{p}}{\psi} = -\i\hbar\dv{x}\psi(x)$\par
        $\mel**{x}{\qty[\hat{x}, \hat{p}]}{x'}$を2種類の方法で計算する.
        まず,愚直に計算すると,
        \begin{align}
          \mel**{x}{\qty[\hat{x}, \hat{p}]}{x'} &= \mel**{x}{\hat{x}\hat{p} - \hat{p}\hat{x}}{x'} \\ 
          &= x\mel**{x}{\hat{p}}{x'} - x'\mel**{x}{\hat{p}}{x'} \\ 
          &= \qty(x - x')\mel**{x}{\hat{p}}{x'}\label{xxpxprime-1}
        \end{align}
        である.一方,$\qty[\hat{x}, \hat{p}] = \i\hbar$を用いれば,
        \begin{align}
          \mel**{x}{\qty[\hat{x}, \hat{p}]}{x'} &= \i\hbar\braket{x}{x'} \\ 
          &= \i\hbar\delta(x - x')\label{xxpxprime-2}
        \end{align}
        2つの方法で計算した$\mel**{x}{\qty[\hat{x}, \hat{p}]}{x'}$である\refe{xxpxprime-1}と\refe{xxpxprime-2}を等号で結んで,
        \refe{delta-function-diff}で示したデルタ函数の微分を用いて表現すれば,
        \begin{align}
          \mel**{x}{\hat{p}}{x'} &= \i\hbar\frac{\delta(x - x')}{x - x'} \\ 
          &= -\i\hbar\dv{\qty(x - x')}\delta(x - x') \\ 
          &= \i\hbar\dv{x'}\delta(x - x')\label{xpxrime}
        \end{align}
        となる.
        \par
        さて,$\mel**{x}{\hat{p}}{\psi}$を計算しよう.
        $\ket{x}$の完全性と,\refe{xpxrime}で示した関係を用いれば,
        \begin{align}
          \mel**{x}{\hat{p}}{\psi} &= \mel**{x}{\hat{p}\hat{1}}{\psi} \\ 
          &= \bra{x}\hat{p}\int\dd{x'}\ketbra{x'}{x'}\ket{\psi} \\ 
          &= \int\dd{x'}\mel**{x}{\hat{p}}{x'}\braket{x'}{\psi} \\ 
          &= \i\hbar\int\dd{x'}\qty[\dv{x}\delta(x - x')]\phi(x') \\ 
          &= \i\hbar\qty{\qty[\delta\qty(x - x')\psi\qty(x')]_{-\infty}^{\infty} - \int\dd{x'}\dv{x'}\phi(x')\delta(x - x')} \\ 
          &= -\i\hbar\dv{x}\phi(x)\label{psix-diff}
        \end{align}
        を得る.
      \item $\braket{x}{p} = \frac{1}{\sqrt{2\pi p}}\exp\qty(\i\frac{xp}{\hbar})$ \par
        $\mel**{x}{\hat{p}}{p}$を2種類の方法で計算する.
        まず,愚直に計算すると,
        \begin{align}
          \mel**{x}{\hat{p}}{p} &= p\braket{x}{p} \\ 
          &= pp(x)\label{xpp-1}
        \end{align}
        となる.ただし$p(x)$は$\ket{p}$の$x$への射影である.
        一方,\refe{psix-diff}で示した関係で$\ket{\psi} \to \ket{p}$を用いると,
        \begin{align}
          \mel**{x}{\hat{p}}{\psi} = -\i\hbar\dv{x}p(x)\label{xpp-2}
        \end{align} 
        となる.\refe{xpp-1}と\refe{xpp-2}より,
        \begin{align}
          -\i\hbar\dv{x}p(x) &= pp(x) \\ 
          \Rightarrow p(x) &= C\exp\qty(\i\frac{xp}{\hbar})
        \end{align}
        となる.$C$は規格化定数である.
        \par
        さて,$C$を求めるために,$\braket{x}{x'}$を計算すると,
        \begin{align}
          \delta(x - x') &= \braket{x}{x'} \\ 
          &= \bra{x}\int\dd{p}\ketbra{p}{p}\ket{x'} \\ 
          &= \int\dd{p}p(x)p(x') \\ 
          &= \abs{C}^2\int\exp\qty(\i\frac{\i\qty(x - x')p}{\hbar})
        \end{align}
        となる.ところで,デルタ函数のFouirer変換とその逆変換が,
        \begin{align}
          1 &= \int_{-\infty}^{\infty}\dd{t}\delta(t)\e^{-\i\omega t} \\ 
          \delta(t) &= \frac{1}{2\pi}\int_{-\infty}^{\infty}\dd{\omega}1\cdot\e^{\i\omega t}\label{delta-fourier-inv}
        \end{align}
        と書けることより,\refe{delta-fourier-inv}において,
        \begin{align}
          \omega &\to \frac{p}{\hbar} \\ 
          t &\to x - x'
        \end{align}
        と変換すれば,
        \begin{align}
          \delta(x - x') = \frac{1}{2\pi\hbar}\int\dd{p}\exp\qty(\i\frac{x - x'}{\hbar}p)
        \end{align}
        となるので,係数を比較して,
        \begin{align}
          \abs{C}^2 &= \frac{1}{2\pi\hbar} \\ 
          \Rightarrow C &= \frac{1}{\sqrt{2\pi\hbar}} 
        \end{align}
        となる.よって,
        \begin{align}
          \braket{x}{p} = \frac{1}{\sqrt{2\pi\hbar}}\exp\qty(\i\frac{xp}{\hbar})\label{braket-xp}
        \end{align}
        となる.
      \item $\mel**{x}{\hat{p}^2}{\psi}$の計算\par
        さて,いよいよ$\mel**{x}{\hat{p}^2}{\psi}$を計算する道具がそろった.
        \refe{braket-xp}を用いながら計算すると,
        \begin{align}
          \mel**{x}{\hat{p}^2}{\psi} &= \mel**{x}{\hat{p}^2\hat{1}}{\psi} \\
          &= \bra{x}\hat{p}^2\int\dd{p}\ketbra{p}{p}\ket{\psi} \\ 
          &= \int\dd{p}p^2\braket{x}{p}\braket{p}{\psi} \\ 
          &= \int\dd{p}p^2\frac{1}{\sqrt{2\pi\hbar}}\exp\qty(\i\frac{xp}{\hbar})\braket{p}{\psi} \\ 
          &= \int\dd{p}\frac{1}{\sqrt{2\pi\hbar}}\qty{\dv[2]{x}\qty(\frac{\hbar}{\i})^2\braket{x}{p}}\braket{p}{\psi} \\ 
          &= \qty(\frac{\hbar}{\i})^2\int\dd{p}\qty{\dv[2]{x}\braket{x}{p}}\braket{p}{\psi} \\ 
          &= \qty(\frac{\hbar}{\i})^2\dv[2]{x}\bra{x}\qty[\int\dd{p}\ketbra{p}{p}]\ket{\psi} \\ 
          &= \qty(\frac{\hbar}{\i})^2\dv[2]{x}\braket{x}{\psi} \\
          &= \qty(\frac{\hbar}{\i})^2\dv[2]{x}\psi(x)\label{before-schrodinger-oscillator}
        \end{align}
        となる.
    \end{enumerate}
    \refe{before-schrodinger-oscillator}を\refe{input-to-schrodinger}に代入すれば,
    \begin{align}
      \frac{1}{2}m\omega^2x^2\psi(x) + \frac{1}{2m}\mel**{x}{\hat{p}^2}{\psi} &= E\psi(x) \\ 
      \Leftrightarrow \frac{1}{2}m\omega^2x^2\psi(x) + \frac{1}{2m}\qty(\frac{\hbar}{\i})^2\dv[2]{x}\psi(x) &= E\psi(x) \\ 
      \Leftrightarrow \qty(-\frac{\hbar}{2m}\dv[2]{x} + \frac{1}{2}m\omega^2x^2)\psi(x) &= E\psi(x)\label{schrodinger-eq-wave-func}
    \end{align}
    となる.
  \subsection{Schrodingerの解法}
    いささか唐突だが,
\end{document}
    \section{電磁場の量子化}
      WIP
      \documentclass{report}
\usepackage{luatexja} % LuaTeXで日本語を使うためのパッケージ
\usepackage{luatexja-fontspec} % LuaTeX用の日本語フォント設定

% --- 数学関連 ---
\usepackage{amsmath, amssymb, amsfonts, mathtools, bm, amsthm} % 基本的な数学パッケージ
\usepackage{type1cm, upgreek} % 数式フォントとギリシャ文字k
\usepackage{physics, mhchem} % 物理や化学の記号や式の表記を簡単にする

% --- 表関連 ---
\usepackage{multirow, longtable, tabularx, array, colortbl, dcolumn, diagbox} % 表のレイアウトを柔軟にする
\usepackage{tablefootnote, truthtable} % 表中に注釈を追加、真理値表
\usepackage{tabularray} % 高度な表組みレイアウト

% --- グラフィック関連 ---
\usepackage{tikz, graphicx} % 図の描画と画像の挿入
% \usepackage{background} % ウォーターマークの設定
\usepackage{caption, subcaption} % 図や表のキャプション設定
\usepackage{float, here} % 図や表の位置指定

% --- レイアウトとページ設定 ---
\usepackage{fancyhdr} % ページヘッダー、フッター、余白の設定
\usepackage[top = 20truemm, bottom = 20truemm, left = 20truemm, right = 20truemm]{geometry}
\usepackage{fancybox, ascmac} % ボックスのデザイン

% --- 色とスタイル ---
\usepackage{xcolor, color, colortbl, tcolorbox} % 色とカラーボックス
\usepackage{listings, jvlisting} % コードの色付けとフォーマット

% --- 参考文献関連 ---
\usepackage{biblatex, usebib} % 参考文献の管理と挿入
\usepackage{url, hyperref} % URLとリンクの設定

% --- その他の便利なパッケージ ---
\usepackage{footmisc} % 脚注のカスタマイズ
\usepackage{multicol} % 複数段組
\usepackage{comment} % コメントアウトの拡張
\usepackage{siunitx} % 単位の表記
\usepackage{docmute}
% \usepackage{appendix}
% --- tcolorboxとtikzの設定 ---
\tcbuselibrary{theorems, breakable} % 定理のボックスと改ページ設定
\usetikzlibrary{decorations.markings, arrows.meta, calc} % tikzの装飾や矢印の設定

% --- 定理スタイルと数式設定 ---
\theoremstyle{definition} % 定義スタイル
\numberwithin{equation}{section} % 式番号をサブセクション単位でリセット

% --- hyperrefの設定 ---
\hypersetup{
  setpagesize = false,
  bookmarks = true,
  bookmarksdepth = tocdepth,
  bookmarksnumbered = true,
  colorlinks = false,
  pdftitle = {}, % PDFタイトル
  pdfsubject = {}, % PDFサブジェクト
  pdfauthor = {}, % PDF作者
  pdfkeywords = {} % PDFキーワード
}

% --- siunitxの設定 ---
\sisetup{
  table-format = 1.5, % 小数点以下の桁数
  table-number-alignment = center, % 数値の中央揃え
}


% --- その他の設定 ---
\allowdisplaybreaks % 数式の途中改ページ許可
\newcolumntype{t}{!{\vrule width 0.1pt}} % 新しいカラムタイプ
\newcolumntype{b}{!{\vrule width 1.5pt}} % 太いカラム
\UseTblrLibrary{amsmath, booktabs, counter, diagbox, functional, hook, html, nameref, siunitx, varwidth, zref} % tabularrayのライブラリ
\setlength{\columnseprule}{0.4pt} % カラム区切り線の太さ
\captionsetup[figure]{font = bf} % 図のキャプションの太字設定
\captionsetup[table]{font = bf} % 表のキャプションの太字設定
\captionsetup[lstlisting]{font = bf} % コードのキャプションの太字設定
\captionsetup[subfigure]{font = bf, labelformat = simple} % サブ図のキャプション設定
\setcounter{secnumdepth}{4} % セクションの深さ設定
\newcolumntype{d}{D{.}{.}{5}} % 数値のカラム
\newcolumntype{M}[1]{>{\centering\arraybackslash}m{#1}} % センター揃えのカラム
\DeclareMathOperator{\diag}{diag}
\everymath{\displaystyle} % 数式のスタイル
\newcommand{\inner}[2]{\left\langle #1, #2 \right\rangle}
\renewcommand{\figurename}{図}
\renewcommand{\i}{\mathrm{i}} % 複素数単位i
\renewcommand{\laplacian}{\grad^2} % ラプラシアンの記号
\renewcommand{\thesubfigure}{(\alph{subfigure})} % サブ図の番号形式
\newcommand{\m}[3]{\multicolumn{#1}{#2}{#3}} % マルチカラムのショートカット
\renewcommand{\r}[1]{\mathrm{#1}} % mathrmのショートカット
\newcommand{\e}{\mathrm{e}} % 自然対数の底e
\newcommand{\Ef}{E_{\mathrm{F}}} % フェルミエネルギー
\renewcommand{\c}{\si{\degreeCelsius}} % 摂氏記号
\renewcommand{\d}{\r{d}} % d記号
\renewcommand{\t}[1]{\texttt{#1}} % タイプライタフォント
\newcommand{\kb}{k_{\mathrm{B}}} % ボルツマン定数
% \renewcommand{\phi}{\varphi} % ϕをφに変更
\renewcommand{\epsilon}{\varepsilon}
\newcommand{\fullref}[1]{\textbf{\ref{#1} \nameref{#1}}}
\newcommand{\reff}[1]{\textbf{図\ref{#1}}} % 図参照のショートカット
\newcommand{\reft}[1]{\textbf{表\ref{#1}}} % 表参照のショートカット
\newcommand{\refe}[1]{\textbf{式\eqref{#1}}} % 式参照のショートカット
\newcommand{\refp}[1]{\textbf{コード\ref{#1}}} % コード参照のショートカット
\renewcommand{\lstlistingname}{コード} % コードリストの名前
\renewcommand{\theequation}{\thesection.\arabic{equation}} % 式番号の形式
\renewcommand{\footrulewidth}{0.4pt} % フッターの線
\newcommand{\mar}[1]{\textcircled{\scriptsize #1}} % 丸囲み文字
\newcommand{\combination}[2]{{}_{#1} \mathrm{C}_{#2}} % 組み合わせ
\newcommand{\thline}{\noalign{\hrule height 0.1pt}} % 細い横線
\newcommand{\bhline}{\noalign{\hrule height 1.5pt}} % 太い横線

% --- カスタム色定義 ---
\definecolor{burgundy}{rgb}{0.5, 0.0, 0.13} % バーガンディ色
\definecolor{charcoal}{rgb}{0.21, 0.27, 0.31} % チャコール色
\definecolor{forest}{rgb}{0.0, 0.35, 0} % 森の緑色

% --- カスタム定理環境の定義 ---
\newtcbtheorem[number within = chapter]{myexc}{練習問題}{
  fonttitle = \gtfamily\sffamily\bfseries\upshape,
  colframe = forest,
  colback = forest!2!white,
  rightrule = 1pt,
  leftrule = 1pt,
  bottomrule = 2pt,
  colbacktitle = forest,
  theorem style = standard,
  breakable,
  arc = 0pt,
}{exc-ref}
\newtcbtheorem[number within = chapter]{myprop}{命題}{
  fonttitle = \gtfamily\sffamily\bfseries\upshape,
  colframe = blue!50!black,
  colback = blue!50!black!2!white,
  rightrule = 1pt,
  leftrule = 1pt,
  bottomrule = 2pt,
  colbacktitle = blue!50!black,
  theorem style = standard,
  breakable,
  arc = 0pt
}{proposition-ref}
\newtcbtheorem[number within = chapter]{myrem}{注意}{
  fonttitle = \gtfamily\sffamily\bfseries\upshape,
  colframe = yellow!20!black,
  colback = yellow!50,
  rightrule = 1pt,
  leftrule = 1pt,
  bottomrule = 2pt,
  colbacktitle = yellow!20!black,
  theorem style = standard,
  breakable,
  arc = 0pt
}{remark-ref}
\newtcbtheorem[number within = chapter]{myex}{例題}{
  fonttitle = \gtfamily\sffamily\bfseries\upshape,
  colframe = black,
  colback = white,
  rightrule = 1pt,
  leftrule = 1pt,
  bottomrule = 2pt,
  colbacktitle = black,
  theorem style = standard,
  breakable,
  arc = 0pt
}{example-ref}
\newtcbtheorem[number within = chapter]{exc}{Requirement}{myexc}{exc-ref}
\newcommand{\rqref}[1]{{\bfseries\sffamily 練習問題 \ref{exc-ref:#1}}}
\newtcbtheorem[number within = chapter]{definition}{Definition}{mydef}{definition-ref}
\newcommand{\dfref}[1]{{\bfseries\sffamily 定義 \ref{definition-ref:#1}}}
\newtcbtheorem[number within = chapter]{prop}{命題}{myprop}{proposition-ref}
\newcommand{\prref}[1]{{\bfseries\sffamily 命題 \ref{proposition-ref:#1}}}
\newtcbtheorem[number within = chapter]{rem}{注意}{myrem}{remark-ref}
\newcommand{\rmref}[1]{{\bfseries\sffamily 注意 \ref{remark-ref:#1}}}
\newtcbtheorem[number within = chapter]{ex}{例題}{myex}{example-ref}
\newcommand{\exref}[1]{{\bfseries\sffamily 例題 \ref{example-ref:#1}}}
% --- 再定義コマンド ---
% \mathtoolsset{showonlyrefs=true} % 必要な式番号のみ表示
\pagestyle{fancy} % ヘッダー・フッターのスタイル設定
% \chead{応用量子物性講義ノート} % 中央ヘッダー
% \rhead{}
\fancyhead[R]{\rightmark}
\renewcommand{\subsectionmark}[1]{\markright{\thesubsection\ #1}}
\cfoot{\thepage} % 中央フッターにページ番号
\lhead{}
\rfoot{Haruki Aoki and Hiroki Fukuhara} % 右フッターに名前
\setcounter{tocdepth}{4} % 目次の深さ
\makeatletter
\@addtoreset{equation}{section} % セクwションごとに式番号をリセット
\makeatother

% --- メタ情報 ---
\title{量子光学・量子情報科学ノート}
\date{更新日: \today}
\author{Haruki Aoki and Hiroki Fukuhara}

\begin{document}
  結果のみ示す.
  その他のことについては,別途\href{https://github.com/YutoMSD/physics_notes/blob/main/qft/main.pdf}{ノート}を参照すること.
  なお,\today 現在,ノートは編集中である.
  Schr\"odingr描像では,電場と磁場は,
  \begin{align}
    \hat{E}(\bm{r}) &= \i\qty(\frac{1}{2\pi})^{3/2}\int\dd[3]{k}\sum_{\sigma = 1}^{2}\sqrt{\frac{\hbar\omega_{\bm{k}}}{2\epsilon_0}}\bm{e}_{\bm{k}\sigma}\qty(\hat{a}_{\bm{k}\sigma}\e^{\i\bm{k}\cdot\bm{r}} - \hat{a}_{\bm{k}\sigma}^{\dag}\e^{-\i\bm{k}\cdot\bm{r}}) \\ 
    \hat{B}(\bm{r}) &= \i\qty(\frac{1}{2\pi})^{3/2}\int\dd[3]{k}\sum_{\sigma = 1}^{2}\sqrt{\frac{\hbar}{2\epsilon_0\omega_{\bm{k}}}}\bm{k}\times\bm{e}_{\bm{k}\sigma}\qty(\hat{a}_{\bm{k}\sigma}\e^{\i\bm{k}\cdot\bm{r}} - \hat{a}_{\bm{k}\sigma}^{\dag}\e^{-\i\bm{k}\cdot\bm{r}}) 
  \end{align}
  と量子化される.
  また,Heisenberg描像では,
  \begin{align}
    \hat{E}(\bm{r}, t) &= \i\qty(\frac{1}{2\pi})^{3/2}\int\dd[3]{k}\sum_{\sigma = 1}^{2}\sqrt{\frac{\hbar\omega_{\bm{k}}}{2\epsilon_0}}\bm{e}_{\bm{k}\sigma}\qty[\hat{a}_{\bm{k}\sigma}\exp\qty{\i\qty(\bm{k}\cdot\bm{r} - \omega_{\bm{k}}t)} - \hat{a}_{\bm{k}\sigma}^{\dag}\exp\qty{-\i\qty(\bm{k}\cdot\bm{r} - \omega_{\bm{k}}t)}] \\ 
    \hat{B}(\bm{r}, t) &= \i\qty(\frac{1}{2\pi})^{3/2}\int\dd[3]{k}\sum_{\sigma = 1}^{2}\sqrt{\frac{\hbar}{2\epsilon_0\omega_{\bm{k}}}}\bm{k}\times\bm{e}_{\bm{k}\sigma}\qty[\hat{a}_{\bm{k}\sigma}\exp\qty{\i\qty(\bm{k}\cdot\bm{r} - \omega_{\bm{k}}t)} - \hat{a}_{\bm{k}\sigma}^{\dag}\exp\qty{-\i\qty(\bm{k}\cdot\bm{r} - \omega_{\bm{k}}t)}]
  \end{align}
  と書ける.
\end{document} % 場の量子論を頑張って勉強するのでちょっとまって plz
    \section{ビームスプリッタ}
      \documentclass{report}
\usepackage{luatexja} % LuaTeXで日本語を使うためのパッケージ
\usepackage{luatexja-fontspec} % LuaTeX用の日本語フォント設定

% --- 数学関連 ---
\usepackage{amsmath, amssymb, amsfonts, mathtools, bm, amsthm} % 基本的な数学パッケージ
\usepackage{type1cm, upgreek} % 数式フォントとギリシャ文字k
\usepackage{physics, mhchem} % 物理や化学の記号や式の表記を簡単にする

% --- 表関連 ---
\usepackage{multirow, longtable, tabularx, array, colortbl, dcolumn, diagbox} % 表のレイアウトを柔軟にする
\usepackage{tablefootnote, truthtable} % 表中に注釈を追加、真理値表
\usepackage{tabularray} % 高度な表組みレイアウト

% --- グラフィック関連 ---
\usepackage{tikz, graphicx} % 図の描画と画像の挿入
% \usepackage{background} % ウォーターマークの設定
\usepackage{caption, subcaption} % 図や表のキャプション設定
\usepackage{float, here} % 図や表の位置指定

% --- レイアウトとページ設定 ---
\usepackage{fancyhdr} % ページヘッダー、フッター、余白の設定
\usepackage[top = 20truemm, bottom = 20truemm, left = 20truemm, right = 20truemm]{geometry}
\usepackage{fancybox, ascmac} % ボックスのデザイン

% --- 色とスタイル ---
\usepackage{xcolor, color, colortbl, tcolorbox} % 色とカラーボックス
\usepackage{listings, jvlisting} % コードの色付けとフォーマット

% --- 参考文献関連 ---
\usepackage{biblatex, usebib} % 参考文献の管理と挿入
\usepackage{url, hyperref} % URLとリンクの設定

% --- その他の便利なパッケージ ---
\usepackage{footmisc} % 脚注のカスタマイズ
\usepackage{multicol} % 複数段組
\usepackage{comment} % コメントアウトの拡張
\usepackage{siunitx} % 単位の表記
\usepackage{docmute}
% \usepackage{appendix}
% --- tcolorboxとtikzの設定 ---
\tcbuselibrary{theorems, breakable} % 定理のボックスと改ページ設定
\usetikzlibrary{decorations.markings, arrows.meta, calc} % tikzの装飾や矢印の設定

% --- 定理スタイルと数式設定 ---
\theoremstyle{definition} % 定義スタイル
\numberwithin{equation}{section} % 式番号をサブセクション単位でリセット

% --- hyperrefの設定 ---
\hypersetup{
  setpagesize = false,
  bookmarks = true,
  bookmarksdepth = tocdepth,
  bookmarksnumbered = true,
  colorlinks = false,
  pdftitle = {}, % PDFタイトル
  pdfsubject = {}, % PDFサブジェクト
  pdfauthor = {}, % PDF作者
  pdfkeywords = {} % PDFキーワード
}

% --- siunitxの設定 ---
\sisetup{
  table-format = 1.5, % 小数点以下の桁数
  table-number-alignment = center, % 数値の中央揃え
}


% --- その他の設定 ---
\allowdisplaybreaks % 数式の途中改ページ許可
\newcolumntype{t}{!{\vrule width 0.1pt}} % 新しいカラムタイプ
\newcolumntype{b}{!{\vrule width 1.5pt}} % 太いカラム
\UseTblrLibrary{amsmath, booktabs, counter, diagbox, functional, hook, html, nameref, siunitx, varwidth, zref} % tabularrayのライブラリ
\setlength{\columnseprule}{0.4pt} % カラム区切り線の太さ
\captionsetup[figure]{font = bf} % 図のキャプションの太字設定
\captionsetup[table]{font = bf} % 表のキャプションの太字設定
\captionsetup[lstlisting]{font = bf} % コードのキャプションの太字設定
\captionsetup[subfigure]{font = bf, labelformat = simple} % サブ図のキャプション設定
\setcounter{secnumdepth}{4} % セクションの深さ設定
\newcolumntype{d}{D{.}{.}{5}} % 数値のカラム
\newcolumntype{M}[1]{>{\centering\arraybackslash}m{#1}} % センター揃えのカラム
\DeclareMathOperator{\diag}{diag}
\everymath{\displaystyle} % 数式のスタイル
\newcommand{\inner}[2]{\left\langle #1, #2 \right\rangle}
\renewcommand{\figurename}{図}
\renewcommand{\i}{\mathrm{i}} % 複素数単位i
\renewcommand{\laplacian}{\grad^2} % ラプラシアンの記号
\renewcommand{\thesubfigure}{(\alph{subfigure})} % サブ図の番号形式
\newcommand{\m}[3]{\multicolumn{#1}{#2}{#3}} % マルチカラムのショートカット
\renewcommand{\r}[1]{\mathrm{#1}} % mathrmのショートカット
\newcommand{\e}{\mathrm{e}} % 自然対数の底e
\newcommand{\Ef}{E_{\mathrm{F}}} % フェルミエネルギー
\renewcommand{\c}{\si{\degreeCelsius}} % 摂氏記号
\renewcommand{\d}{\r{d}} % d記号
\renewcommand{\t}[1]{\texttt{#1}} % タイプライタフォント
\newcommand{\kb}{k_{\mathrm{B}}} % ボルツマン定数
% \renewcommand{\phi}{\varphi} % ϕをφに変更
\renewcommand{\epsilon}{\varepsilon}
\newcommand{\fullref}[1]{\textbf{\ref{#1} \nameref{#1}}}
\newcommand{\reff}[1]{\textbf{図\ref{#1}}} % 図参照のショートカット
\newcommand{\reft}[1]{\textbf{表\ref{#1}}} % 表参照のショートカット
\newcommand{\refe}[1]{\textbf{式\eqref{#1}}} % 式参照のショートカット
\newcommand{\refp}[1]{\textbf{コード\ref{#1}}} % コード参照のショートカット
\renewcommand{\lstlistingname}{コード} % コードリストの名前
\renewcommand{\theequation}{\thesection.\arabic{equation}} % 式番号の形式
\renewcommand{\footrulewidth}{0.4pt} % フッターの線
\newcommand{\mar}[1]{\textcircled{\scriptsize #1}} % 丸囲み文字
\newcommand{\combination}[2]{{}_{#1} \mathrm{C}_{#2}} % 組み合わせ
\newcommand{\thline}{\noalign{\hrule height 0.1pt}} % 細い横線
\newcommand{\bhline}{\noalign{\hrule height 1.5pt}} % 太い横線

% --- カスタム色定義 ---
\definecolor{burgundy}{rgb}{0.5, 0.0, 0.13} % バーガンディ色
\definecolor{charcoal}{rgb}{0.21, 0.27, 0.31} % チャコール色
\definecolor{forest}{rgb}{0.0, 0.35, 0} % 森の緑色

% --- カスタム定理環境の定義 ---
\newtcbtheorem[number within = chapter]{myexc}{練習問題}{
  fonttitle = \gtfamily\sffamily\bfseries\upshape,
  colframe = forest,
  colback = forest!2!white,
  rightrule = 1pt,
  leftrule = 1pt,
  bottomrule = 2pt,
  colbacktitle = forest,
  theorem style = standard,
  breakable,
  arc = 0pt,
}{exc-ref}
\newtcbtheorem[number within = chapter]{myprop}{命題}{
  fonttitle = \gtfamily\sffamily\bfseries\upshape,
  colframe = blue!50!black,
  colback = blue!50!black!2!white,
  rightrule = 1pt,
  leftrule = 1pt,
  bottomrule = 2pt,
  colbacktitle = blue!50!black,
  theorem style = standard,
  breakable,
  arc = 0pt
}{proposition-ref}
\newtcbtheorem[number within = chapter]{myrem}{注意}{
  fonttitle = \gtfamily\sffamily\bfseries\upshape,
  colframe = yellow!20!black,
  colback = yellow!50,
  rightrule = 1pt,
  leftrule = 1pt,
  bottomrule = 2pt,
  colbacktitle = yellow!20!black,
  theorem style = standard,
  breakable,
  arc = 0pt
}{remark-ref}
\newtcbtheorem[number within = chapter]{myex}{例題}{
  fonttitle = \gtfamily\sffamily\bfseries\upshape,
  colframe = black,
  colback = white,
  rightrule = 1pt,
  leftrule = 1pt,
  bottomrule = 2pt,
  colbacktitle = black,
  theorem style = standard,
  breakable,
  arc = 0pt
}{example-ref}
\newtcbtheorem[number within = chapter]{exc}{Requirement}{myexc}{exc-ref}
\newcommand{\rqref}[1]{{\bfseries\sffamily 練習問題 \ref{exc-ref:#1}}}
\newtcbtheorem[number within = chapter]{definition}{Definition}{mydef}{definition-ref}
\newcommand{\dfref}[1]{{\bfseries\sffamily 定義 \ref{definition-ref:#1}}}
\newtcbtheorem[number within = chapter]{prop}{命題}{myprop}{proposition-ref}
\newcommand{\prref}[1]{{\bfseries\sffamily 命題 \ref{proposition-ref:#1}}}
\newtcbtheorem[number within = chapter]{rem}{注意}{myrem}{remark-ref}
\newcommand{\rmref}[1]{{\bfseries\sffamily 注意 \ref{remark-ref:#1}}}
\newtcbtheorem[number within = chapter]{ex}{例題}{myex}{example-ref}
\newcommand{\exref}[1]{{\bfseries\sffamily 例題 \ref{example-ref:#1}}}
% --- 再定義コマンド ---
% \mathtoolsset{showonlyrefs=true} % 必要な式番号のみ表示
\pagestyle{fancy} % ヘッダー・フッターのスタイル設定
% \chead{応用量子物性講義ノート} % 中央ヘッダー
% \rhead{}
\fancyhead[R]{\rightmark}
\renewcommand{\subsectionmark}[1]{\markright{\thesubsection\ #1}}
\cfoot{\thepage} % 中央フッターにページ番号
\lhead{}
\rfoot{Haruki Aoki and Hiroki Fukuhara} % 右フッターに名前
\setcounter{tocdepth}{4} % 目次の深さ
\makeatletter
\@addtoreset{equation}{section} % セクwションごとに式番号をリセット
\makeatother

% --- メタ情報 ---
\title{量子光学・量子情報科学ノート}
\date{更新日: \today}
\author{Haruki Aoki and Hiroki Fukuhara}

\begin{document}
  \subsection{電磁場のハミルトニアン}
    前節での議論により,系のハミルトニアンは,
    \begin{align}
      \hat{H}_{\r{sys}} = \int\dd[3]{k} \sum_{\sigma = 1}^{2}\frac{\hbar\omega_{\bm{k}}}{2}\qty(\hat{a}_{\bm{k\sigma}}^{\dag}\hat{a}_{\bm{k}\sigma} + \hat{a}_{\bm{k}\sigma}\hat{a}_{\bm{k}\sigma}^{\dag})
    \end{align}
    と書けるのであった.
    以下では,簡単のために,1方向成分・シングルモードの波を考える.
    \begin{align}
      \hat{H}_{\r{sys}} &= \frac{\hbar\omega}{2}\qty(\hat{a}^{\dag}\hat{a} + \hat{a}\hat{a}^{\dag}) \\ 
      &= \hbar\omega\qty(\hat{a}^{\dag}\hat{a} + \frac{1}{2})
    \end{align}
    と書ける.
    屈折率が$n$の物質中では\footnote{謎である.屈折率により波動は変化しないはずである.},
    \begin{align}
      \hat{H}_{n, \r{sys}} = \frac{\hbar\omega}{n}\qty(\hat{a}^{\dag}\hat{a} + \frac{1}{2})
    \end{align}
    と書ける.
  \subsection{数学的準備}\label{unitary-transform}
    ユニタリ行列は一般に,
    \begin{align}
      U = \e^{\i\Lambda/2}
      \mqty(
        \e^{\i\Psi/2} & 0 \\ 
        0 & \e^{-\i\Psi/2}
      )
      \mqty(
        \cos(\Theta/2) & \sin(\Theta/2) \\ 
        -\sin(\Theta/2) & \cos(\Theta/2)
      )\mqty(
        \e^{\i\Phi/2} & 0 \\ 
        0 & \e^{-\i\Phi/2}
      )
    \end{align}
    と分解できる.
    具体的に$U$を計算すると,
    \begin{align}
      U &= \e^{\i\Lambda/2}
      \mqty(
        \e^{\i\Psi/2} & 0 \\ 
        0 & \e^{-\i\Psi/2}
      )
      \mqty(
        \cos(\Theta/2) & \sin(\Theta/2) \\ 
        -\sin(\Theta/2) & \cos(\Theta/2)
      )\mqty(
        \e^{\i\Phi/2} & 0 \\ 
        0 & \e^{-\i\Phi/2}
      ) \\ 
      &= \e^{\i\Lambda/2}\mqty(
        \e^{\i\Psi/2}\cos(\Theta/2) & \e^{\i\Psi/2}\sin(\Theta/2) \\ 
        -\e^{-\i\Psi/2}\sin(\Theta/2) & \e^{-\i\Psi/2}\cos(\Theta/2)
      )\mqty(
        \e^{\i\Phi/2} & 0 \\ 
        0 & \e^{-\i\Phi/2}
      ) \\ 
      &= \e^{\i\Lambda/2}\mqty(
        \e^{\i(\Psi + \Psi) / 2}\cos(\Theta/2) & \e^{\i(\Psi - \Phi) / 2}\sin(\Theta/2) \\ 
        -\e^{-\i(\Psi - \Phi) / 2}\sin(\Theta/2) & \e^{-\i(\Psi + \Phi) / 2}\cos(\Theta/2)
      )\label{unitary-representation}
    \end{align}
    であり,$\alpha = \Psi + \Phi$,$\beta = \Psi - \Phi$とすると,
    \begin{align}
      U &= \e^{\i\Lambda/2}\mqty(
        \e^{\i\alpha / 2}\cos(\Theta/2) & \e^{\i\beta / 2}\sin(\Theta/2) \\ 
        -\e^{-\i\beta / 2}\sin(\Theta/2) & \e^{-\i\alpha / 2}\cos(\Theta/2)
      ) \\ 
      &= \mqty(
        \e^{\i(\Lambda + \alpha) / 2}\cos(\Theta/2) & \e^{\i(\Lambda + \beta) / 2}\sin(\Theta/2) \\ 
        -\e^{\i(\Lambda - \beta) / 2}\sin(\Theta/2) & \e^{\i(\Lambda - \alpha) / 2}\cos(\Theta/2)
      ) 
    \end{align}
    と書ける.
    \begin{proof}
      任意$2\times 2$の行列は,実数$r_{ij}$と$\theta_{ij}$を用いて,
      \begin{align}
        M = \mqty(
          r_{11}\e^{\i\theta_{11}} & r_{12}\e^{\i\theta_{12}} \\ 
          r_{21}\e^{\i\theta_{21}} & r_{22}\e^{\i\theta_{22}} \\ 
        )
      \end{align}
      と書けて,
      \begin{align}
        M^{\dag}M &= \mqty(
          r_{11}\e^{-\i\theta_{11}} & r_{21}\e^{-\i\theta_{21}} \\ 
          r_{12}\e^{-\i\theta_{12}} & r_{22}\e^{-\i\theta_{22}} \\ 
        )\mqty(
          r_{11}\e^{\i\theta_{11}} & r_{12}\e^{\i\theta_{12}} \\ 
          r_{21}\e^{\i\theta_{21}} & r_{22}\e^{\i\theta_{22}} \\ 
        ) \\ 
        &= \mqty(
          r_{11}^2 + r_{21}^2 & r_{11}r_{12}\e^{-\i(\theta_{11} - \theta_{12})} + r_{21}r_{22}\e^{-\i(\theta_{21} - \theta_{22})} \\ 
          r_{11}r_{12}\e^{\i(\theta_{11} - \theta_{12})} + r_{21}r_{22}\e^{\i(\theta_{21} - \theta_{22})} & r_{12}^2 + r_{22}^2 \\ 
        ) \\ 
        MM^{\dag} &= \mqty(
          r_{11}\e^{\i\theta_{11}} & r_{12}\e^{\i\theta_{12}} \\ 
          r_{21}\e^{\i\theta_{21}} & r_{22}\e^{\i\theta_{22}} \\ 
        )\mqty(
          r_{11}\e^{-\i\theta_{11}} & r_{21}\e^{-\i\theta_{21}} \\ 
          r_{12}\e^{-\i\theta_{12}} & r_{22}\e^{-\i\theta_{22}} \\ 
        ) \\ 
        &= \mqty(
          r_{11}^2 + r_{12}^2 & r_{11}r_{21}\e^{\i(\theta_{11} - \theta_{21})} + r_{11}r_{22}\e^{\i(\theta_{12} - \theta_{22})} \\ 
          r_{11}r_{21}\e^{-\i(\theta_{11} - \theta_{21})} + r_{12}r_{22}\e^{-\i(\theta_{12} - \theta_{22})} & r_{21}^2 + r_{22}^2 \\ 
        )
      \end{align}
      となる.$M$がユニタリ行列であることの必要十分条件は,
      \begin{align}
        r_{11}^2 + r_{21}^2 = 1 \label{r11-r21}\\ 
        r_{12}^2 + r_{22}^2 = 1 \label{r12-r22}\\ 
        r_{11}^2 + r_{12}^2 = 1 \label{r11-r12}\\ 
        r_{21}^2 + r_{22}^2 = 1 \label{r21-r22}\\ 
        r_{11}r_{12}\e^{\i(\theta_{11} - \theta_{12})} + r_{21}r_{22}\e^{\i(\theta_{21} - \theta_{22})} = 0 \label{angle-1}\\ 
        r_{11}r_{21}\e^{\i(\theta_{11} - \theta_{21})} + r_{11}r_{22}\e^{\i(\theta_{12} - \theta_{22})} = 0 \label{angle-2}
      \end{align}
      である.$M^{\dag}M$や$MM^{\dag}$の非対角成分は複素共役になっていることに注意する.
      \refe{r11-r21}から\refe{r21-r22}を満たすような$r_{ij}$の組は,実数$\Theta$を用いて,
      \begin{align}
        r_{11} = r_{22} = \cos(\Theta/2) \\ 
        r_{12} = -r_{21} = \sin(\Theta/2)
      \end{align}
      なるものである.
      また,これらの$r_{ij}$の値を\refe{angle-1}と\refe{angle-2}に代入すると,
      \begin{align}
        \e^{\i(\theta_{11} - \theta_{12})} - \e^{\i(\theta_{21} - \theta_{22})} = 0 \\ 
        -\e^{\i(\theta_{11} - \theta_{21})} + \e^{\i(\theta_{12} - \theta_{22})} = 0
      \end{align}
      が成立する.
      \begin{align}
        \Phi = \theta_{11} - \theta_{12} = \theta_{21} - \theta_{22} \\ 
        \Psi = \theta_{11} - \theta_{21} = \theta_{12} - \theta_{22} \\ 
      \end{align}
      とすると,
      \begin{align}
        \theta_{11} = \frac{\Lambda + \Psi + \Phi}{2} \\ 
        \theta_{12} = \frac{\Lambda + \Psi - \Phi}{2} \\ 
        \theta_{21} = \frac{\Lambda - \Psi + \Phi}{2} \\ 
        \theta_{22} = \frac{\Lambda - \Psi - \Phi}{2}
      \end{align}
      となり,\refe{unitary-representation}を得る.
      つまり,任意のユニタリ行列は\refe{unitary-representation}で書けることが示された.
    \end{proof}
    実際に\refe{unitary-representation}がユニタリ行列であることを確かめる.
    \begin{align}
      U^{\dag}U &= \e^{-\i\Lambda/2}\mqty(
        \e^{-\i\alpha / 2}\cos(\Theta/2) & -\e^{\i\beta / 2}\sin(\Theta/2) \\ 
        \e^{-\i\beta / 2}\sin(\Theta/2) & \e^{\i\alpha / 2}\cos(\Theta/2)
      )
      \e^{\i\Lambda/2}\mqty(
        \e^{\i\alpha / 2}\cos(\Theta/2) & \e^{\i\beta / 2}\sin(\Theta/2) \\ 
        -\e^{-\i\beta / 2}\sin(\Theta/2) & \e^{-\i\alpha / 2}\cos(\Theta/2)
      )
      = \mqty(1 & 0 \\ 0 & 1) \\ 
      UU^{\dag} &= \e^{\i\Lambda/2}\mqty(
        \e^{\i\alpha / 2}\cos(\Theta/2) & \e^{\i\beta / 2}\sin(\Theta/2) \\ 
        -\e^{-\i\beta / 2}\sin(\Theta/2) & \e^{-\i\alpha / 2}\cos(\Theta/2)
      )
      \e^{-\i\Lambda/2}\mqty(
        \e^{-\i\alpha / 2}\cos(\Theta/2) & -\e^{\i\beta / 2}\sin(\Theta/2) \\ 
        \e^{-\i\beta / 2}\sin(\Theta/2) & \e^{\i\alpha / 2}\cos(\Theta/2)
      )
      = \mqty(1 & 0 \\ 0 & 1)
    \end{align}
    となり,$U$はユニタリ行列であることが分かる.
  \subsection{ビームスプリッタ行列}
    2入力2出力のビームスプリッタを考える.
    $E_1$と$E_2$の電場が入射して,$E'_1$と$E'_2$が出力されるとする.
    古典的に考えると,
    \begin{align}
      \mqty(E'_1 \\ E'_2) = \mqty(\xmat*{B}{2}{2})\mqty(E_1 \\ E_2)
    \end{align}
    と書ける.
    このまま電場演算子を中心に議論を進めることはいささか冗長である.
    なぜならば,$\hat{a}_1$と$\hat{a}_1^{\dag}$は複素共役の関係にあるのだから,片方が定まれば自然ともう片方が定まるからだ.
    よって,
    \begin{align}
      \mqty(\hat{a}_1' \\ \hat{a}_2') = \mqty(\xmat*{B}{2}{2})\mqty(\hat{a}_1 \\ \hat{a}_2)
    \end{align}
    と書ける.
    $B$はビームスプリッタ行列という.
    光子数が保存することから,
    \begin{align}
      \hat{a}_1^{\dag}\hat{a}_1 + \hat{a}_2^{\dag}\hat{a}_2 &= \hat{a}_1'^{\dag}\hat{a}_1' + \hat{a}_2'^{\dag}\hat{a}_2' \\
      &= \mqty(B_{11}\hat{a}_1 + B_{12}\hat{a}_2)^{\dag}\mqty(B_{11}\hat{a}_1 + B_{12}\hat{a}_2) + \mqty(B_{21}\hat{a}_1 + B_{22}\hat{a}_2)^{\dag}\mqty(B_{21}\hat{a}_1 + B_{22}\hat{a}_2) \\ 
      &= \mqty(B_{11}^*\hat{a}_1^{\dag} + B_{12}^*\hat{a}_2^{\dag})\mqty(B_{11}\hat{a}_1 + B_{12}\hat{a}_2) + \mqty(B_{21}^*\hat{a}_1^{\dag} + B_{22}^*\hat{a}_2^{\dag})\mqty(B_{21}\hat{a}_1 + B_{22}\hat{a}_2) \\ 
      &= \mqty(\abs{B_{11}}^2 + \abs{B_{21}}^2)\hat{a}_1^{\dag}\hat{a}_1 + \mqty(\abs{B_{12}}^2 + \abs{B_{22}}^2)\hat{a}_2^{\dag}\hat{a}_2 + \mqty(B_{11}^*B_{12} + B_{21}^*B_{22})\hat{a}_1^{\dag}\hat{a}_2 + \qty(B_{12}^*B_{11} + B_{21}^*B_{21})\hat{a}_2^{\dag}\hat{a}_1 \\ 
      &= \mqty(\abs{B_{11}}^2 + \abs{B_{21}}^2)\hat{a}_1^{\dag}\hat{a}_1 + \mqty(\abs{B_{12}}^2 + \abs{B_{22}}^2)\hat{a}_2^{\dag}\hat{a}_2 + \mqty(B_{11}^*B_{12} + B_{21}^*B_{22})\hat{a}_1^{\dag}\hat{a}_2 + \mqty(B_{11}^*B_{12} + B_{21}^*B_{22})^*\hat{a}_2^{\dag}\hat{a}_1
    \end{align}
    となり,
    \begin{align}
      \begin{dcases}
        \abs{B_{11}}^2 + \abs{B_{21}}^2 = \abs{B_{12}}^2 + \abs{B_{22}}^2 = 1 \\ 
        B_{11}^*B_{12} + B_{21}^*B_{22} = 0\\ 
      \end{dcases} \\ 
      \Leftrightarrow 
      B^{\dag}B = \mqty(B_{11}^* & B_{21}^* \\ B_{12}^* & B_{22}^*)\mqty(B_{11} & B_{12} \\ B_{21} & B_{22}) = \mqty(1 & 0 \\ 0 & 1)
    \end{align}
    となればよい.
    つまり,ビームスプリッタ行列$B$がユニタリ行列であれば良い.
    \ref{unitary-transform}での議論において,
    \begin{align}
      \mqty(\e^{\i\Psi/2} & 0 \\ 0 & \e^{-\i\Psi/2})
    \end{align}
    は2つの入力電場$E_1$,$E_2$に位相差をかけること,
    \begin{align}
      \mqty(\e^{\i\Phi/2} & 0 \\ 0 & \e^{-\i\Phi/2})
    \end{align}
    は2つの出力電場$E_1'$,$E_2'$に位相差をかけること,
    \begin{align}
      \e^{\i\Lambda/2}
    \end{align}
    は2つの出力電場場$E_1'$,$E_2'$に共通するグローバル位相を書けることに対応するから,実験のセットアップとして,
    \begin{align}
      \Lambda = \Psi = \Phi = 0
    \end{align}
    とすることができる.
    また,透過率$T$と反射率$R$を,
    \begin{align}
      \sqrt{T} &\coloneqq \cos(\Theta/2) \\ 
      \sqrt{R} &\coloneqq -\sin(\Theta/2)
    \end{align}
    と定義すれば,ビームスプリッタ行列$B$は,
    \begin{align}
      B &= \mqty(
        \cos(\Theta/2) & \sin(\Theta/2) \\ 
        -\sin(\Theta/2) & \cos(\Theta/2)
      ) \\ 
      &= \mqty(\sqrt{T} & -\sqrt{R} \\ \sqrt{R} & \sqrt{T})
    \end{align}
    と書ける.
  \subsection{ビームスプリッタハミルトニアン}
\end{document}
    \section{コヒーレント状態}
      \documentclass{report}
\usepackage{luatexja} % LuaTeXで日本語を使うためのパッケージ
\usepackage{luatexja-fontspec} % LuaTeX用の日本語フォント設定

% --- 数学関連 ---
\usepackage{amsmath, amssymb, amsfonts, mathtools, bm, amsthm} % 基本的な数学パッケージ
\usepackage{type1cm, upgreek} % 数式フォントとギリシャ文字k
\usepackage{physics, mhchem} % 物理や化学の記号や式の表記を簡単にする

% --- 表関連 ---
\usepackage{multirow, longtable, tabularx, array, colortbl, dcolumn, diagbox} % 表のレイアウトを柔軟にする
\usepackage{tablefootnote, truthtable} % 表中に注釈を追加、真理値表
\usepackage{tabularray} % 高度な表組みレイアウト

% --- グラフィック関連 ---
\usepackage{tikz, graphicx} % 図の描画と画像の挿入
% \usepackage{background} % ウォーターマークの設定
\usepackage{caption, subcaption} % 図や表のキャプション設定
\usepackage{float, here} % 図や表の位置指定

% --- レイアウトとページ設定 ---
\usepackage{fancyhdr} % ページヘッダー、フッター、余白の設定
\usepackage[top = 20truemm, bottom = 20truemm, left = 20truemm, right = 20truemm]{geometry}
\usepackage{fancybox, ascmac} % ボックスのデザイン

% --- 色とスタイル ---
\usepackage{xcolor, color, colortbl, tcolorbox} % 色とカラーボックス
\usepackage{listings, jvlisting} % コードの色付けとフォーマット

% --- 参考文献関連 ---
\usepackage{biblatex, usebib} % 参考文献の管理と挿入
\usepackage{url, hyperref} % URLとリンクの設定

% --- その他の便利なパッケージ ---
\usepackage{footmisc} % 脚注のカスタマイズ
\usepackage{multicol} % 複数段組
\usepackage{comment} % コメントアウトの拡張
\usepackage{siunitx} % 単位の表記
\usepackage{docmute}
% \usepackage{appendix}
% --- tcolorboxとtikzの設定 ---
\tcbuselibrary{theorems, breakable} % 定理のボックスと改ページ設定
\usetikzlibrary{decorations.markings, arrows.meta, calc} % tikzの装飾や矢印の設定

% --- 定理スタイルと数式設定 ---
\theoremstyle{definition} % 定義スタイル
\numberwithin{equation}{section} % 式番号をサブセクション単位でリセット

% --- hyperrefの設定 ---
\hypersetup{
  setpagesize = false,
  bookmarks = true,
  bookmarksdepth = tocdepth,
  bookmarksnumbered = true,
  colorlinks = false,
  pdftitle = {}, % PDFタイトル
  pdfsubject = {}, % PDFサブジェクト
  pdfauthor = {}, % PDF作者
  pdfkeywords = {} % PDFキーワード
}

% --- siunitxの設定 ---
\sisetup{
  table-format = 1.5, % 小数点以下の桁数
  table-number-alignment = center, % 数値の中央揃え
}


% --- その他の設定 ---
\allowdisplaybreaks % 数式の途中改ページ許可
\newcolumntype{t}{!{\vrule width 0.1pt}} % 新しいカラムタイプ
\newcolumntype{b}{!{\vrule width 1.5pt}} % 太いカラム
\UseTblrLibrary{amsmath, booktabs, counter, diagbox, functional, hook, html, nameref, siunitx, varwidth, zref} % tabularrayのライブラリ
\setlength{\columnseprule}{0.4pt} % カラム区切り線の太さ
\captionsetup[figure]{font = bf} % 図のキャプションの太字設定
\captionsetup[table]{font = bf} % 表のキャプションの太字設定
\captionsetup[lstlisting]{font = bf} % コードのキャプションの太字設定
\captionsetup[subfigure]{font = bf, labelformat = simple} % サブ図のキャプション設定
\setcounter{secnumdepth}{4} % セクションの深さ設定
\newcolumntype{d}{D{.}{.}{5}} % 数値のカラム
\newcolumntype{M}[1]{>{\centering\arraybackslash}m{#1}} % センター揃えのカラム
\DeclareMathOperator{\diag}{diag}
\everymath{\displaystyle} % 数式のスタイル
\newcommand{\inner}[2]{\left\langle #1, #2 \right\rangle}
\renewcommand{\figurename}{図}
\renewcommand{\i}{\mathrm{i}} % 複素数単位i
\renewcommand{\laplacian}{\grad^2} % ラプラシアンの記号
\renewcommand{\thesubfigure}{(\alph{subfigure})} % サブ図の番号形式
\newcommand{\m}[3]{\multicolumn{#1}{#2}{#3}} % マルチカラムのショートカット
\renewcommand{\r}[1]{\mathrm{#1}} % mathrmのショートカット
\newcommand{\e}{\mathrm{e}} % 自然対数の底e
\newcommand{\Ef}{E_{\mathrm{F}}} % フェルミエネルギー
\renewcommand{\c}{\si{\degreeCelsius}} % 摂氏記号
\renewcommand{\d}{\r{d}} % d記号
\renewcommand{\t}[1]{\texttt{#1}} % タイプライタフォント
\newcommand{\kb}{k_{\mathrm{B}}} % ボルツマン定数
% \renewcommand{\phi}{\varphi} % ϕをφに変更
\renewcommand{\epsilon}{\varepsilon}
\newcommand{\fullref}[1]{\textbf{\ref{#1} \nameref{#1}}}
\newcommand{\reff}[1]{\textbf{図\ref{#1}}} % 図参照のショートカット
\newcommand{\reft}[1]{\textbf{表\ref{#1}}} % 表参照のショートカット
\newcommand{\refe}[1]{\textbf{式\eqref{#1}}} % 式参照のショートカット
\newcommand{\refp}[1]{\textbf{コード\ref{#1}}} % コード参照のショートカット
\renewcommand{\lstlistingname}{コード} % コードリストの名前
\renewcommand{\theequation}{\thesection.\arabic{equation}} % 式番号の形式
\renewcommand{\footrulewidth}{0.4pt} % フッターの線
\newcommand{\mar}[1]{\textcircled{\scriptsize #1}} % 丸囲み文字
\newcommand{\combination}[2]{{}_{#1} \mathrm{C}_{#2}} % 組み合わせ
\newcommand{\thline}{\noalign{\hrule height 0.1pt}} % 細い横線
\newcommand{\bhline}{\noalign{\hrule height 1.5pt}} % 太い横線

% --- カスタム色定義 ---
\definecolor{burgundy}{rgb}{0.5, 0.0, 0.13} % バーガンディ色
\definecolor{charcoal}{rgb}{0.21, 0.27, 0.31} % チャコール色
\definecolor{forest}{rgb}{0.0, 0.35, 0} % 森の緑色

% --- カスタム定理環境の定義 ---
\newtcbtheorem[number within = chapter]{myexc}{練習問題}{
  fonttitle = \gtfamily\sffamily\bfseries\upshape,
  colframe = forest,
  colback = forest!2!white,
  rightrule = 1pt,
  leftrule = 1pt,
  bottomrule = 2pt,
  colbacktitle = forest,
  theorem style = standard,
  breakable,
  arc = 0pt,
}{exc-ref}
\newtcbtheorem[number within = chapter]{myprop}{命題}{
  fonttitle = \gtfamily\sffamily\bfseries\upshape,
  colframe = blue!50!black,
  colback = blue!50!black!2!white,
  rightrule = 1pt,
  leftrule = 1pt,
  bottomrule = 2pt,
  colbacktitle = blue!50!black,
  theorem style = standard,
  breakable,
  arc = 0pt
}{proposition-ref}
\newtcbtheorem[number within = chapter]{myrem}{注意}{
  fonttitle = \gtfamily\sffamily\bfseries\upshape,
  colframe = yellow!20!black,
  colback = yellow!50,
  rightrule = 1pt,
  leftrule = 1pt,
  bottomrule = 2pt,
  colbacktitle = yellow!20!black,
  theorem style = standard,
  breakable,
  arc = 0pt
}{remark-ref}
\newtcbtheorem[number within = chapter]{myex}{例題}{
  fonttitle = \gtfamily\sffamily\bfseries\upshape,
  colframe = black,
  colback = white,
  rightrule = 1pt,
  leftrule = 1pt,
  bottomrule = 2pt,
  colbacktitle = black,
  theorem style = standard,
  breakable,
  arc = 0pt
}{example-ref}
\newtcbtheorem[number within = chapter]{exc}{Requirement}{myexc}{exc-ref}
\newcommand{\rqref}[1]{{\bfseries\sffamily 練習問題 \ref{exc-ref:#1}}}
\newtcbtheorem[number within = chapter]{definition}{Definition}{mydef}{definition-ref}
\newcommand{\dfref}[1]{{\bfseries\sffamily 定義 \ref{definition-ref:#1}}}
\newtcbtheorem[number within = chapter]{prop}{命題}{myprop}{proposition-ref}
\newcommand{\prref}[1]{{\bfseries\sffamily 命題 \ref{proposition-ref:#1}}}
\newtcbtheorem[number within = chapter]{rem}{注意}{myrem}{remark-ref}
\newcommand{\rmref}[1]{{\bfseries\sffamily 注意 \ref{remark-ref:#1}}}
\newtcbtheorem[number within = chapter]{ex}{例題}{myex}{example-ref}
\newcommand{\exref}[1]{{\bfseries\sffamily 例題 \ref{example-ref:#1}}}
% --- 再定義コマンド ---
% \mathtoolsset{showonlyrefs=true} % 必要な式番号のみ表示
\pagestyle{fancy} % ヘッダー・フッターのスタイル設定
% \chead{応用量子物性講義ノート} % 中央ヘッダー
% \rhead{}
\fancyhead[R]{\rightmark}
\renewcommand{\subsectionmark}[1]{\markright{\thesubsection\ #1}}
\cfoot{\thepage} % 中央フッターにページ番号
\lhead{}
\rfoot{Haruki Aoki and Hiroki Fukuhara} % 右フッターに名前
\setcounter{tocdepth}{4} % 目次の深さ
\makeatletter
\@addtoreset{equation}{section} % セクwションごとに式番号をリセット
\makeatother

% --- メタ情報 ---
\title{量子光学・量子情報科学ノート}
\date{更新日: \today}
\author{Haruki Aoki and Hiroki Fukuhara}

\begin{document}
  本節ではコヒーレント状態について議論する.
  コヒーレント状態$\ket{\alpha}$は,
  \begin{align}
    \hat{a}\ket{\alpha} = \alpha\ket{\alpha}\label{coherent-def}
  \end{align}
  なる状態である.
  また,$\alpha = \abs{\alpha}\e^{\i\theta}$となるように$\theta$を定義しておく.
  \subsection{物理量の平均値・分散}
    具体的な$\ket{\alpha}$の形を知らなくても,いくつかの物理量の平均値と分散については調べることができる.
    まず,電場の期待値を調べる.
    電場演算子$\hat{\bm{E}}(\bm{r}, t)$を自然単位系を用いて書くと,
    \begin{align}
      \hat{\bm{E}}(\bm{r}, t) = \frac{\i}{2}\bm{e}\qty(\hat{a}\exp\qty{\i(\bm{k}\cdot\bm{r} - \omega t)} - \hat{a}^{\dag}\exp\qty{-\i(\bm{k}\cdot\bm{r} - \omega t)})
    \end{align}
    であるから,
    \begin{align}
      \ev{\bm{E}(\bm{r}, t)} &= \mel**{\alpha}{\bm{E}(\bm{r}, t)}{\alpha} \\ 
      &= \frac{\i}{2}\bm{e}\qty(\mel**{\alpha}{\hat{a}}{\alpha}\exp\qty{\i(\bm{k}\cdot\bm{r} - \omega t)} - \mel**{\alpha}{\hat{a}^{\dag}}{\alpha}\exp\qty{-\i(\bm{k}\cdot\bm{r} - \omega t)}) \\ 
      &= -\frac{1}{2\i}\bm{e}\qty(\alpha\exp\qty{\i(\bm{k}\cdot\bm{r} - \omega t)} - \alpha^*\exp\qty{-\i(\bm{k}\cdot\bm{r} - \omega t)}) \\ 
      &= -\frac{1}{2\i}\bm{e}\qty(\abs{\alpha}\e^{\i\theta}\exp\qty{\i(\bm{k}\cdot\bm{r} - \omega t)} - \abs{\alpha}\e^{-\i\theta}\exp\qty{-\i(\bm{k}\cdot\bm{r} - \omega t)}) \\ 
      &= -\abs{\alpha}\bm{e}\sin\qty(\bm{k}\cdot\bm{r} - \omega t + \theta)
    \end{align}
    である.
    \par
    次に,位置と運動量の平均値と分散について議論する.
    位置演算子と運動量演算子は生成演算子と消滅演算子を用いて,
    \begin{align}
      \hat{x} &\coloneqq \frac{1}{2}\qty(\hat{a} + \hat{a}^{\dag}) \\ 
      \hat{p} &\coloneqq \frac{1}{2\i}\qty(\hat{a} - \hat{a}^{\dag})
    \end{align}
    と書けるから,
    \begin{align}
      \ev{x} &= \mel**{\alpha}{\frac{1}{2}\qty(\hat{a} + \hat{a}^{\dag})}{\alpha} \\ 
      &= \frac{1}{2}\qty(\alpha + \alpha^*) \\ 
      \ev{x^2} &= \mel**{\alpha}{\frac{1}{4}\qty(\hat{a} + \hat{a}^{\dag})^2}{\alpha} \\ 
      &= \frac{1}{4}\mel**{\alpha}{\hat{a}^2}{\alpha} + \frac{1}{4}\mel**{\alpha}{\hat{a}\hat{a}^{\dag}}{\alpha} + \frac{1}{4}\mel**{\alpha}{\hat{a}^{\dag}\hat{a}}{\alpha} + \frac{1}{4}\mel**{\alpha}{\hat{a}^{\dag 2}}{\alpha} \\ 
      &= \frac{1}{4}\alpha^2 + \frac{1}{4}\mel**{\alpha}{\qty(\hat{a}^{\dag}\hat{a} + 1)}{\alpha} + \frac{1}{4}\abs{\alpha}^2 + \frac{1}{4}\alpha^{*2} \\ 
      &= \frac{1}{4}\alpha^2 + \frac{1}{4}\abs{\alpha}^2 + \frac{1}{4} + \frac{1}{4}\abs{\alpha}^2 + \frac{1}{4}\alpha^{*2} \\ 
      &= \frac{1}{4}\qty(\alpha + \alpha^*)^2 + \frac{1}{4} \\ 
      \ev{p} &= \mel**{\alpha}{\frac{1}{2\i}\qty(\hat{a} - \hat{a}^{\dag})}{\alpha} \\ 
      &= \frac{1}{2\i}\qty(\alpha - \alpha^*) \\ 
      \ev{p^2} &= \mel**{\alpha}{-\frac{1}{4}\qty(\hat{a} - \hat{a}^{\dag})^2}{\alpha} \\ 
      &= -\frac{1}{4}\mel**{\alpha}{\hat{a}^2}{\alpha} + \frac{1}{4}\mel**{\alpha}{\hat{a}\hat{a}^{\dag}}{\alpha} + \frac{1}{4}\mel**{\alpha}{\hat{a}^{\dag}\hat{a}}{\alpha} - \frac{1}{4}\mel**{\alpha}{\hat{a}^{\dag 2}}{\alpha} \\ 
      &= -\frac{1}{4}\alpha^2 + \frac{1}{4}\mel**{\alpha}{\qty(\hat{a}^{\dag}\hat{a} + 1)}{\alpha} + \frac{1}{4}\abs{\alpha}^2 - \frac{1}{4}\alpha^{* 2} \\ 
      &= -\frac{1}{4}\alpha^2 + \frac{1}{4}\abs{\alpha}^2 + \frac{1}{4} +\frac{1}{4}\abs{\alpha}^2 - \frac{1}{4}\alpha^{* 2} \\ 
      &= -\frac{1}{4}\qty(\alpha + \alpha^*)^2 + \frac{1}{4}
    \end{align}
    より,
    \begin{align}
      \Delta x_{\r{coh}} &\coloneqq \sqrt{\ev{x^2} - \ev{x}^2} = \frac{1}{4} \\ 
      \Delta p_{\r{coh}} &\coloneqq \sqrt{\ev{p^2} - \ev{p}^2} = \frac{1}{4}
    \end{align}
    となる.
  \subsection{個数状態での展開}
    コヒーレント状態$\ket{\alpha}$を,Hermite演算子である$\hat{n}$の固有状態である個数状態$\ket{n}$で展開することを考える.
    \begin{align}
      \ket{\alpha} = \sum_{n = 0}^{\infty}w_n\ket{n}\label{coherent-expantion}
    \end{align}
    である.
    \refe{coherent-def}に\refe{coherent-expantion}を代入して,\refe{annihilation}の関係式を用いると,
    \begin{align}
      \hat{a}\ket{\alpha} = \alpha\ket{\alpha} \\ 
      \iff \hat{a}\sum_{n = 0}^{\infty}w_n\ket{n} &= \alpha\sum_{n = 0}^{\infty}w_n\ket{n} \\ 
      \iff \sum_{n = 1}^{\infty}\sqrt{n}w_n\ket{n - 1} &= \sum_{n = 0}^{\infty}\alpha w_n\ket{n} \\ 
      \iff \sum_{n = 0}^{\infty}\sqrt{n + 1}w_{n + 1}\ket{n} &= \sum_{n = 0}^{\infty}\alpha w_n\ket{n}
    \end{align}
    であるから,
    \begin{align}
      \sqrt{n + 1}w_{n + 1} &= \alpha w_n \\ 
      \iff w_n &= \frac{\alpha^n}{\sqrt{n!}}C
    \end{align}
    である.
    $C \coloneqq w_0$と定義した.
    $\ket{\alpha}$の規格化条件より,
    \begin{align}
      1 = \braket{\alpha}{\alpha} &= \qty(\sum_{n}w_n\ket{n})^{\dag}\qty(\sum_{m}w_m\ket{m}) \\ 
      &= \qty(\sum_{n}\frac{\qty(\alpha^*)^n}{\sqrt{n!}}C^*\bra{n})\qty(\sum_{m}\frac{\alpha^m}{\sqrt{m!}}C\ket{m}) \\ 
      &= \abs{C}^2\sum_{n, m}\frac{\qty(\alpha^*)^n\alpha^m}{\sqrt{n!}\sqrt{m!}}\braket{n}{m} \\ 
      &= \abs{C}^2\sum_{n}\frac{\qty(\abs{\alpha}^2)^n}{n!} \\ 
      &= \abs{C}^2\e^{\abs{\alpha}^2}
    \end{align}
    であるから,
    \begin{align}
      C = \exp\qty(-\frac{\abs{\alpha}^2}{2})
    \end{align}
    とすればよい.
    よって,コヒーレント状態は,
    \begin{align}
      \ket{\alpha} = \sum_n \frac{\alpha^n}{\sqrt{n!}}\exp\qty(-\frac{\abs{\alpha}^2}{2})\ket{n}\label{coherent-number-state-expantion}
    \end{align}
    と書ける.
    \refe{coherent-number-state-expantion}より,コヒーレント状態とは,個数状態を$\ket{n}$をPoisson分布に従って重ね合わせたものだと分かる.
    \par
    \refe{coherent-number-state-expantion}の表式より,異なるコヒーレント状態$\ket{\alpha}$と$\ket{\alpha'}$は直交することがわかる.
    実際に計算すると,
    \begin{align}
      \abs{\braket{\alpha}{\alpha'}}^2 &= \abs{\qty(\sum_n \frac{\qty(\alpha^*)^n}{\sqrt{n!}}\exp\qty(-\frac{\abs{\alpha}^2}{2})\bra{n})\qty(\sum_{n'} \frac{\alpha'^{n'}}{\sqrt{{n'}!}}\exp\qty(-\frac{\abs{\alpha'}^2}{2})\ket{n'})}^2 \\ 
      &= \exp\qty{-\qty(\abs{\alpha}^2 + \abs{\alpha'}^2)}\qty(\sum_{n, n'}\frac{\qty(\alpha^*)^n\alpha'^{n'}}{\sqrt{n!}\sqrt{n'!}}\braket{n}{n'})\qty(\sum_{m', m}\frac{\qty(\alpha'^*)^m\alpha^{m'}}{\sqrt{m!}\sqrt{m'!}}\braket{m'}{m}) \\ 
      &= \exp\qty{-\qty(\abs{\alpha}^2 + \abs{\alpha'}^2)}\qty(\sum_{n}\frac{\qty(\alpha^*\alpha')^n}{n!})\qty(\sum_{m}\frac{\qty(\alpha'^*\alpha)^m}{m!}) \\ 
      &= \exp\qty{-\qty(\abs{\alpha}^2 - \alpha^*\alpha' - \alpha'^*\alpha + \abs{\alpha'}^2)} \\ 
      &= \exp\qty(-\abs{\alpha - \alpha'}^2)
    \end{align}
    となる.
  \subsection{調和振動子ハミルトニアンでの時間発展}
    まず,コヒーレント状態$\ket{\alpha}$が調和振動子ハミルトニアン$\hat{H}$があるときにどのように時間発展発展するか調べる.
    系のハミルトニアンは,
    \begin{align}
      \hat{H} = \hbar\omega\qty(\hat{n} + \frac{1}{2})
    \end{align}
    と書けるから,
    \begin{align}
      \ket{\alpha(t)} &= \exp\qty(-\i\frac{\hat{H}}{\hbar}t)\ket{\alpha} \\ 
      &= \exp\qty(-\i\frac{\omega}{2}t)\e^{-\i\omega t \hat{n}}\ket{\alpha} \\ 
      &= \exp\qty(-\i\frac{\omega}{2}t)\sum_{m}\frac{(-\i\omega t)^m}{m!}\hat{n}^m \sum_{n}\frac{\alpha^n}{\sqrt{n!}}\exp\qty(-\frac{\abs{\alpha}^2}{2})\ket{n} \\ 
      &= \exp\qty(-\i\frac{\omega}{2}t)\exp\qty(-\frac{\abs{\alpha}^2}{2}) \sum_{n}\frac{\alpha^n}{\sqrt{n!}}\sum_{m}\frac{(-\i\omega t)^m}{m!}\hat{n}^m\ket{n} \\ 
      &= \exp\qty(-\i\frac{\omega}{2}t)\exp\qty(-\frac{\abs{\alpha}^2}{2}) \sum_{n}\frac{\alpha^n}{\sqrt{n!}}\sum_{m}\frac{(-\i n\omega t)^m}{m!} \ket{n} \\ 
      &= \exp\qty(-\i\frac{\omega}{2}t)\exp\qty(-\frac{\abs{\alpha}^2}{2}) \sum_{n}\frac{\alpha^n}{\sqrt{n!}}\e^{-\i n\omega t}\ket{n} \\ 
      &= \exp\qty(-\i\frac{\omega}{2}t) \sum_{n}\frac{\qty(\alpha\e^{-\i\omega t})^n}{\sqrt{n!}}\exp\qty(-\frac{\abs{\alpha}^2}{2})\ket{n} \\ 
      &= \exp\qty(-\i\frac{\omega}{2}t) \sum_{n}\frac{\qty(\alpha\e^{-\i\omega t})^n}{\sqrt{n!}}\exp\qty(-\frac{\abs{\alpha\e^{-\i\omega t}}^2}{2})\ket{n} \\ 
      &= \exp\qty(-\i\frac{\omega}{2}t)\ket{\alpha\e^{-\i\omega t}}
    \end{align}
    グローバル位相は無視してよいので,
    \begin{align}
      \ket{\alpha(t)} = \ket{\alpha\e^{-\i\omega t}}
    \end{align}
    と分かる.
    \par
    次に,個数状態の時間発展を調べる.
    \begin{align}
      \ket{n(t)} &= \exp\qty(-\i\frac{\hat{H}}{\hbar}t)\ket{n} \\ 
      &= \exp\qty(-\i\frac{\omega}{2}t)\e^{-\i\omega t \hat{n}}\ket{n} \\ 
      &= \exp\qty(-\i\frac{\omega}{2}t)\sum_{m}\frac{(-\i\omega t)^m}{m!}\hat{n}^m \ket{n} \\ 
      &= \exp\qty(-\i\frac{\omega}{2}t)\sum_{m}\frac{(-\i n\omega t)^m}{m!} \ket{n} \\ 
      &= \exp\qty(-\i\frac{\omega}{2}t)\e^{-\i n\omega t} \ket{n}
    \end{align}
    となり,周波数が2倍になったように見える\footnote{らしい}.
  \subsection{レーザのハミルトニアンでの時間発展}
    真空場$\ket{0}$が$\ket{\alpha}$に時間変化するものがレーザである.
    レーザのハミルトニアンを
    \begin{align}
      \hat{H}_{\r{laser}} \coloneqq \i\frac{\hbar}{t}\qty(\alpha\hat{a}^{\dag} - \alpha^*\hat{a})
    \end{align}
    とかける.
    実際,この系における真空状態$\ket{0}$の時間発展は,
    \begin{align}
      \exp\qty(-\i\frac{\hat{H}}{\hbar}t)\ket{0} &= \exp\qty(-\i\cdot\i\frac{\hbar}{t}\qty(\alpha\hat{a}^{\dag} - \alpha^*\hat{a})\frac{t}{\hbar})\ket{0} \\ 
      &= \e^{\alpha\hat{a}^{\dag} - \alpha^*\hat{a}}\ket{0} \\ 
      &= \e^{\alpha\hat{a}^{\dag}}\e^{-\alpha^*\hat{a}}\exp\qty(-\frac{1}{2}\qty[\alpha\hat{a}^{\dag}, -\alpha^*\hat{a}])\ket{0} \\ 
      &= \exp\qty(-\frac{\abs{\alpha}^2}{2})\e^{\alpha\hat{a}^{\dag}}\e^{-\alpha^*\hat{a}}\ket{0} \\ 
      &= \exp\qty(-\frac{\abs{\alpha}^2}{2})\e^{\alpha\hat{a}^{\dag}}\sum_{n = 0}\frac{\qty(-\alpha^*\hat{a})^n}{n!}\ket{0} \\ 
      &= \exp\qty(-\frac{\abs{\alpha}^2}{2})\e^{\alpha\hat{a}^{\dag}}\ket{0} \\ 
      &= \exp\qty(-\frac{\abs{\alpha}^2}{2})\sum_{n = 0}\frac{\qty(\alpha\hat{a}^{\dag})^n}{n!}\ket{0} \\ 
      &= \exp\qty(-\frac{\abs{\alpha}^2}{2})\sum_{n = 0}\frac{\alpha^n}{n!}\qty(\hat{a}^{\dag})^n\ket{0} \\ 
      &= \exp\qty(-\frac{\abs{\alpha}^2}{2})\sum_{n = 0}\frac{\alpha^n}{n!}\sqrt{n!}\ket{0} \\ 
      &= \sum_{n = 0}\frac{\alpha^n}{\sqrt{n!}}\exp\qty(-\frac{\abs{\alpha}^2}{2})\ket{0} \\ 
      &= \ket{\alpha}
    \end{align}
    となり,やはり$\hat{H}_{\r{laser}}$はレーザのハミルトニアンである.
    計算の途中に,2つ目のBaker-Campbell-Hausdorffの公式を用いた.
    \par
    変位演算子$\hat{D}(\alpha)$を,
    \begin{align}
      \hat{D}(\alpha) \coloneqq \e^{\alpha\hat{a}^{\dag} - \alpha^*\hat{a}}
    \end{align}
    と定義する.
    Schr\"odinger描像では系の時間発展を状態ベクトルに押し付けて,
    \begin{align}
      \ket{\alpha} = \hat{D}(\alpha)\ket{0}
    \end{align}
    としたのであった.
    Heisenberg描像で,光の振幅に対応する物理量である消滅演算子$\hat{a}$の時間発展の様子を調べる.
    1つ目のBaker-Campbell-Hausdorffの公式を用いて計算すると,
    \begin{align}
      \hat{D}^{\dag}\hat{a}\hat{D}(\alpha) &= \e^{\qty(\alpha\hat{a}^{\dag} - \alpha^*\hat{a})^{\dag}}\hat{a}\e^{\qty(\alpha\hat{a}^{\dag} - \alpha^*\hat{a})} \\ 
      &= \e^{-\qty(\alpha\hat{a}^{\dag} - \alpha^*\hat{a})}\hat{a}\e^{\qty(\alpha\hat{a}^{\dag} - \alpha^*\hat{a})} \\ 
      &= \hat{a} + \qty[-\qty(\alpha\hat{a}^{\dag} - \alpha^*\hat{a}), \hat{a}] \\ 
      &= \hat{a} + \alpha
    \end{align}
    となる.
\end{document}
    \section{スクイーズド状態}
      
\end{document}