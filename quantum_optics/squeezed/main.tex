\documentclass{report}
\input{../../head.tex}
\begin{document}
  本節では,スクイーズド状態$\ket{\beta}_{\r{g}}$について議論する.
  スクイーズド状態は,消滅演算子$\hat{a}$を変換して得られる$\hat{b}$の固有状態であり,その変換は,
  \begin{align}
    \hat{b} \coloneqq \mu\hat{a} + \nu\hat{a}^{\dag},\ \abs{\mu}^2 - \abs{\nu}^2 = 1
  \end{align}
  と定義され,Bogoliubov変換と呼ばれる.
  縮退パラメトリック過程は,電場の2乗に比例して分極が起こって,その比例係数が$\chi^{(2)}$である非線型結晶に対して,
  周波数$\omega$の入射電場$E_{\r{in}}$と$2\omega$の強い電場$E_3$を入射して,
  2つの電場の差周波$2\omega - \omega = \omega$の周波数の電場$E_{\r{out}}$が放出される過程である.
  $E_{\r{in}}$と$E_{\r{out}}$の間には,
  \begin{align}
    E_{\r{out}} = E_{\r{in}}\cosh r - E_{\r{in}}^*\sinh r \label{spdc-classical}
  \end{align}
  なる関係がある.
  ただし,$r$は,
  \begin{align}
    r \coloneqq \frac{z}{2c}\cdot\frac{\omega}{n}
  \end{align}
  と定義され,$z$は非線型結晶の長さ,$c$は光速,$n$は非線型結晶の屈折率である.
  また,$\cosh r$と$\sinh r$は,
  \begin{align}
    \cosh r &\coloneqq \frac{\e^r + \e^{-r}}{2} \\ 
    \sinh r &\coloneqq \frac{\e^r - \e^{-r}}{2} \\ 
  \end{align}
  と定義されていて,
  \begin{align}
    \abs{\cosh r}^2 - \abs{-\sinh r}^2 &= \abs{\frac{\e^r + \e^{-r}}{2}}^2 - \abs{-\frac{\e^r - \e^{-r}}{2}}^2 \\ 
    &= \frac{\e^{2r} + 2 + \e^{-2r}}{4} - \frac{\e^{2r} - 2 + \e^{-2r}}{4} \\ 
    &= 1 \label{cosh-sinh-relation}
  \end{align}
  なる関係がある.
  さて,\refe{spdc-classical}を量子化すると,$\hat{a}$が電場の振幅に対応するのだから,
  出力電場の振幅に対応する物理量を$\hat{b}$とすれば,
  \begin{align}
    \hat{b} = \hat{a}\cosh r - \hat{a}^{\dag}\sinh r
  \end{align}
  となる.
  \subsection{縮退パラメトリック過程のハミルトニアンによる消滅演算子の時間発展}
    Heisenber描像において$\hat{a}$を$\hat{b}$に変換するものが,縮退パラメトリック変換である.
    縮退パラメトリック変換のハミルトニアン$\hat{H}_{\r{para}}$を,
    \begin{align}
      \hat{H}_{\r{para}} \coloneqq \i\frac{\hbar}{t}\frac{r}{2}\qty(\hat{a}^2 - \qty(\hat{a}^{\dag})^2)
    \end{align}
    と定義する.
    $\qty(\hat{a}^2 - \qty(\hat{a}^{\dag})^2)$と$\hat{a}$や,$\qty(\hat{a}^2 - \qty(\hat{a}^{\dag})^2)$と$\hat{a}^{\dag}$との交換関係は,
    \begin{align}
      \qty[\qty(\hat{a}^2 - \qty(\hat{a}^{\dag})^2), \hat{a}] &= \qty(\hat{a}^2 - \qty(\hat{a}^{\dag})^2)\hat{a} - \hat{a}\qty(\hat{a}^2 - \qty(\hat{a}^{\dag})^2) \\ 
      &= \hat{a}\qty(\hat{a}^{\dag})^2 - \qty(\hat{a}^{\dag})^2\hat{a} \\ 
      &= \qty(\hat{a}^{\dag}\hat{a} + 1)\hat{a}^{\dag} - \qty(\hat{a}^{\dag})^2\hat{a} \\ 
      &= \hat{a}^{\dag}\hat{a}\hat{a}^{\dag} + \hat{a}^{\dag} - \qty(\hat{a}^{\dag})^2\hat{a} \\ 
      &= \hat{a}^{\dag}\qty(\hat{a}^{\dag}\hat{a} + 1) + \hat{a}^{\dag} - \qty(\hat{a}^{\dag})^2\hat{a} \\ 
      &= 2\hat{a}^{\dag} \\ 
      \qty[\qty(\hat{a}^2 - \qty(\hat{a}^{\dag})^2), \hat{a}^{\dag}] &= \qty(\hat{a}^2 - \qty(\hat{a}^{\dag})^2)\hat{a}^{\dag} - \hat{a}^{\dag}\qty(\hat{a}^2 - \qty(\hat{a}^{\dag})^2) \\ 
      &= \hat{a}^2\hat{a}^{\dag} - \hat{a}^{\dag}\hat{a}^2 \\ 
      &= \hat{a}\qty(\hat{a}^{\dag}\hat{a} + 1) - \hat{a}^{\dag}\hat{a}^2 \\ 
      &= \hat{a}\hat{a}^{\dag}\hat{a} + \hat{a} - \hat{a}^{\dag}\hat{a}^2 \\ 
      &= \qty(\hat{a}^{\dag}\hat{a} + 1)\hat{a} + \hat{a} - \hat{a}^{\dag}\hat{a}^2 \\ 
      &= 2\hat{a}
    \end{align}
    となることを用いると,$\hat{a}$の時間発展は1つ目のBaker-Campbell-Hausdorffの公式を用いて,この系における消滅演算子$\hat{a}$の時間発展は,
    \begin{align}
      \exp\qty(\i\frac{\hat{H}_{\r{para}}}{\hbar}t)\hat{a}\exp\qty(-\i\frac{\hat{H}_{\r{para}}}{\hbar}t) &= \exp\qty(-\frac{r}{2}\qty(\hat{a}^2 - \qty(\hat{a}^{\dag})^2))\hat{a}\exp\qty(\frac{r}{2}\qty(\hat{a}^2 - \qty(\hat{a}^{\dag})^2)) \\ 
      &= \hat{a} + \qty[-\frac{r}{2}\qty(\hat{a}^2 - \qty(\hat{a}^{\dag})^2), \hat{a}] + \frac{1}{2!}\qty[-\frac{r}{2}\qty(\hat{a}^2 - \qty(\hat{a}^{\dag})^2), \qty[-\frac{r}{2}\qty(\hat{a}^2 - \qty(\hat{a}^{\dag})^2), \hat{a}]] + \cdots \\ 
      &= \hat{a} + \frac{1}{1!}\qty(-\frac{r}{2})^1\cdot 2^1\hat{a}^{\dag} + \frac{1}{2!}\qty(-\frac{r}{2})^2\cdot 2^2\hat{a} + \frac{1}{3!}\qty(-\frac{r}{2})^3\cdot 2^3\hat{a}^{\dag} + \frac{1}{4!}\qty(-\frac{r}{2})^4\cdot 2^4\hat{a} + \cdots \\ 
      &= \hat{a}\qty(1 + \frac{r^2}{2!} + \frac{r^4}{4!} + \cdots) - \hat{a}^{\dag}\qty(\frac{r^1}{1!} + \frac{r^3}{3!} + \frac{r^5}{5!} + \cdots) \\ 
      &= \hat{a}\cosh r - \hat{a}^{\dag}\sinh r
    \end{align}
    となる.
    \refe{cosh-sinh-relation}で示した関係式より,縮退パラメトリック変換はBogoliubov変換である.
    計算の途中で$\cosh r$と$\sinh r$が,
    \begin{align}
      \cosh r &= \frac{\e^r + \e^{-r}}{2} \\ 
      &= \frac{1}{2}\qty{\qty(1 + \frac{r}{1!} + \frac{r^2}{2!} + \cdots) + \qty(1 + \frac{\qty(-r)}{1!} + \frac{\qty(-r)^2}{2!} + \cdots)} \\ 
      &= 1 + \frac{r^2}{2!} + \frac{r^4}{4!} + \cdots \\ 
      \sinh r &= \frac{\e^r - \e^{-r}}{2} \\ 
      &= \frac{1}{2}\qty{\qty(1 + \frac{r}{1!} + \frac{r^2}{2!} + \cdots) - \qty(1 + \frac{\qty(-r)}{1!} + \frac{\qty(-r)^2}{2!} + \cdots)} \\ 
      &= \frac{r^1}{1!} + \frac{r^3}{3!} + \frac{r^5}{5!} + \cdots \\
    \end{align}
    となることを用いた.
    $\hat{a}$を$\hat{b}$に変換するユニタリをスクイーズ演算子$\hat{S}(r)$といい,
    \begin{align}
      \hat{S}(r) \coloneqq \exp\qty{\frac{r}{2}\qty(\hat{a}^2 - \qty(\hat{a}^{\dag})^2)}
    \end{align}
    と定義する.
  \subsection{真空状態とスクイーズされた真空場の関係}
    個数状態は$\hat{a}^{\dag}\hat{a}$の固有状態であり,$\hat{a}$と$\hat{a}^{\dag}$に課している条件は,$\qty[\hat{a}, \hat{a}^{\dag}] = 1$のみである.
    さて,同様に$\hat{b}$と$\hat{b}^{\dag}$の交換関係を調べると,
    \begin{align}
      \qty[\hat{b}, \hat{b}^{\dag}] &= \qty(\mu\hat{a} + \nu\hat{a}^{\dag})\qty(\mu^*\hat{a}^{\dag} + \nu^*\hat{a}) - \qty(\mu^*\hat{a}^{\dag} + \nu^*\hat{a})\qty(\mu\hat{a} + \nu\hat{a}^{\dag}) \\ 
      &= \qty(\abs{mu}^2 - \abs{\nu}^2)\qty(\hat{a}\hat{a}^{\dag} - \hat{a}^{\dag}\hat{a}) \\ 
      &= 1
    \end{align}
    より,$\hat{a}^{\dag}\hat{a}$の固有状態$\ket{n}$のように$\hat{b}^{\dag}\hat{b}$の固有状態$\ket{m_{\r{g}}}$が存在する.
    また,$\hat{a}$の固有状態$\ket{\alpha}$が存在して$\ket{n}$で展開できるように,$\hat{b}$の固有状態$\ket{\beta}_{\r{g}}$が存在して$\ket{m_{\r{g}}}$で展開できる.
    どちらも展開係数の2乗はPoisson分布になるはずであるから,
    \begin{align}
      \ket{\beta}_{\r{g}} = \sum_{m_{\r{g}}}\frac{\beta^{m_{\r{g}}}}{\sqrt{m_{\r{g}}}}\exp\qty(-\frac{\abs{\beta^2}}{2})\ket{m_{\r{g}}}
    \end{align}
    と展開できる.
    明らかに,
    \begin{align}
      \ket{\beta = 0}_{\r{g}} = \ket{m_{\r{g}} = 0}\label{bg-mg-relation}
    \end{align}
    である.
    \par
    Heisenber描像において$\hat{a}$を$\hat{b}$はユニタリ演算子$\hat{S}(r)$によって変換されたのであった.
    Schr\"odinger描像では,状態ベクトルに$\hat{S}(r)$を作用させて時間発展を考えることができる.
    個数演算子$\hat{a}^{\dag}\hat{a}$の固有状態の1つである$\ket{0}$を考える.
    Heisenber描像では,個数演算子は,
    \begin{align}
      \hat{S}(r)^{\dag}\hat{a}^{\dag}\hat{a}\hat{S}(r) = \hat{S}(r)^{\dag}\hat{a}^{\dag}\hat{S}(r)\hat{S}(r)^{\dag}\hat{a}\hat{S}(r) = \hat{b}^{\dag}\hat{b}
    \end{align}
    とユニタリ変換されている.
    Schr\"odinger描像では,$\ket{0}$が時間発展して$\ket{m_{\r{g}} = 0}$になるとすれば\refe{bg-mg-relation}より,
    \begin{align}
      \ket{\beta = 0} = \hat{S}(r)\ket{0}
    \end{align}
    となる.
  \subsection{位置と運動量の平均値・分散}
    % 位置と運動量
\end{document}