\documentclass{report}
\input{../../head.tex}
\begin{document}
  結果のみ示す.
  その他のことについては,別途\href{https://github.com/YutoMSD/physics_notes/blob/main/qft/main.pdf}{ノート}を参照すること.
  なお,\today 現在,ノートは編集中である.
  Schr\"odingr描像では,電場と磁場は,
  \begin{align}
    \hat{E}(\bm{r}) &= \i\qty(\frac{1}{2\pi})^{3/2}\int\dd[3]{k}\sum_{\sigma = 1}^{2}\sqrt{\frac{\hbar\omega_{\bm{k}}}{2\epsilon_0}}\bm{e}_{\bm{k}\sigma}\qty(\hat{a}_{\bm{k}\sigma}\e^{\i\bm{k}\cdot\bm{r}} - \hat{a}_{\bm{k}\sigma}^{\dag}\e^{-\i\bm{k}\cdot\bm{r}}) \\ 
    \hat{B}(\bm{r}) &= \i\qty(\frac{1}{2\pi})^{3/2}\int\dd[3]{k}\sum_{\sigma = 1}^{2}\sqrt{\frac{\hbar}{2\epsilon_0\omega_{\bm{k}}}}\bm{k}\times\bm{e}_{\bm{k}\sigma}\qty(\hat{a}_{\bm{k}\sigma}\e^{\i\bm{k}\cdot\bm{r}} - \hat{a}_{\bm{k}\sigma}^{\dag}\e^{-\i\bm{k}\cdot\bm{r}}) 
  \end{align}
  と量子化される.
  また,Heisenberg描像では,
  \begin{align}
    \hat{E}(\bm{r}, t) &= \i\qty(\frac{1}{2\pi})^{3/2}\int\dd[3]{k}\sum_{\sigma = 1}^{2}\sqrt{\frac{\hbar\omega_{\bm{k}}}{2\epsilon_0}}\bm{e}_{\bm{k}\sigma}\qty[\hat{a}_{\bm{k}\sigma}\exp\qty{\i\qty(\bm{k}\cdot\bm{r} - \omega_{\bm{k}}t)} - \hat{a}_{\bm{k}\sigma}^{\dag}\exp\qty{-\i\qty(\bm{k}\cdot\bm{r} - \omega_{\bm{k}}t)}] \\ 
    \hat{B}(\bm{r}, t) &= \i\qty(\frac{1}{2\pi})^{3/2}\int\dd[3]{k}\sum_{\sigma = 1}^{2}\sqrt{\frac{\hbar}{2\epsilon_0\omega_{\bm{k}}}}\bm{k}\times\bm{e}_{\bm{k}\sigma}\qty[\hat{a}_{\bm{k}\sigma}\exp\qty{\i\qty(\bm{k}\cdot\bm{r} - \omega_{\bm{k}}t)} - \hat{a}_{\bm{k}\sigma}^{\dag}\exp\qty{-\i\qty(\bm{k}\cdot\bm{r} - \omega_{\bm{k}}t)}]
  \end{align}
  と書ける.
\end{document}