\documentclass{report}
\input{../../head.tex}
\begin{document}
  本節ではコヒーレント状態について議論する.
  コヒーレント状態$\ket{\alpha}$は,
  \begin{align}
    \hat{a}\ket{\alpha} = \alpha\ket{\alpha}\label{coherent-def}
  \end{align}
  なる状態である.
  また,$\alpha = \abs{\alpha}\e^{\i\theta}$となるように$\theta$を定義しておく.
  \subsection{物理量の平均値・分散}
    具体的な$\ket{\alpha}$の形を知らなくても,いくつかの物理量の平均値と分散については調べることができる.
    まず,電場の期待値を調べる.
    電場演算子$\hat{\bm{E}}(\bm{r}, t)$を自然単位系を用いて書くと,
    \begin{align}
      \frac{\i}{2}\bm{e}\qty(\hat{a}\exp\qty{\i(\bm{k}\cdot\bm{r} - \omega t)} - \hat{a}^{\dag}\exp\qty{-\i(\bm{k}\cdot\bm{r} - \omega t)})
    \end{align}
    であるから,
    \begin{align}
      \ev{\bm{E}(\bm{r}, t)} &= \mel**{\alpha}{\bm{E}(\bm{r}, t)}{\alpha} \\ 
      &= \frac{\i}{2}\bm{e}\qty(\mel**{\alpha}{\hat{a}}{\alpha}\exp\qty{\i(\bm{k}\cdot\bm{r} - \omega t)} - \mel**{\alpha}{\hat{a}^{\dag}}{\alpha}\exp\qty{-\i(\bm{k}\cdot\bm{r} - \omega t)}) \\ 
      &= -\frac{1}{2\i}\bm{e}\qty(\alpha\exp\qty{\i(\bm{k}\cdot\bm{r} - \omega t)} - \alpha^*\exp\qty{-\i(\bm{k}\cdot\bm{r} - \omega t)}) \\ 
      &= -\frac{1}{2\i}\bm{e}\qty(\abs{\alpha}\e^{\i\theta}\exp\qty{\i(\bm{k}\cdot\bm{r} - \omega t)} - \abs{\alpha}\e^{-\i\theta}\exp\qty{-\i(\bm{k}\cdot\bm{r} - \omega t)}) \\ 
      &= -\abs{\alpha}\bm{e}\sin\qty(\bm{k}\cdot\bm{r} - \omega t + \theta)
    \end{align}
    である.
    \par
    次に,位置と運動量の平均値と分散について議論する.
    位置演算子と運動量演算子は生成演算子と消滅演算子を用いて,
    \begin{align}
      \hat{x} &\coloneqq \frac{1}{2}\qty(\hat{a} + \hat{a}^{\dag}) \\ 
      \hat{p} &\coloneqq \frac{1}{2\i}\qty(\hat{a} - \hat{a}^{\dag})
    \end{align}
    と書けるから,
    \begin{align}
      \ev{x} &= \mel**{\alpha}{\frac{1}{2}\qty(\hat{a} + \hat{a}^{\dag})}{\alpha} \\ 
      &= \frac{1}{2}\qty(\alpha + \alpha^*) \\ 
      \ev{x^2} &= \mel**{\alpha}{\frac{1}{4}\qty(\hat{a} + \hat{a}^{\dag})^2}{\alpha} \\ 
      &= \frac{1}{4}\mel**{\alpha}{\hat{a}^2}{\alpha} + \frac{1}{4}\mel**{\alpha}{\hat{a}\hat{a}^{\dag}}{\alpha} + \frac{1}{4}\mel**{\alpha}{\hat{a}^{\dag}\hat{a}}{\alpha} + \frac{1}{4}\mel**{\alpha}{\qty(\hat{a}^{\dag})^2}{\alpha} \\ 
      &= \frac{1}{4}\alpha^2 + \frac{1}{4}\mel**{\alpha}{\hat{a}^{\dag}\hat{a} + 1}{\alpha} + \frac{1}{4}\abs{\alpha}^2 + \frac{1}{4}\qty(\alpha^*)^2 \\ 
      &= \frac{1}{4}\qty(\alpha + \alpha^*)^2 + \frac{1}{4} \\ 
      \ev{p} &= \mel**{\alpha}{\frac{1}{2\i}\qty(\hat{a} - \hat{a}^{\dag})}{\alpha} \\ 
      &= \frac{1}{2\i}\qty(\alpha - \alpha^*) \\ 
      \ev{p^2} &= \mel**{\alpha}{-\frac{1}{4}\qty(\hat{a} - \hat{a}^{\dag})^2}{\alpha} \\ 
      &= -\frac{1}{4}\mel**{\alpha}{\hat{a}^2}{\alpha} + \frac{1}{4}\mel**{\alpha}{\hat{a}\hat{a}^{\dag}}{\alpha} + \frac{1}{4}\mel**{\alpha}{\hat{a}^{\dag}\hat{a}}{\alpha} - \frac{1}{4}\mel**{\alpha}{\qty(\hat{a}^{\dag})^2}{\alpha} \\ 
      &= -\frac{1}{4}\alpha^2 + \frac{1}{4}\mel**{\alpha}{\hat{a}^{\dag}\hat{a} + 1}{\alpha} + \frac{1}{4}\abs{\alpha}^2 - \frac{1}{4}\qty(\abs{\alpha}^{*})^2 \\ 
      &= -\frac{1}{4}\qty(\alpha^2 + \alpha^*)^2 + \frac{1}{4}
    \end{align}
    より,
    \begin{align}
      \Delta x_{\r{coh}} &\coloneqq \sqrt{\ev{x^2} - \ev{x}^2} = \frac{1}{4} \\ 
      \Delta p_{\r{coh}} &\coloneqq \sqrt{\ev{p^2} - \ev{p}^2} = \frac{1}{4}
    \end{align}
    となる.
  \subsection{個数状態での展開}
    コヒーレント状態$\ket{\alpha}$を,Hermite演算子である$\hat{n}$の固有状態である個数状態$\ket{n}$で展開することを考える.
    \begin{align}
      \ket{\alpha} = \sum_{n = 0}^{\infty}w_n\ket{n}\label{coherent-expantion}
    \end{align}
    である.
    \refe{coherent-def}に\refe{coherent-expantion}を代入して,\refe{annihilation}の関係式を用いると,
    \begin{align}
      \hat{a}\ket{\alpha} = \alpha\ket{\alpha} \\ 
      \iff \hat{a}\sum_{n = 0}^{\infty}w_n\ket{n} &= \alpha\sum_{n = 0}^{\infty}w_n\ket{n} \\ 
      \iff \sum_{n = 1}^{\infty}\sqrt{n}w_n\ket{n - 1} &= \sum_{n = 0}^{\infty}\alpha w_n\ket{n} \\ 
      \iff \sum_{n = 0}^{\infty}\sqrt{n + 1}w_{n + 1}\ket{n} &= \sum_{n = 0}^{\infty}\alpha w_n\ket{n}
    \end{align}
    であるから,
    \begin{align}
      \sqrt{n + 1}w_{n + 1} &= \alpha w_n \\ 
      \iff w_n &= \frac{\alpha^n}{\sqrt{n!}}C
    \end{align}
    である.
    $C \coloneqq w_0$と定義した.
    $\ket{\alpha}$の規格化条件より,
    \begin{align}
      1 = \braket{\alpha}{\alpha} &= \qty(\sum_{n}w_n\ket{n})^{\dag}\qty(\sum_{m}w_m\ket{m}) \\ 
      &= \qty(\sum_{n}\frac{\qty(\alpha^*)^n}{\sqrt{n!}}C^*\bra{n})\qty(\sum_{m}\frac{\alpha^m}{\sqrt{m!}}C\ket{m}) \\ 
      &= \abs{C}^2\sum_{n, m}\frac{\qty(\alpha^*)^n\alpha^m}{\sqrt{n!}\sqrt{m!}}\braket{n}{m} \\ 
      &= \abs{C}^2\sum_{n}\frac{\qty(\abs{\alpha}^2)^n}{n!} \\ 
      &= \abs{C}^2\e^{\abs{\alpha}^2}
    \end{align}
    であるから,
    \begin{align}
      C = \exp\qty(\frac{\abs{\alpha}^2}{2})
    \end{align}
    とすればよい.
    よって,コヒーレント状態は,
    \begin{align}
      \ket{\alpha} = \sum_n \frac{\alpha^n}{\sqrt{n!}}\exp\qty(\frac{\abs{\alpha}^2}{2})\ket{n}\label{coherent-number-state-expantion}
    \end{align}
    と書ける.
    \refe{coherent-number-state-expantion}より,コヒーレント状態とは,個数状態を$\ket{n}$をPoisson分布に従って重ね合わせたものだと分かる.
  \subsection{調和振動子ハミルトニアンでの時間発展}
    まず,コヒーレント状態$\ket{\alpha}$が調和振動子ハミルトニアン$\hat{H}$があるときにどのように時間発展発展するか調べる.
    系のハミルトニアンは,
    \begin{align}
      \hat{H} = \hbar\omega\qty(\hat{n} + \frac{1}{2})
    \end{align}
    と書けるから,
    \begin{align}
      \ket{\alpha(t)} &= \exp\qty(-\i\frac{\hat{H}}{\hbar}t)\ket{\alpha} \\ 
      &= \exp\qty(-\i\frac{\omega}{2}t)\e^{-\i\omega t \hat{n}}\ket{\alpha} \\ 
      &= \exp\qty(-\i\frac{\omega}{2}t)\sum_{m}\frac{(-\i\omega t)^m}{m!}\hat{n}^m \sum_{n}\frac{\alpha^n}{\sqrt{n!}}\exp\qty(\frac{\abs{\alpha}^2}{2})\ket{n} \\ 
      &= \exp\qty(-\i\frac{\omega}{2}t)\exp\qty(\frac{\abs{\alpha}^2}{2}) \sum_{n}\frac{\alpha^n}{\sqrt{n!}}\sum_{m}\frac{(-\i\omega t)^m}{m!}\hat{n}^m\ket{n} \\ 
      &= \exp\qty(-\i\frac{\omega}{2}t)\exp\qty(\frac{\abs{\alpha}^2}{2}) \sum_{n}\frac{\alpha^n}{\sqrt{n!}}\sum_{m}\frac{(-\i n\omega t)^m}{m!} \ket{n} \\ 
      &= \exp\qty(-\i\frac{\omega}{2}t)\exp\qty(\frac{\abs{\alpha}^2}{2}) \sum_{n}\frac{\alpha^n}{\sqrt{n!}}\e^{-\i n\omega t}\ket{n} \\ 
      &= \exp\qty(-\i\frac{\omega}{2}t) \sum_{n}\frac{\qty(\alpha\e^{-\i\omega t})^n}{\sqrt{n!}}\exp\qty(\frac{\abs{\alpha}^2}{2})\ket{n} \\ 
      &= \exp\qty(-\i\frac{\omega}{2}t) \sum_{n}\frac{\qty(\alpha\e^{-\i\omega t})^n}{\sqrt{n!}}\exp\qty(\frac{\abs{\alpha\e^{-\i\omega t}}^2}{2})\ket{n} \\ 
      &= \exp\qty(-\i\frac{\omega}{2}t)\ket{\alpha\e^{-\i\omega t}}
    \end{align}
    グローバル位相は無視してよいので,
    \begin{align}
      \ket{\alpha(t)} = \ket{\alpha\e^{-\i\omega t}}
    \end{align}
    と分かる.
    \par
    次に個数状態の時間発展を調べる.
    \begin{align}
      \ket{n(t)} &= \exp\qty(-\i\frac{\hat{H}}{\hbar}t)\ket{n} \\ 
      &= \exp\qty(-\i\frac{\omega}{2}t)\e^{-\i\omega t \hat{n}}\ket{n} \\ 
      &= \exp\qty(-\i\frac{\omega}{2}t)\sum_{m}\frac{(-\i\omega t)^m}{m!}\hat{n}^m \ket{n} \\ 
      &= \exp\qty(-\i\frac{\omega}{2}t)\sum_{m}\frac{(-\i n\omega t)^m}{m!} \ket{n} \\ 
      &= \exp\qty(-\i\frac{\omega}{2}t)\e^{-\i n\omega t} \ket{n}
    \end{align}
    となり,周波数が2倍になったように見える\footnote{らしい}.
  \subsection{レーザのハミルトニアン}
    真空場$\ket{0}$が$\ket{\alpha}$に時間変化するものがレーザである.
    レーザのハミルトニアンは,
    \begin{align}
      \hat{H}_{\r{laser}} \propto \i\qty(\alpha\hat{a}^{\dag} - \alpha^*\hat{a})
    \end{align}
    とかける\footnote{らしい}.
    この系において,真空状態$\ket{0}$の時間発展は,
    \begin{align}
      \exp\qty(-\i\frac{\hat{H}}{\hbar}t)\ket{0} &= \exp\qty(-\i\frac{\i\qty(\alpha\hat{a}^{\dag} - \alpha^*\hat{a})}{\hbar}t) \\ 
      &= \exp(\frac{\qty(\alpha\hat{a}^{\dag} - \alpha^*\hat{a})}{\hbar}t) \\ 
      &= 
    \end{align}
\end{document}