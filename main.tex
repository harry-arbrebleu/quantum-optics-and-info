\documentclass{report}
\input{head.tex}
\begin{document}
  \maketitle
  \tableofcontents
  \chapter{量子光学}
    量子力学は物理量$\hat{A}$の期待値$\ev{A} = \mel**{\psi}{\hat{A}}{\psi}$を検討する学問である.
    本章は以下のような構成である.
    まず,Sch\"odinger描像とHeisenberg描像や電場の量子化について説明する.
    また,量子もつれにおいて重要なビームスプリッタを紹介する.
    次に,量子光学において重要なコヒーレント状態とスクイーズド状態について議論する.
    続いて密度演算子を定義したあと,今まで議論した状態を具体的に測定するための方法として,バランス型ホモダイン測定を紹介する.
    \section{Sch\"odinger描像とHeisenberg描像}
      WIP
    \section{調和振動子}
      \documentclass{report}
\input{../../head.tex}
\begin{document}
  本節では,1次元調和振動子モデルでハミルトニアンが書けるときの波動函数の表示を求める.
  波動函数とは,Schr\"odinger方程式,
  \begin{align}
    \hat{H}\ket{\psi} = E\ket{\psi}\label{schrodinger-equation}
  \end{align}
  を満たす$\ket{\psi}$について,
  \begin{align}
    \psi(x) \coloneqq \braket{x}{\psi}
  \end{align}
  となるように(一般化)座標$x$へ射影したものである.
  \refe{schrodinger-equation}に対して$\bra{x}$を左から書ければ,
  \begin{align}
    \mel**{x}{\hat{H}}{\psi} = E\psi(x)\label{bra-x-from-left}
  \end{align}
  となるのだから,左辺を計算して$\psi(x)$に演算子がかかる形に変形すれば,波動函数を求めることができる.
  本ノートにおいて,$\hat{\cdot}$を演算子として,その固有値を$\cdot$,固有ベクトル(固有函数)を$\ket{\cdot}$と書く.
  \begin{align}
    \hat{x}\ket{x} &= x\ket{x} \\ 
    \hat{p}\ket{x} &= p\ket{x}
  \end{align}
  である.また,$\hat{x}$や$\hat{p}$は物理量であり,Hermite演算子だからその固有ベクトルは,
  \begin{align}
    \braket{x'}{x} &= \delta\qty(x' - x) \\ 
    \braket{p'}{p} &= \delta\qty(p' - p)
  \end{align}
  と規格化してあり,
  \begin{align}
    \int\dd{x}\ketbra{x}{x} &= \hat{1}
    \int\dd{p}\ketbra{p}{p} &= \hat{1}
  \end{align}
  が成立する.なお,特に断らない限り積分範囲は$-\infty$から$\infty$である.
  \subsection{ハミルトニアン}
    古典的な1次元調和振動子のハミルトニアン$H$は,
    \begin{align}
      H = \frac{1}{2}m\omega^2x^2 + \frac{1}{2m}p^2
    \end{align}
    である.ただし,質量を$m$,固有角周波数を$\omega$,座標を$x$,運動量を$p$とした.
    $x$と$p$は正準共役な変数の組であるから,$x\to \hat{x}$,$p \to \hat{p}$として,
    \begin{align}
      \hat{H} &= \frac{1}{2}m\omega^2\hat{x}^2 + \frac{1}{2m}\hat{p}^2 \label{harmonic-oscillator-hamiltonian}\\ 
      &= \hbar\omega\qty(\frac{m\omega}{2\hbar}\hat{x}^2 + \frac{1}{2m\hbar\omega}\hat{p}^2) \\ 
      &= \hbar\omega\qty[\qty(\sqrt{\frac{m\omega}{2\hbar}}\hat{x} - \i\sqrt{\frac{1}{2m\hbar\omega}}\hat{p})\qty(\sqrt{\frac{m\omega}{2\hbar}}\hat{x} + \i\sqrt{\frac{1}{2m\hbar\omega}}\hat{p}) - \i\sqrt{\frac{m\omega}{2\hbar}}\sqrt{\frac{1}{2m\hbar\omega}}\qty[\hat{x}, \hat{p}]] \\ 
      &= \hbar\omega\qty(\hat{a}^{\dag}\hat{a} + \frac{1}{2})
    \end{align}
    となる.ただし,
    \begin{align}
      \hat{a}^{\dag} &\coloneqq \qty(\sqrt{\frac{m\omega}{2\hbar}}\hat{x} - \i\sqrt{\frac{1}{2m\hbar\omega}}\hat{p}) \\ 
      \hat{a} &\coloneqq \qty(\sqrt{\frac{m\omega}{2\hbar}}\hat{x} + \i\sqrt{\frac{1}{2m\hbar\omega}}\hat{p})
    \end{align}
    と定義した.
  \subsection{Hermite多項式}
    以降の議論で用いるために,特殊函数の1つであるHermite多項式を紹介しておこう.
    Hermite多項式はStrum-Liouville演算子のうちの1つの演算子の固有函数であり,実数全体で定義された実数函数$H_n(s)$に対して,
    \begin{align}
      \qty(\dv[2]{s} - 2s\dv{s} + 2n)H_n(s) = 0
    \end{align}
    なる$H_n(s)$である.
    なお,$H_n(s)$は適当な回数だけ微分可能であるとする.
    また,$H_n(s)$が張る空間$V$の内積は,$f, g\in V$として,
    \begin{align}
      \inner{f}{g} \coloneqq \int_{-\infty}^{\infty}\dd{s}f(s)g(s)\e^{-s^2}
    \end{align}
    である\footnote{
      Strum-Liouville演算子の形,
      \begin{align*}
        \frac{1}{\rho(x)}\qty[\dv{x}\qty{p(x)\dv{x}} + q(x)]
      \end{align*}
      と,Strum-Liouville演算子の固有函数が張る空間の内積が,
      \begin{align*}
        \int_{a}^{b} f^*(x)g(x)\rho(x)\dd{x}
      \end{align*}
      と書けることを思い出せば,内積に$\e^{-s^2}$なる重み函数が入ることは自然なことである.
    }.
    Hermite多項式は,適切に境界条件が設定された(Hermite性のある)Strum-Liouville演算子の固有函数であり,
    そのような演算子の固有函数は直交基底となり,完全系を成すことが知られていて,実際,
    \begin{align}
      \int_{-\infty}^{\infty}H_m(s)H_n(s)\e^{-s^2}\dd{s} = \sqrt{\pi}2^nn!\delta_n^m
    \end{align}
    のように直交する.
  \subsection{波動函数を用いたSchr\"odinger方程式}
    以下では,波動函数を用いたchr\"odinger方程式である,
    \begin{align}
      \qty(-\frac{\hbar^2}{2m}\dv[2]{x} + \frac{1}{2}m\omega^2x^2)\psi(x) = E\psi(x)
    \end{align}
    を得る.
    \par
    \refe{bra-x-from-left}に\refe{harmonic-oscillator-hamiltonian}で示した$\hat{H}$の表式を代入して,
    \begin{align}
      \frac{1}{2}m\omega^2\mel**{x}{\hat{x}^2}{\psi} + \frac{1}{2m}\mel**{x}{\hat{p}^2}{\psi} &= E\psi(x) \\ 
      \frac{1}{2}m\omega^2x^2\psi(x) + \frac{1}{2m}\mel**{x}{\hat{p}^2}{\psi} &= E\psi(x) \label{input-to-schrodinger}
    \end{align}
    となる.
    $\mel**{x}{\hat{p}^2}{\psi}$は以下のレシピで計算できる.
    \begin{enumerate}
      \item $f(x)\dv{x}\delta(x) = -\dv{x}f(x)\delta(x)$
      \item $\mel**{x}{\hat{p}}{\psi} = -\i\hbar\dv{x}\psi(x)$
      \item $\braket{x}{p} = \frac{1}{\sqrt{2\pi p}}\exp\qty(\i\frac{xp}{\hbar})$
      \item $\mel**{x}{\hat{p}^2}{\psi}$の計算
    \end{enumerate}
    $\delta(x)$はデルタ函数であり,積分して初めて意味を持つ函数である.
    \begin{enumerate}
      \item $f(x)\dv{x}\delta(x) = -\dv{x}f(x)\delta(x)$
        左辺を積分して右辺になればよい.ただし,$f(x)$は,
        \begin{align}
          \lim_{\abs{x} \to \infty}f(x) = 0
        \end{align}
        であるとする.
        実際に,
        \begin{align}
          \int_{-\infty}^{\infty}\dd{x}f(x)\dv{x}\delta(x) &= \qty[f(x)\delta(x)]_{-\infty}{\infty} - \int_{-\infty}^{\infty}\dd{x}\dv{x}f(x)\delta(x) \\ 
          &= - \int_{-\infty}^{\infty}\dd{x}\dv{x}f(x)\delta(x)
        \end{align}
        であるから,
        \begin{align}
          f(x)\dv{x}\delta(x) = -\dv{x}f(x)\delta(x)\label{delta-function-diff}
        \end{align}
        である.
      \item $\mel**{x}{\hat{p}}{\psi} = -\i\hbar\dv{x}\psi(x)$
        $\mel**{x}{\qty[\hat{x}, \hat{p}]}{x'}$を2種類の方法で計算する.
        まず,愚直に計算すると,
        \begin{align}
          \mel**{x}{\qty[\hat{x}, \hat{p}]}{x'} &= \mel**{x}{\hat{x}\hat{p} - \hat{p}\hat{x}}{x'} \\ 
          &= x\mel**{x}{\hat{p}}{x'} - x'\mel**{x}{\hat{p}}{x'} \\ 
          &= \qty(x - x')\mel**{x}{\hat{p}}{x'}\label{xxpxprime-1}
        \end{align}
        である.一方,$\qty[\hat{x}, \hat{p}] = \i\hbar$を用いれば,
        \begin{align}
          \mel**{x}{\qty[\hat{x}, \hat{p}]}{x'} &= \i\hbar\braket{x}{x'} \\ 
          &= \i\hbar\delta(x - x')\label{xxpxprime-2}
        \end{align}
        2つの方法で計算した$\mel**{x}{\qty[\hat{x}, \hat{p}]}{x'}$である\refe{xxpxprime-1}と\refe{xxpxprime-2}を等号で結んで,
        \refe{delta-function-diff}で示したデルタ函数の微分を用いて表現すれば,
        \begin{align}
          \mel**{x}{\hat{p}}{x'} &= \i\hbar\frac{\delta(x - x')}{x - x'} \\ 
          &= -\i\hbar\dv{\qty(x - x')}\delta(x - x') \\ 
          &= \i\hbar\dv{x'}\delta(x - x')\label{xpxrime}
        \end{align}
        となる.
        \par
        さて,$\mel**{x}{\hat{p}}{\psi}$を計算しよう.
        $\ket{x}$の完全性と,\refe{xpxrime}で示した関係を用いれば,
        \begin{align}
          \mel**{x}{\hat{p}}{\psi} &= \mel**{x}{\hat{p}\hat{1}}{\psi} \\ 
          &= \bra{x}\hat{p}\int\dd{x'}\ketbra{x'}{x'}\ket{\psi} \\ 
          &= \int\dd{x'}\mel**{x}{\hat{p}}{x'}\braket{x'}{\psi} \\ 
          &= \i\hbar\int\dd{x'}\qty[\dv{x}\delta(x - x')]\phi(x') \\ 
          &= \i\hbar\qty{[\delta(x - x')]_{-\infty}^{\infty} - \int\dd{x'}\dv{x'}\phi(x')\delta(x - x')} \\ 
          &= -\i\hbar\dv{x}\phi(x)\label{psix-diff}
        \end{align}
        を得る.
      \item $\braket{x}{p} = \frac{1}{\sqrt{2\pi p}}\exp\qty(\i\frac{xp}{\hbar})$
        $\mel**{x}{\hat{p}}{p}$を2種類の方法で計算する.
        まず,愚直に計算すると,
        \begin{align}
          \mel**{x}{\hat{p}}{p} &= p\braket{x}{p} \\ 
          &= pp(x)\label{xpp-1}
        \end{align}
        となる.ただし$p(x)$は$\ket{p}$の$x$への射影である.
        一方,\refe{psix-diff}で示した関係で$\ket{\psi} \to \ket{p}$を用いると,
        \begin{align}
          \mel**{x}{\hat{p}}{\psi} = -\i\hbar\dv{x}p(x)\label{xpp-2}
        \end{align}
        となる.\refe{xpp-1}と\reff{xpp-2}より,
        \begin{align}
          -\i\hbar\dv{x}p(x) &= pp(x) \\ 
          \Rightarrow p(x) &= C\exp\qty(\i\frac{xp}{\hbar})
        \end{align}
        となる.$C$は規格化定数である.
        \par
        さて,$C$を求めるために,$\braket{x}{x'}$を計算すると,
        \begin{align}
          \delta(x - x') &= \braket{x}{x'} \\ 
          &= \bra{x}\int\dd{p}\ketbra{p}{p}\ket{x'} \\ 
          &= \int\dd{p}p(x)p(x') \\ 
          &= \abs{C}^2\int\exp\qty(\i\frac{\i\qty(x - x')p}{\hbar})
        \end{align}
        となる.ところで,デルタ函数のFouirer変換とその逆変換が,
        \begin{align}
          1 &= \int_{-\infty}^{\infty}\dd{t}\delta(t)\e^{-\i\omega t} \\ 
          \delta(t) &= \frac{1}{2\pi}\int\dd{\omega}1\cdot\e^{\i\omega t}\label{delta-fourier-inv}
        \end{align}
        と書けることより,\refe{delta-fourier-inv}において,
        \begin{align}
          \omega &\to \frac{p}{\hbar} \\ 
          t &\to x - x'
        \end{align}
        と変換すれば,
        \begin{align}
          \delta(x - x') = \frac{1}{2\pi\hbar}\int\dd{p}\exp\qty(\i\frac{x - x'}{\hbar}p)
        \end{align}
        となるので,係数を比較して,
        \begin{align}
          \abs{C}^2 &= \frac{1}{2\pi\hbar} \\ 
          \Rightarrow C &= \frac{1}{\sqrt{2\pi\hbar}} 
        \end{align}
        となる.よって,
        \begin{align}
          \braket{x}{p} = \frac{1}{\sqrt{2\pi\hbar}}\exp\qty(\i\frac{xp}{\hbar})\label{braket-xp}
        \end{align}
        となる.
      \item $\mel**{x}{\hat{p}^2}{\psi}$の計算
        さて,いよいよ$\mel**{x}{\hat{p}^2}{\psi}$を計算する道具がそろった.
        \refe{braket-xp}を用いながら計算すると,
        \begin{align}
          \mel**{x}{\hat{p}^2}{\psi} &= \mel**{x}{\hat{p}^2\hat{1}}{\psi} \\
          &= \bra{x}\hat{p}^2\int\dd{p}\ketbra{p}{p}\ket{\psi} \\ 
          &= \int\dd{p}p^2\braket{x}{p}\braket{p}{\psi} \\ 
          &= \int\dd{p}p^2\frac{1}{\sqrt{2\pi\hbar}}\exp\qty(\i\frac{xp}{\hbar})\braket{p}{\psi} \\ 
          &= \int\dd{p}\frac{1}{\sqrt{2\pi\hbar}}\qty{\dv[2]{x}\qty(\frac{\hbar}{\i})^2\braket{x}{p}}\braket{p}{\psi} \\ 
          &= \qty(\frac{\hbar}{\i})^2\int\dd{p}\qty{\dv[2]{x}\braket{x}{p}}\braket{p}{\psi} \\ 
          &= \qty(\frac{\hbar}{\i})^2\dv[2]{x}\bra{x}\qty[\int\dd{p}\ketbra{p}{p}]\ket{\psi} \\ 
          &= \qty(\frac{\hbar}{\i})^2\dv[2]{x}\braket{x}{\psi} \\
          &= \qty(\frac{\hbar}{\i})^2\dv[2]{x}\psi(x)\label{before-schrodinger-oscillator}
        \end{align}
        となる.
    \end{enumerate}
    \refe{before-schrodinger-oscillator}を\refe{input-to-schrodinger}に代入すれば,
    \begin{align}
      \frac{1}{2}m\omega^2x^2\psi(x) + \frac{1}{2m}\mel**{x}{\hat{p}^2}{\psi} &= E\psi(x)\psi(x) + \frac{1}{2m}\mel**{x}{\hat{p}^2}{\psi} &= E\psi(x) \\ 
      \Leftrightarrow \frac{1}{2}m\omega^2x^2\psi(x) + \frac{1}{2m}\qty(\frac{\hbar}{\i})^2\dv[2]{x}\psi(x) = E\psi(x) \\ 
      \Leftrightarrow \qty(-\frac{\hbar}{2m}\dv[2]{x} + \frac{1}{2}m\omega^2x^2)\psi(x) = E\psi(x)
    \end{align}
    となる.
  \subsection{Schrodingerの解法}
\end{document}
    \section{電磁場の量子化}
      WIP
      \documentclass{report}
\input{../../head.tex}
\begin{document}
  結果のみ示す.
  その他のことについては,別途\href{https://github.com/YutoMSD/physics_notes/blob/main/qft/main.pdf}{ノート}を参照すること.
  なお,\today 現在,ノートは編集中である.
  Schr\"odingr描像では,電場と磁場は,
  \begin{align}
    \hat{E}(\bm{r}) &= \i\qty(\frac{1}{2\pi})^{3/2}\int\dd[3]{k}\sum_{\sigma = 1}^{2}\sqrt{\frac{\hbar\omega_{\bm{k}}}{2\epsilon_0}}\bm{e}_{\bm{k}\sigma}\qty(\hat{a}_{\bm{k}\sigma}\e^{\i\bm{k}\cdot\bm{r}} - \hat{a}_{\bm{k}\sigma}^{\dag}\e^{-\i\bm{k}\cdot\bm{r}}) \\ 
    \hat{B}(\bm{r}) &= \i\qty(\frac{1}{2\pi})^{3/2}\int\dd[3]{k}\sum_{\sigma = 1}^{2}\sqrt{\frac{\hbar}{2\epsilon_0\omega_{\bm{k}}}}\bm{k}\times\bm{e}_{\bm{k}\sigma}\qty(\hat{a}_{\bm{k}\sigma}\e^{\i\bm{k}\cdot\bm{r}} - \hat{a}_{\bm{k}\sigma}^{\dag}\e^{-\i\bm{k}\cdot\bm{r}}) 
  \end{align}
  と量子化される.
  また,Heisenberg描像では,
  \begin{align}
    \hat{E}(\bm{r}, t) &= \i\qty(\frac{1}{2\pi})^{3/2}\int\dd[3]{k}\sum_{\sigma = 1}^{2}\sqrt{\frac{\hbar\omega_{\bm{k}}}{2\epsilon_0}}\bm{e}_{\bm{k}\sigma}\qty[\hat{a}_{\bm{k}\sigma}\exp\qty{\i\qty(\bm{k}\cdot\bm{r} - \omega_{\bm{k}}t)} - \hat{a}_{\bm{k}\sigma}^{\dag}\exp\qty{-\i\qty(\bm{k}\cdot\bm{r} - \omega_{\bm{k}}t)}] \\ 
    \hat{B}(\bm{r}, t) &= \i\qty(\frac{1}{2\pi})^{3/2}\int\dd[3]{k}\sum_{\sigma = 1}^{2}\sqrt{\frac{\hbar}{2\epsilon_0\omega_{\bm{k}}}}\bm{k}\times\bm{e}_{\bm{k}\sigma}\qty[\hat{a}_{\bm{k}\sigma}\exp\qty{\i\qty(\bm{k}\cdot\bm{r} - \omega_{\bm{k}}t)} - \hat{a}_{\bm{k}\sigma}^{\dag}\exp\qty{-\i\qty(\bm{k}\cdot\bm{r} - \omega_{\bm{k}}t)}]
  \end{align}
  と書ける.
\end{document} % 場の量子論を頑張って勉強するのでちょっとまって plz
    \section{ビームスプリッタ}
      \documentclass{report}
\input{../../head.tex}
\begin{document}
  \subsection{電磁場のハミルトニアン}
    前節での議論により,系のハミルトニアンは,
    \begin{align}
      \hat{H}_{\r{sys}} = \int\dd[3]{k} \sum_{\sigma = 1}^{2}\frac{\hbar\omega_{\bm{k}}}{2}\qty(\hat{a}_{\bm{k\sigma}}^{\dag}\hat{a}_{\bm{k}\sigma} + \hat{a}_{\bm{k}\sigma}\hat{a}_{\bm{k}\sigma}^{\dag})
    \end{align}
    と書けるのであった.
    以下では,簡単のために,1方向成分・シングルモードの波を考える.
    \begin{align}
      \hat{H}_{\r{sys}} &= \frac{\hbar\omega}{2}\qty(\hat{a}^{\dag}\hat{a} + \hat{a}\hat{a}^{\dag}) \\ 
      &= \hbar\omega\qty(\hat{a}^{\dag}\hat{a} + \frac{1}{2})
    \end{align}
    と書ける.
    屈折率が$n$の物質中では\footnote{謎である.屈折率により波動は変化しないはずである.},
    \begin{align}
      \hat{H}_{n, \r{sys}} = \frac{\hbar\omega}{n}\qty(\hat{a}^{\dag}\hat{a} + \frac{1}{2})
    \end{align}
    と書ける.
  \subsection{ユニタリ行列の分解}\label{unitary-transform}
    ユニタリ行列は一般に,
    \begin{align}
      U = \e^{\i\Lambda/2}
      \mqty(
        \e^{\i\Psi/2} & 0 \\ 
        0 & \e^{-\i\Psi/2}
      )
      \mqty(
        \cos(\Theta / 2) & \sin(\Theta / 2) \\ 
        -\sin(\Theta / 2) & \cos(\Theta / 2)
      )\mqty(
        \e^{\i\Phi/2} & 0 \\ 
        0 & \e^{-\i\Phi/2}
      )
    \end{align}
    と分解できる.
    具体的に$U$を計算すると,
    \begin{align}
      U &= \e^{\i\Lambda/2}
      \mqty(
        \e^{\i\Psi/2} & 0 \\ 
        0 & \e^{-\i\Psi/2}
      )
      \mqty(
        \cos(\Theta / 2) & \sin(\Theta / 2) \\ 
        -\sin(\Theta / 2) & \cos(\Theta / 2)
      )\mqty(
        \e^{\i\Phi/2} & 0 \\ 
        0 & \e^{-\i\Phi/2}
      ) \\ 
      &= \e^{\i\Lambda/2}\mqty(
        \e^{\i\Psi/2}\cos(\Theta / 2) & \e^{\i\Psi/2}\sin(\Theta / 2) \\ 
        -\e^{-\i\Psi/2}\sin(\Theta / 2) & \e^{-\i\Psi/2}\cos(\Theta / 2)
      )\mqty(
        \e^{\i\Phi/2} & 0 \\ 
        0 & \e^{-\i\Phi/2}
      ) \\ 
      &= \e^{\i\Lambda/2}\mqty(
        \e^{\i(\Psi + \Phi) / 2}\cos(\Theta / 2) & \e^{\i(\Psi - \Phi) / 2}\sin(\Theta / 2) \\ 
        -\e^{-\i(\Psi - \Phi) / 2}\sin(\Theta / 2) & \e^{-\i(\Psi + \Phi) / 2}\cos(\Theta / 2)
      )\label{unitary-representation}
    \end{align} 
    であり,$\alpha = \Psi + \Phi$,$\beta = \Psi - \Phi$とすると,
    \begin{align}
      U &= \e^{\i\Lambda/2}\mqty(
        \e^{\i\alpha / 2}\cos(\Theta / 2) & \e^{\i\beta / 2}\sin(\Theta / 2) \\ 
        -\e^{-\i\beta / 2}\sin(\Theta / 2) & \e^{-\i\alpha / 2}\cos(\Theta / 2)
      ) \\ 
      &= \mqty(
        \e^{\i(\Lambda + \alpha) / 2}\cos(\Theta / 2) & \e^{\i(\Lambda + \beta) / 2}\sin(\Theta / 2) \\ 
        -\e^{\i(\Lambda - \beta) / 2}\sin(\Theta / 2) & \e^{\i(\Lambda - \alpha) / 2}\cos(\Theta / 2)
      ) 
    \end{align}
    と書ける.
    \begin{proof}
      任意$2\times 2$の行列は,実数$r_{ij}$と$\theta_{ij}$を用いて,
      \begin{align}
        M = \mqty(
          r_{11}\e^{\i\theta_{11}} & r_{12}\e^{\i\theta_{12}} \\ 
          r_{21}\e^{\i\theta_{21}} & r_{22}\e^{\i\theta_{22}} \\ 
        )
      \end{align}
      と書けて,
      \begin{align}
        M^{\dag}M &= \mqty(
          r_{11}\e^{-\i\theta_{11}} & r_{21}\e^{-\i\theta_{21}} \\ 
          r_{12}\e^{-\i\theta_{12}} & r_{22}\e^{-\i\theta_{22}} \\ 
        )\mqty(
          r_{11}\e^{\i\theta_{11}} & r_{12}\e^{\i\theta_{12}} \\ 
          r_{21}\e^{\i\theta_{21}} & r_{22}\e^{\i\theta_{22}} \\ 
        ) \\ 
        &= \mqty(
          r_{11}^2 + r_{21}^2 & r_{11}r_{12}\e^{-\i(\theta_{11} - \theta_{12})} + r_{21}r_{22}\e^{-\i(\theta_{21} - \theta_{22})} \\ 
          r_{11}r_{12}\e^{\i(\theta_{11} - \theta_{12})} + r_{21}r_{22}\e^{\i(\theta_{21} - \theta_{22})} & r_{12}^2 + r_{22}^2 \\ 
        ) \\ 
        MM^{\dag} &= \mqty(
          r_{11}\e^{\i\theta_{11}} & r_{12}\e^{\i\theta_{12}} \\ 
          r_{21}\e^{\i\theta_{21}} & r_{22}\e^{\i\theta_{22}} \\ 
        )\mqty(
          r_{11}\e^{-\i\theta_{11}} & r_{21}\e^{-\i\theta_{21}} \\ 
          r_{12}\e^{-\i\theta_{12}} & r_{22}\e^{-\i\theta_{22}} \\ 
        ) \\ 
        &= \mqty(
          r_{11}^2 + r_{12}^2 & r_{11}r_{21}\e^{\i(\theta_{11} - \theta_{21})} + r_{11}r_{22}\e^{\i(\theta_{12} - \theta_{22})} \\ 
          r_{11}r_{21}\e^{-\i(\theta_{11} - \theta_{21})} + r_{12}r_{22}\e^{-\i(\theta_{12} - \theta_{22})} & r_{21}^2 + r_{22}^2 \\ 
        )
      \end{align}
      となる.$M$がユニタリ行列であることの必要十分条件は,
      \begin{align}
        r_{11}^2 + r_{21}^2 = 1 \label{r11-r21}\\ 
        r_{12}^2 + r_{22}^2 = 1 \label{r12-r22}\\ 
        r_{11}^2 + r_{12}^2 = 1 \label{r11-r12}\\ 
        r_{21}^2 + r_{22}^2 = 1 \label{r21-r22}\\ 
        r_{11}r_{12}\e^{\i(\theta_{11} - \theta_{12})} + r_{21}r_{22}\e^{\i(\theta_{21} - \theta_{22})} = 0 \label{angle-1}\\ 
        r_{11}r_{21}\e^{\i(\theta_{11} - \theta_{21})} + r_{11}r_{22}\e^{\i(\theta_{12} - \theta_{22})} = 0 \label{angle-2}
      \end{align}
      である.$M^{\dag}M$や$MM^{\dag}$の非対角成分は複素共役になっていることに注意する.
      \refe{r11-r21}から\refe{r21-r22}を満たすような$r_{ij}$の組は,実数$\Theta$を用いて,
      \begin{align}
        r_{11} = r_{22} = \cos(\Theta / 2) \\ 
        r_{12} = -r_{21} = \sin(\Theta / 2)
      \end{align}
      なるものである.
      また,これらの$r_{ij}$の値を\refe{angle-1}と\refe{angle-2}に代入すると,
      \begin{align}
        \e^{\i(\theta_{11} - \theta_{12})} - \e^{\i(\theta_{21} - \theta_{22})} = 0 \\ 
        -\e^{\i(\theta_{11} - \theta_{21})} + \e^{\i(\theta_{12} - \theta_{22})} = 0
      \end{align}
      が成立する.
      \begin{align}
        \Phi = \theta_{11} - \theta_{12} = \theta_{21} - \theta_{22} \\ 
        \Psi = \theta_{11} - \theta_{21} = \theta_{12} - \theta_{22} \\ 
      \end{align}
      とすると,
      \begin{align}
        \theta_{11} = \frac{\Lambda + \Psi + \Phi}{2} \\ 
        \theta_{12} = \frac{\Lambda + \Psi - \Phi}{2} \\ 
        \theta_{21} = \frac{\Lambda - \Psi + \Phi}{2} \\ 
        \theta_{22} = \frac{\Lambda - \Psi - \Phi}{2}
      \end{align}
      となり,\refe{unitary-representation}を得る.
      つまり,任意のユニタリ行列は\refe{unitary-representation}で書けることが示された.
    \end{proof}
    実際に\refe{unitary-representation}がユニタリ行列であることを確かめる.
    \begin{align}
      U^{\dag}U &= \e^{-\i\Lambda/2}\mqty(
        \e^{-\i\alpha / 2}\cos(\Theta / 2) & -\e^{\i\beta / 2}\sin(\Theta / 2) \\ 
        \e^{-\i\beta / 2}\sin(\Theta / 2) & \e^{\i\alpha / 2}\cos(\Theta / 2)
      )
      \e^{\i\Lambda/2}\mqty(
        \e^{\i\alpha / 2}\cos(\Theta / 2) & \e^{\i\beta / 2}\sin(\Theta / 2) \\ 
        -\e^{-\i\beta / 2}\sin(\Theta / 2) & \e^{-\i\alpha / 2}\cos(\Theta / 2)
      )
      = \mqty(1 & 0 \\ 0 & 1) \\ 
      UU^{\dag} &= \e^{\i\Lambda/2}\mqty(
        \e^{\i\alpha / 2}\cos(\Theta / 2) & \e^{\i\beta / 2}\sin(\Theta / 2) \\ 
        -\e^{-\i\beta / 2}\sin(\Theta / 2) & \e^{-\i\alpha / 2}\cos(\Theta / 2)
      )
      \e^{-\i\Lambda/2}\mqty(
        \e^{-\i\alpha / 2}\cos(\Theta / 2) & -\e^{\i\beta / 2}\sin(\Theta / 2) \\ 
        \e^{-\i\beta / 2}\sin(\Theta / 2) & \e^{\i\alpha / 2}\cos(\Theta / 2)
      )
      = \mqty(1 & 0 \\ 0 & 1)
    \end{align}
    となり,$U$はユニタリ行列であることが分かる.
  \subsection{ビームスプリッタ行列}
    2入力2出力のビームスプリッタを考える.
    $E_1$と$E_2$の電場が入射して,$E'_1$と$E'_2$が出力されるとする.
    古典的に考えると,
    \begin{align}
      \mqty(E'_1 \\ E'_2) = \mqty(\xmat*{B}{2}{2})\mqty(E_1 \\ E_2)\label{input-classical}
    \end{align}
    と書ける.
    このまま電場演算子を中心に議論を進めることはいささか冗長である.
    なぜならば,$\hat{a}_1$と$\hat{a}_1^{\dag}$は複素共役の関係にあるのだから,片方が定まれば自然ともう片方が定まるからだ.
    よって\refe{input-classical}を量子化して,消滅演算子$\hat{a}_1$,$\hat{a}_2$を用いて表せば,
    \begin{align}
      \mqty(\hat{a}_1' \\ \hat{a}_2') = \mqty(\xmat*{B}{2}{2})\mqty(\hat{a}_1 \\ \hat{a}_2)
    \end{align}
    と書ける.
    2つの消滅演算子の交換関係は,
    \begin{align}
      \qty[\hat{a}_i, \hat{a}_j^{\dag}] = \delta_{j}^{i} \\ 
      \qty[\hat{a}_i, \hat{a}_j] = 0
    \end{align}
    である.
    $B$はビームスプリッタ行列という.
    光子数が保存することから,
    \begin{align}
      \hat{a}_1^{\dag}\hat{a}_1 + \hat{a}_2^{\dag}\hat{a}_2 &= \hat{a}_1'^{\dag}\hat{a}_1' + \hat{a}_2'^{\dag}\hat{a}_2' \\
      &= \mqty(B_{11}\hat{a}_1 + B_{12}\hat{a}_2)^{\dag}\mqty(B_{11}\hat{a}_1 + B_{12}\hat{a}_2) + \mqty(B_{21}\hat{a}_1 + B_{22}\hat{a}_2)^{\dag}\mqty(B_{21}\hat{a}_1 + B_{22}\hat{a}_2) \\ 
      &= \mqty(B_{11}^*\hat{a}_1^{\dag} + B_{12}^*\hat{a}_2^{\dag})\mqty(B_{11}\hat{a}_1 + B_{12}\hat{a}_2) + \mqty(B_{21}^*\hat{a}_1^{\dag} + B_{22}^*\hat{a}_2^{\dag})\mqty(B_{21}\hat{a}_1 + B_{22}\hat{a}_2) \\ 
      &= \mqty(\abs{B_{11}}^2 + \abs{B_{21}}^2)\hat{a}_1^{\dag}\hat{a}_1 + \mqty(\abs{B_{12}}^2 + \abs{B_{22}}^2)\hat{a}_2^{\dag}\hat{a}_2 + \mqty(B_{11}^*B_{12} + B_{21}^*B_{22})\hat{a}_1^{\dag}\hat{a}_2 + \qty(B_{12}^*B_{11} + B_{21}^*B_{21})\hat{a}_2^{\dag}\hat{a}_1 \\ 
      &= \mqty(\abs{B_{11}}^2 + \abs{B_{21}}^2)\hat{a}_1^{\dag}\hat{a}_1 + \mqty(\abs{B_{12}}^2 + \abs{B_{22}}^2)\hat{a}_2^{\dag}\hat{a}_2 + \mqty(B_{11}^*B_{12} + B_{21}^*B_{22})\hat{a}_1^{\dag}\hat{a}_2 + \mqty(B_{11}^*B_{12} + B_{21}^*B_{22})^*\hat{a}_2^{\dag}\hat{a}_1
    \end{align}
    となり,
    \begin{align}
      \begin{dcases}
        \abs{B_{11}}^2 + \abs{B_{21}}^2 = \abs{B_{12}}^2 + \abs{B_{22}}^2 = 1 \\ 
        B_{11}^*B_{12} + B_{21}^*B_{22} = 0\\ 
      \end{dcases} \\ 
      \Leftrightarrow 
      B^{\dag}B = \mqty(B_{11}^* & B_{21}^* \\ B_{12}^* & B_{22}^*)\mqty(B_{11} & B_{12} \\ B_{21} & B_{22}) = \mqty(1 & 0 \\ 0 & 1)
    \end{align}
    となればよい.
    つまり,ビームスプリッタ行列$B$がユニタリ行列であれば良い.
    \ref{unitary-transform}での議論によりビームスプリッタ演算子は,
    \begin{align}
      B = \e^{\i\Lambda/2}
      \mqty(
        \e^{\i\Psi/2} & 0 \\ 
        0 & \e^{-\i\Psi/2}
      )
      \mqty(
        \cos(\Theta / 2) & \sin(\Theta / 2) \\ 
        -\sin(\Theta / 2) & \cos(\Theta / 2)
      )\mqty(
        \e^{\i\Phi/2} & 0 \\ 
        0 & \e^{-\i\Phi/2}
      )
    \end{align}
    と書ける.
    ところが,
    \begin{align}
      \mqty(\e^{\i\Psi/2} & 0 \\ 0 & \e^{-\i\Psi/2})
    \end{align}
    は2つの入力電場$E_1$,$E_2$に位相差をかけること,
    \begin{align}
      \mqty(\e^{\i\Phi/2} & 0 \\ 0 & \e^{-\i\Phi/2})
    \end{align}
    は2つの出力電場$E_1'$,$E_2'$に位相差をかけること,
    \begin{align}
      \e^{\i\Lambda/2}
    \end{align}
    は2つの出力電場場$E_1'$,$E_2'$に共通するグローバル位相を書けることに対応するから,実験のセットアップとして,
    \begin{align}
      \Lambda = \Psi = \Phi = 0
    \end{align}
    とすることができる.
    また,透過率$T$と反射率$R$を,
    \begin{align}
      \sqrt{T} &\coloneqq \cos(\Theta / 2) \\ 
      \sqrt{R} &\coloneqq -\sin(\Theta / 2)
    \end{align}
    と定義すれば,ビームスプリッタ行列$B$は,
    \begin{align}
      B &= \mqty(
        \cos(\Theta / 2) & \sin(\Theta / 2) \\ 
        -\sin(\Theta / 2) & \cos(\Theta / 2)
      ) \\ 
      &= \mqty(\sqrt{T} & -\sqrt{R} \\ \sqrt{R} & \sqrt{T})\label{beam-splitter-no-delay}
    \end{align}
    と書ける.
    \begin{align}
      T + R = 1
    \end{align}
    が成立することに注意する.
  \subsection{Baker-Campbell-Hausdorffの公式}
    次小節以降で頻出するBaker-Campbell-Hausdorffの公式を示しておこう.
    Baker-Campbell-Hausdorffの公式は,
    \begin{align}
      \e^{\hat{A}}\hat{B}\e^{-\hat{A}} = B + \qty[\hat{A}, \hat{B}] + \frac{1}{2!}\qty[\hat{A}, \qty[\hat{A}, \hat{B}]] + \cdots
    \end{align}
    なる式である.
    \begin{proof}
      函数$f(t)$を,
      \begin{align}
        f(t) \coloneqq \e^{t\hat{A}}\hat{B}\e^{-t\hat{A}}
      \end{align}
      と定義する.
      $f(t)$を$t = 0$の周りで展開することを考えると,
      \begin{align}
        f(t) = f(0) + \eval{\dv{f}{x}}_{t = 0}t + \frac{1}{2!}\eval{\dv[2]{f}{x}}_{t = 0} t^2 + \cdots\label{taylor-bch}
      \end{align}
      と書ける.
      さて,
      \begin{align}
        \dv{f}{x} &= \hat{A}\e^{t\hat{A}}B\e^{-t\hat{A}} - \e^{t\hat{A}}\hat{B}A\e^{-t\hat{A}} \\ 
        &= \e^{t\hat{A}}\hat{A}B\e^{-t\hat{A}} - \e^{t\hat{A}}\hat{B}A\e^{-t\hat{A}} \\ 
        &= \e^{t\hat{A}}\qty(\hat{A}\hat{B} - \hat{B}\hat{A})\e^{-t\hat{A}} \\ 
        &= \e^{t\hat{A}}\qty[\hat{A}, \hat{B}]\e^{-t\hat{A}}\label{1st-diff-bch}
      \end{align}
      である.
      よって,
      \begin{align}
        \eval{\dv{f}{x}}_{t = 0} = \qty[\hat{A}, \hat{B}]
      \end{align}
      である.
      2階以上の微分では,\refe{1st-diff-bch}において,$\hat{B} \to \qty[\hat{A}, \hat{B}]$とすればよい.
      よって,\refe{taylor-bch}に\refe{1st-diff-bch}を代入すると,
      \begin{align}
        f(t) = B + \qty[\hat{A}, \hat{B}]t + \frac{1}{2!}\qty[\hat{A}, \qty[\hat{A}, \hat{B}]]t^2 + \cdots
      \end{align}
      である.
      $t = 1$とすれば,
      \begin{align}
        \e^{\hat{A}}\hat{B}\e^{\hat{A}} = B + \qty[\hat{A}, \hat{B}] + \frac{1}{2!}\qty[\hat{A}, \qty[\hat{A}, \hat{B}]] + \cdots
      \end{align}
      となる.
    \end{proof}
    また,
    % https://www2.yukawa.kyoto-u.ac.jp/~akio.tomiya/filebox/Campbell-Baker-Hausdorff.pdf
    % \begin{align}
      
    % \end{align}
    \begin{align}
      \e^{\hat{A}}\hat{B}\
    \end{align}
  \subsection{ビームスプリッタハミルトニアン}
    ビームスプリッタ行列を再び考えよう.
    今度は入力電場と出力電場の位相差が存在することにして,$\Lambda = 0$のみ課しておく.
    するとビームスプリッタ行列は,
    \begin{align}
      B &= \mqty(
        \e^{\i(\Psi + \Phi) / 2}\cos(\Theta / 2) & \e^{\i(\Psi - \Phi) / 2}\sin(\Theta / 2) \\ 
        -\e^{-\i(\Psi - \Phi) / 2}\sin(\Theta / 2) & \e^{-\i(\Psi + \Phi) / 2}\cos(\Theta / 2)
      )
    \end{align}
    と書ける.
    ビームスプリッタ行列を用いて,
    \begin{align}
      \mqty(\hat{a}_1' \\ \hat{a}_2')
      &= \mqty(
        \e^{\i(\Psi + \Phi) / 2}\cos(\Theta / 2) & \e^{\i(\Psi - \Phi) / 2}\sin(\Theta / 2) \\ 
        -\e^{-\i(\Psi - \Phi) / 2}\sin(\Theta / 2) & \e^{-\i(\Psi + \Phi) / 2}\cos(\Theta / 2) \\ 
      )\mqty(\hat{a}_1 \\ \hat{a}_2) \\ 
      &= \mqty(
        \e^{\i(\Psi + \Phi) / 2}\cos(\Theta / 2)\hat{a}_1 + \e^{\i(\Psi - \Phi) / 2}\sin(\Theta / 2)\hat{a}_2 \\ 
        -\e^{-\i(\Psi - \Phi) / 2}\sin(\Theta / 2)\hat{a}_1 + \e^{-\i(\Psi + \Phi) / 2}\cos(\Theta / 2)\hat{a}_2 \\ 
      ) \\ 
      &= \mqty(
        \e^{\i(\Psi + \Phi) / 2}\sqrt{T}\hat{a}_1 - \e^{\i(\Psi - \Phi) / 2}\sqrt{R}\hat{a}_2 \\ 
        \e^{-\i(\Psi - \Phi) / 2}\sqrt{R}\hat{a}_1 + \e^{-\i(\Psi + \Phi) / 2}\sqrt{T}\hat{a}_2 \\ 
      )
    \end{align}
    と書ける.
    出力それぞれでの光の強度は,$\hat{a}_1$と$\hat{a}_2^{\dag}$や$\hat{a}_1^{\dag}$と$\hat{a}_2$が交換することを思い出せば,
    \begin{align}
      \hat{a}_1'^{\dag}\hat{a}_1' &= \qty(\e^{\i(\Psi + \Phi) / 2}\sqrt{T}\hat{a}_1 - \e^{\i(\Psi - \Phi) / 2}\sqrt{R}\hat{a}_2)^{\dag}\qty(\e^{\i(\Psi + \Phi) / 2}\sqrt{T}\hat{a}_1 - \e^{\i(\Psi - \Phi) / 2}\sqrt{R}\hat{a}_2) \\ 
      &= T\hat{a}_1^{\dag}\hat{a}_1 + R \hat{a}_2^{\dag}\hat{a}_2 - \sqrt{T}\sqrt{R}\qty(\e^{\i\Phi}\hat{a}_1\hat{a}_2^{\dag} + \e^{-\i\Phi}\hat{a}_1^{\dag}\hat{a}_2)\label{power-a1} \\ 
      \hat{a}_2'^{\dag}\hat{a}_2' &= \qty(\e^{-\i(\Psi - \Phi) / 2}\sqrt{R}\hat{a}_1 + \e^{-\i(\Psi + \Phi) / 2}\sqrt{T}\hat{a}_2)^{\dag}\qty(\e^{-\i(\Psi - \Phi) / 2}\sqrt{R}\hat{a}_1 + \e^{-\i(\Psi + \Phi) / 2}\sqrt{T}\hat{a}_2) \\ 
      &= R\hat{a}_1^{\dag}\hat{a}_1 + T\hat{a}_2^{\dag}\hat{a}_2 + \sqrt{T}\sqrt{R}\qty(\e^{\i\Phi}\hat{a}_1\hat{a}_2^{\dag} + \e^{-\i\Phi}\hat{a}_1^{\dag}\hat{a}_2)\label{power-a2}
    \end{align}
    となる.
    \refe{power-a1}と\refe{power-a2}について,第1項と第2項はそれぞれモード1の入力光子数,モード2の入力光子数に対応する.
    これらの重ね合わせに依って位相が変化して,そのパラメータは$T$である.
    相互作用を表す項は第3項であるから,ビームスプリッタによる相互作用ハミルトニアン$\hat{H}_{\r{int}}$を,
    \begin{align}
      \hat{H}_{\r{int}} \coloneqq \frac{1}{2}\qty(\e^{\i\Phi}\hat{a}_1\hat{a}_2^{\dag} + \e^{-\i\Phi}\hat{a}_1^{\dag}\hat{a}_2)
    \end{align}
    と定義する.
    \par
    また,以下の演算子を定義する.
    \begin{align}
      \hat{L}_0 &\coloneqq \frac{1}{2}\qty(\hat{a}_1^{\dag}\hat{a}_1 + \hat{a}_2^{\dag}\hat{a}_2) \\ 
      \hat{L}_1 &\coloneqq \frac{1}{2}\qty(\hat{a}_1^{\dag}\hat{a}_2 + \hat{a}_1\hat{a}_2^{\dag}) \\ 
      \hat{L}_2 &\coloneqq \frac{1}{2\i}\qty(\hat{a}_1^{\dag}\hat{a}_2 - \hat{a}_1\hat{a}_2^{\dag}) \\ 
      \hat{L}_3 &\coloneqq \frac{1}{2}\qty(\hat{a}_1^{\dag}\hat{a}_1 - \hat{a}_2^{\dag}\hat{a}_2) 
    \end{align}
    $\hat{L}_2$と$\hat{H}_{\r{int}}$の関係を調べよう.
    唐突だが,
    \begin{align}
      \e^{-\i\Theta\hat{L}_2}\mqty(\hat{a}_1 \\ \hat{a}_2)\e^{\i\Theta\hat{L}_2} \label{ml2a1a2ml2}
    \end{align}
    考える.
    \refe{ml2a1a2ml2}の第1成分について,Baker-Campbell-Hausdorffの公式より,
    \begin{align}
      \e^{-\i\Theta\hat{L}_2}\hat{a}_1\e^{\i\Theta\hat{L}_2} &= \hat{a}_1 + \qty[-\i\Theta\hat{L}_2, \hat{a}_1] + \frac{1}{2!}\qty[-\i\Theta\hat{L}_2, \qty[-\i\Theta\hat{L}_2, \hat{a}_1]] \notag\\ 
      &+ \frac{1}{3!}\qty[-\i\Theta\hat{L}_2, \qty[-\i\Theta\hat{L}_2, \qty[-\i\Theta\hat{L}_2, \hat{a}_1]]] + \frac{1}{4!}\qty[-\i\Theta\hat{L}_2, \qty[-\i\Theta\hat{L}_2, \qty[-\i\Theta\hat{L}_2, \qty[-\i\Theta\hat{L}_2, \hat{a}_1]]]] + \cdots \\ 
      &= \hat{a}_1 + \qty(-\i\Theta)\qty[\hat{L}_2, \hat{a}_1] + \frac{\qty(-\i\Theta)^2}{2!}\qty[\hat{L}_2, \qty[\hat{L}_2, \hat{a}_1]]\notag \\ 
      &+ \frac{\qty(-\i\Theta)^3}{3!}\qty[\hat{L}_2, \qty[\hat{L}_2, \qty[\hat{L}_2, \hat{a}_1]]] + \frac{\qty(-\i\Theta)^4}{4!}\qty[\hat{L}_2, \qty[\hat{L}_2, \qty[\hat{L}_2, \qty[\hat{L}_2, \hat{a}_1]]]] + \cdots \label{etl2-a1}
    \end{align}
    となる.$\hat{L}_2$と$\hat{a}_1$,$\hat{L}_2$と$\hat{a}_2$との交換関係についてそれぞれ,
    \begin{align}
      \qty[\hat{L}_2, \hat{a}_1] &= \qty[\frac{1}{2\i}\qty(\hat{a}_1^{\dag}\hat{a}_2 - \hat{a}_1\hat{a}_2^{\dag}), \hat{a}_1] \\ 
      &= \frac{1}{2\i}\qty(\hat{a}_2\qty[\hat{a}_1^{\dag}, \hat{a}_1] - \hat{a}_2^{\dag}\qty[\hat{a}_1, \hat{a}_1]) \\ 
      &= -\frac{1}{2\i}\hat{a}_2 \label{commute-l2a1}\\ 
      \qty[\hat{L}_2, \hat{a}_2] &= \qty[\frac{1}{2\i}\qty(\hat{a}_1^{\dag}\hat{a}_2 - \hat{a}_1\hat{a}_2^{\dag}), \hat{a}_2] \\ 
      &= \frac{1}{2\i}\qty(\hat{a}_1^{\dag}\qty[\hat{a}_2, \hat{a}_2] - \hat{a}_1\qty[\hat{a}_2^{\dag}, \hat{a}_2]) \\ 
      &= \frac{1}{2\i}\hat{a}_1 \label{commute-l2a2}
    \end{align}
    となる.ただし,$\hat{a}_1$と$\hat{a}_2$が交換することを用いた.
    よって,
    \begin{align}
      \qty[\hat{L}_2, \hat{a}_1] = -\qty(\frac{1}{2\i})^1\hat{a}_2 \\ 
      \qty[\hat{L}_2, \qty[\hat{L}_2, \hat{a}_1]] = -\frac{1}{2\i}\qty[\hat{L}_2, \hat{a}_2] = -\qty(\frac{1}{2\i})^2\hat{a}_1 \\ 
      \qty[\hat{L}_2, \qty[\hat{L}_2, \qty[\hat{L}_2, \hat{a}_1]]] = -\qty(\frac{1}{2\i})^2\qty[\hat{L}_2, \hat{a}_1] = \qty(\frac{1}{2\i})^3\hat{a}_2 \\ 
      \qty[\hat{L}_2, \qty[\hat{L}_2, \qty[\hat{L}_2, \qty[\hat{L}_2, \hat{a}_1]]]] = \qty(\frac{1}{2\i})^3\qty[\hat{L}_2, \hat{a}_2] = \qty(\frac{1}{2\i})^4\hat{a}_1 
    \end{align}
    であるから\refe{etl2-a1}は,
    \begin{align}
      \e^{-\i\Theta\hat{L}_2}\hat{a}_1\e^{\i\Theta\hat{L}_2} &= \hat{a}_1 + \qty(-\i\Theta)\qty[\hat{L}_2, \hat{a}_1] + \frac{\qty(-\i\Theta)^2}{2!}\qty[\hat{L}_2, \qty[\hat{L}_2, \hat{a}_1]]\notag \\ 
      &+ \frac{\qty(-\i\Theta)^3}{3!}\qty[\hat{L}_2, \qty[\hat{L}_2, \qty[\hat{L}_2, \hat{a}_1]]] + \frac{\qty(-\i\Theta)^4}{4!}\qty[\hat{L}_2, \qty[\hat{L}_2, \qty[\hat{L}_2, \qty[\hat{L}_2, \hat{a}_1]]]] + \cdots \\ 
      &= \hat{a}_1 + \qty(-\i\Theta)\qty(-1)\qty(\frac{1}{2\i})^1\hat{a}_2 + \frac{\qty(-\i\Theta)^2}{2!}\qty(-1)\qty(\frac{1}{2\i})^2\hat{a}_1 + \frac{\qty(-\i\Theta)^3}{3!}\qty(\frac{1}{2\i})^3\hat{a}_2 + \frac{\qty(-\i\Theta)^4}{4!}\qty(\frac{1}{2\i})^4\hat{a}_1 + \cdots \\ 
      &= \hat{a}_1 + \qty(\frac{\Theta}{2})^1\hat{a}_2 - \frac{1}{2!}\qty(\frac{\Theta}{2})^2\hat{a}_1 - \frac{1}{3!}\qty(\frac{\Theta}{2})^3\hat{a}_2 + \frac{1}{4!}\qty(\frac{\Theta}{2})^4\hat{a}_1 + \cdots \\ 
      &= \qty[1 - \frac{1}{2!}\qty(\frac{\Theta}{2})^2 + \frac{1}{4!}\qty(\frac{\Theta}{2})^4 - \cdots]\hat{a}_1 + \qty[\qty(\frac{\Theta}{2})^1 - \frac{1}{3!}\qty(\frac{\Theta}{2})^3 + \cdots]\hat{a}_2 \\ 
      &= \cos(\Theta / 2)\hat{a}_1 + \sin(\Theta / 2)\hat{a}_2
    \end{align}
    となる.
    同様に,\refe{ml2a1a2ml2}の第2成分について,
    \begin{align}
      \qty[\hat{L}_2, \hat{a}_2] = \qty(\frac{1}{2\i})^1\hat{a}_1 \\ 
      \qty[\hat{L}_2, \qty[\hat{L}_2, \hat{a}_2]] = \frac{1}{2\i}\qty[\hat{L}_2, \hat{a}_1] = -\qty(\frac{1}{2\i})^2\hat{a}_2 \\ 
      \qty[\hat{L}_2, \qty[\hat{L}_2, \qty[\hat{L}_2, \hat{a}_2]]] = -\qty(\frac{1}{2\i})^2\qty[\hat{L}_2, \hat{a}_2] = -\qty(\frac{1}{2\i})^3\hat{a}_1 \\ 
      \qty[\hat{L}_2, \qty[\hat{L}_2, \qty[\hat{L}_2, \qty[\hat{L}_2, \hat{a}_2]]]] = -\qty(\frac{1}{2\i})^3\qty[\hat{L}_2, \hat{a}_1] = \qty(\frac{1}{2\i})^4\hat{a}_2
    \end{align}
    なる関係を用いると,
    \begin{align}
      \e^{-\i\Theta\hat{L}_2}\hat{a}_2\e^{\i\Theta\hat{L}_2} &= \hat{a}_2 + \qty(-\i\Theta)\qty[\hat{L}_2, \hat{a}_2] + \frac{\qty(-\i\Theta)^2}{2!}\qty[\hat{L}_2, \qty[\hat{L}_2, \hat{a}_2]]\notag \\ 
      &+ \frac{\qty(-\i\Theta)^3}{3!}\qty[\hat{L}_2, \qty[\hat{L}_2, \qty[\hat{L}_2, \hat{a}_2]]] + \frac{\qty(-\i\Theta)^4}{4!}\qty[\hat{L}_2, \qty[\hat{L}_2, \qty[\hat{L}_2, \qty[\hat{L}_2, \hat{a}_2]]]] + \cdots \\ 
      &= \hat{a}_2 + \qty(-\i\Theta)\qty(\frac{1}{2\i})^1\hat{a}_1 + \frac{\qty(-\i\Theta)^2}{2!}\qty(-1)\qty(\frac{1}{2\i})^2\hat{a}_2 + \frac{\qty(-\i\Theta)^3}{3!}\qty(-1)\qty(\frac{1}{2\i})^3\hat{a}_1 + \frac{\qty(-\i\Theta)^4}{4!}\qty(\frac{1}{2\i})^4\hat{a}_2 + \cdots \\ 
      &= \hat{a}_2 - \qty(\frac{\Theta}{2})^1\hat{a}_1 - \frac{1}{2!}\qty(\frac{\Theta}{2})^2\hat{a}_2 + \frac{1}{3!}\qty(\frac{\Theta}{2})^3\hat{a}_1 + \frac{1}{4!}\qty(\frac{\Theta}{2})^4\hat{a}_2 + \cdots \\ 
      &= -\qty[\qty(\frac{\Theta}{2})^1 - \frac{1}{3!}\qty(\frac{\Theta}{2})^3 + \cdots]\hat{a}_1 + \qty[1 - \frac{1}{2!}\qty(\frac{\Theta}{2})^2 + \frac{1}{4!}\qty(\frac{\Theta}{2})^4 - \cdots]\hat{a}_2 \\ 
      &= -\sin(\Theta / 2)\hat{a}_1 + \cos(\Theta / 2)\hat{a}_2
    \end{align}
    である.
    よって,\refe{ml2a1a2ml2}は,
    \begin{align}
      \e^{-\i\Theta\hat{L}_2}\mqty(\hat{a}_1 \\ \hat{a}_2)\e^{\i\Theta\hat{L}_2} = \mqty(
        \cos(\Theta / 2) & \sin(\Theta / 2) \\ 
        -\sin(\Theta / 2) & \cos(\Theta / 2)
      )\mqty(\hat{a}_1 \\ \hat{a}_2)\label{beam-splitter-hamiltonian-prep}
    \end{align}
    と書ける.
    \refe{beam-splitter-hamiltonian-prep}の解釈を考えよう.
    相互作用ハミルトニアン$\hat{H}_{\r{int}}$の定義は,
    \begin{align}
      \hat{H}_{\r{int}} \coloneqq \frac{1}{2}\qty(\e^{\i\Phi}\hat{a}_1\hat{a}_2^{\dag} + \e^{-\i\Phi}\hat{a}_1^{\dag}\hat{a}_2)
    \end{align}
    であった.
    $\Phi = \pi / 2$とすると,
    \begin{align}
      \hat{H}_{\r{int}} &= \frac{1}{2}\qty(\e^{\i\pi / 2}\hat{a}_1\hat{a}_2^{\dag} + \e^{-\i\pi / 2}\hat{a}_1^{\dag}\hat{a}_2) \\ 
      &= \frac{1}{2}\qty(\i\hat{a}_1\hat{a}_2^{\dag} - \i\hat{a}_1^{\dag}\hat{a}_2) \\ 
      &= \frac{1}{2\i}\qty(\hat{a}_1^{\dag} - \hat{a}_1\hat{a}_2^{\dag}) \\ 
      &= \hat{L}_2
    \end{align}
    と書ける.
    さらに,\refe{beam-splitter-hamiltonian-prep}において,$\Theta = -t / \hbar$とすれば,
    \begin{align}
      \exp\qty(-\i\frac{\hat{H}_{\r{int}}}{\hbar}t)\mqty(\hat{a}_1 \\ \hat{a}_2)\exp\qty(\i\frac{\hat{H}_{\r{int}}}{\hbar}t) = \mqty(
        \cos(-t / 2\hbar) & \sin(-t / 2\hbar) \\ 
        -\sin(-t / 2\hbar) & \cos(-t / 2\hbar)
      )\mqty(\hat{a}_1 \\ \hat{a}_2)
    \end{align}
    となる.
    左辺は,
    \begin{align}
      \mqty(\hat{a}_1 \\ \hat{a}_2)
    \end{align}
    なる消滅演算子のペアを時間発展演算子で挟んでいる格好である.
    となれば,右辺はHeisenberg描像で表した消滅演算子であろう\footnote{
      右辺に出てくる行列はビームスプリッタ行列でないことに注意する.
      確かに2つの入力電場間の位相ずれや,2つの出力電場間の位相ずれがないと仮定したとき,ビームスプリッタ演算子は\refe{beam-splitter-no-delay}と書ける.
      しかし,$\Phi = \pi / 2$なる仮定のもと議論している.
      このような入力電場の位相ずれ$\Phi$に対して,出力電場の位相ずれ$\Psi$をうまく定めれば\refe{beam-splitter-no-delay}の形を実現することができると思うかもしれないが,
      その試みははかなく終わる.
      そのような$\Psi$は,$\pi / 2 + \Psi = 2n\pi$かつ$\pi / 2 - \Psi = 2m\pi$,$n, m \in \mathbb{Z}$としなければいけないが,
      2式を足して,$\pi = 2(n + m)\pi$となり,そのような$n$,$m$は存在しない.
      要するに,\refe{beam-splitter-hamiltonian-prep}の右辺の行列はビームスプリッタ行列ではないのだ.
      なお,テキストでの(1.106)は何を言いたいのかわからない.
      }.
\end{document}
    \section{コヒーレント状態}
      \documentclass{report}
\input{../../head.tex}
\begin{document}
  本節ではコヒーレント状態について議論する.
  コヒーレント状態$\ket{\alpha}$は,
  \begin{align}
    \hat{a}\ket{\alpha} = \alpha\ket{\alpha}
  \end{align}
  なる状態である.
  また,$\alpha = \abs{\alpha}\e^{\i\theta}$となるように$\theta$を定義しておく.
  \subsection{物理量の平均値・分散}
    具体的な$\ket{\alpha}$の形を知らなくても,いくつかの物理量の平均値と分散については調べることができる.
    まず,電場の期待値を調べる.
    電場演算子$\hat{\bm{E}}(\bm{r}, t)$を自然単位系を用いて書くと,
    \begin{align}
      \frac{\i}{2}\bm{e}\qty(\hat{a}\exp\qty{\i(\bm{k}\cdot\bm{r} - \omega t)} - \hat{a}^{\dag}\exp\qty{-\i(\bm{k}\cdot\bm{r} - \omega t)})
    \end{align}
    であるから,
    \begin{align}
      \ev{\bm{E}(\bm{r}, t)} &= \mel**{\alpha}{\bm{E}(\bm{r}, t)}{\alpha} \\ 
      &= \frac{\i}{2}\bm{e}\qty(\mel**{\alpha}{\hat{a}}{\alpha}\exp\qty{\i(\bm{k}\cdot\bm{r} - \omega t)} - \mel**{\alpha}{\hat{a}^{\dag}}{\alpha}\exp\qty{-\i(\bm{k}\cdot\bm{r} - \omega t)}) \\ 
      &= -\frac{1}{2\i}\bm{e}\qty(\alpha\exp\qty{\i(\bm{k}\cdot\bm{r} - \omega t)} - \alpha^*\exp\qty{-\i(\bm{k}\cdot\bm{r} - \omega t)}) \\ 
      &= -\frac{1}{2\i}\bm{e}\qty(\abs{\alpha}\e^{\i\theta}\exp\qty{\i(\bm{k}\cdot\bm{r} - \omega t)} - \abs{\alpha}\e^{-\i\theta}\exp\qty{-\i(\bm{k}\cdot\bm{r} - \omega t)}) \\ 
      &= -\abs{\alpha}\bm{e}\sin\qty(\bm{k}\cdot\bm{r} - \omega t + \theta)
    \end{align}
    である.
    \par
    次に,位置と運動量の平均値と分散について議論する.
    位置演算子と運動量演算子は生成演算子と消滅演算子を用いて,
    \begin{align}
      \hat{x} &\coloneqq \frac{1}{2}\qty(\hat{a} + \hat{a}^{\dag}) \\ 
      \hat{p} &\coloneqq \frac{1}{2\i}\qty(\hat{a} - \hat{a}^{\dag})
    \end{align}
    と書けるから,
    \begin{align}
      \ev{x} &= \mel**{\alpha}{\frac{1}{2}\qty(\hat{a} + \hat{a}^{\dag})}{\alpha} \\ 
      &= \frac{1}{2}\qty(\alpha + \alpha^*) \\ 
      \ev{x^2} &= \mel**{\alpha}{\frac{1}{4}\qty(\hat{a} + \hat{a}^{\dag})^2}{\alpha} \\ 
      &= \frac{1}{4}\mel**{\alpha}{\hat{a}^2}{\alpha} + \frac{1}{4}\mel**{\alpha}{\hat{a}\hat{a}^{\dag}}{\alpha} + \frac{1}{4}\mel**{\alpha}{\hat{a}^{\dag}\hat{a}}{\alpha} + \frac{1}{4}\mel**{\alpha}{\qty(\hat{a}^{\dag})^2}{\alpha} \\ 
      &= \frac{1}{4}\alpha^2 + \frac{1}{4}\mel**{\alpha}{\hat{a}^{\dag}\hat{a} + 1}{\alpha} + \frac{1}{4}\abs{\alpha}^2 + \frac{1}{4}\qty(\alpha^*)^2 \\ 
      &= \frac{1}{4}\qty(\alpha + \alpha^*)^2 + \frac{1}{4} \\ 
      \ev{p} &= \mel**{\alpha}{\frac{1}{2\i}\qty(\hat{a} - \hat{a}^{\dag})}{\alpha} \\ 
      &= \frac{1}{2\i}\qty(\alpha - \alpha^*) \\ 
      \ev{p^2} &= \mel**{\alpha}{-\frac{1}{4}\qty(\hat{a} - \hat{a}^{\dag})^2}{\alpha} \\ 
      &= -\frac{1}{4}\mel**{\alpha}{\hat{a}^2}{\alpha} + \frac{1}{4}\mel**{\alpha}{\hat{a}\hat{a}^{\dag}}{\alpha} + \frac{1}{4}\mel**{\alpha}{\hat{a}^{\dag}\hat{a}}{\alpha} - \frac{1}{4}\mel**{\alpha}{\qty(\hat{a}^{\dag})^2}{\alpha} \\ 
      &= -\frac{1}{4}\alpha^2 + \frac{1}{4}\mel**{\alpha}{\hat{a}^{\dag}\hat{a} + 1}{\alpha} + \frac{1}{4}\abs{\alpha}^2 - \frac{1}{4}\qty(\abs{\alpha}^{*})^2 \\ 
      &= -\frac{1}{4}\qty(\alpha^2 + \alpha^*)^2 + \frac{1}{4}
    \end{align}
    より,
    \begin{align}
      \Delta x_{\r{coh}} &\coloneqq \sqrt{\ev{x^2} - \ev{x}^2} = \frac{1}{4} \\ 
      \Delta p_{\r{coh}} &\coloneqq \sqrt{\ev{p^2} - \ev{p}^2} = \frac{1}{4}
    \end{align}
    となる.
  \subsection{個数状態での展開}
  \subsection{時間発展}
\end{document}
    \section{スクイーズド状態}
      
\end{document}